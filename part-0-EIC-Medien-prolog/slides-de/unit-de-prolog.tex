\clearpage
%%%%%%%%%%%%%%% METADATA %%%%%%%%%%%%%%%%
\bsunitname{Einleitung}
\setcounter{bsunit}{1}
%%%%%%%%%%%%%%%%%%%%%%%%%%%%%%%%%%%%%%%%
%%% Unit statistics.
%%%
%%% corollary:   0
%%% definition:  0
%%% lemma:       0
%%% page:        0
%%% proof:       0
%%% theorem:     0
%%%%%%%%%%%%%%%%%%%%%%%%%%%%%%%%%%%%%%%%%%%%%%%%%%%%%%%%%%%%%%%%%%%%%%%%
%%% SOURCE. \cf[Kapitel 1]{Malak, Butz, Hu�mann - Einf�hrung Medieninformatik}
%%%%%%%%%%%%%%%%%%%%%%%%%%%%%%%%%%%%%%%%%%%%%%%%%%%%%%%%%%%%%%%%%%%%%%%%
\begin{bsslide}[Einleitung]
  \colortext{Technische Infrastruktur Internet}
  \bspar1
  Der erste Teil behandelt \textbf{die technische
    Infrastruktur des Internets zur \"Ubertragung von Information und Medien}
  \begin{itemize}
    \item Netzwerkschichten (e.g. ISO/OSI)
    \item Protokolle (e.g. HTTP)
    \item Beschreibungssprachen (HTML, CSS)
    \item Web-Programmiersprachen (e.g. JavaScript, PHP)
  \end{itemize}
\end{bsslide}
%%%%%%%%%%%%%%%%%%%%%%%%%%%%%%%%%%%%%%%%%%%%%%%%%%%%%%%%%%%%%%%%%%%%%%%%
\begin{bsslide}[Einleitung]
  \colortext{Medien und Information}
   \bspar1
  Der zweite Teil behandelt die \textbf{Grundlagen des Aufbaus von
    Information und digitaler Medien}.  Digitale Medien sind dabei Medien, welche in digitaler Form
    gespeichert, verarbeitet oder manipuliert werden.
  \bspar1
  Im Speziellen addressieren wir folgende Themen
  \begin{itemize}
    \item Digitalisierung und Kodierung von Information mit einem
      Schwerpunkt auf Bin\"arcodes (2 Einheiten)
    \item Informationstheorie und Kompressionsalgorithmen (2 Einheiten)
    \item Medium Bild: Rasterbilder vs. Vektorgrafiken (2 Einheiten)
  \end{itemize}
  \bspar1
   \"Ubungen:
  \begin{itemize}
    \item Digitalisierung und Kodierung von Information mit einem
      Schwerpunkt auf Bin\"arcodes (Jan. 11 - 17, keine Freitags\"ubung)
    \item Fragestunde (Jan. 18-24)
    \item Informationstheorie und Kompressionsalgorithmen (Pflichtabgabe, \textbf{Jan. 25-31}, keine Freitags\"ubung)
  \end{itemize}  
\end{bsslide}
%%%%%%%%%%%%%%%%%%%%%%%%%%%%%%%%%%%%%%%%%%%%%%%%%%%%%%%%%%%%%%%%%%%%%%%%
\begin{bsslide}[Einleitung]
  \colortext{Literatur}
  \bspar1
   \begin{itemize}
    \item Malaka, Butz, Hu\ss mann - Medieninformatik - Eine
      Einf\"uhrung, Pearson Studium
    \item Herold, Lurz, Wohlrab - Grundlagen der Informatik, Pearson Studium
  \end{itemize}
\end{bsslide}

%%% Local Variables: 
%%% mode: latex
%%% TeX-master: "part-de-EIC-medien-prolog
%%% End: 

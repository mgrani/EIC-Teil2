

\begin{exercise}{SVG Scalable Vector Graphics - 4 Pkt.}
\label{ex-de-mt-SVG}
Erkl\"aren Sie die Elemente und Attribute der folgenden SVG Grafiken und skizzieren sie das Ergebnis.

\begin{enumerate}
  \item Listing 1: Gradient
  {\small
  \lstset{language=SVG}
  \lstinputlisting{figure/gradient.svg}
  } 
  
  \answer{
  \url{http://www.w3schools.com/svg/tryit.asp?filename=trysvg_linear}
  \begin{itemize}
    \item The id attribute of the <linearGradient> tag defines a unique name for the gradient
    \item     The x1, x2, y1,y2 attributes of the <linearGradient> tag define the start and end position of the gradient
    \item The color range for a gradient can be composed of two or more colors. Each color is specified with a <stop> tag. The offset attribute is used to define where the gradient color begin and end
    \item The fill attribute links the ellipse element to the gradient
  \end{itemize}
  }
  \item Listing 2: Animation
  {\small
  \lstset{language=SVG}
  \lstinputlisting{figure/color-animation.svg}
  } 
  \answer{
  \url{http://www.w3schools.com/svg/tryit.asp?filename=animatecolor_1&type=svg}
  \begin{itemize}
    \item three rectangles that change color
    \item animateColor specified by id, change of which attribute over time (attributname,), color and time
    \item time can be set based on properties of other elements
  \end{itemize}
  }
  \item Listing 3: Filter and Definitions
  {\small
  \lstset{language=SVG}
  \lstinputlisting{figure/filter-blur.svg}
  } 
  \item \answer{
  \url{ http://www.w3schools.com/svg/tryit.asp?filename=filter_1&type=svg}
  \begin{itemize}
    \item filterUnits definiert die Art der Gr��enangabe des Filters (Prozent, Pixel etc.)
    \item feGaussianBlur Gau�-Weichzeichner
    \item in selektiert den Kanal (hier Alphakanal des gezeichneten (nicht aktuellen) Bildes)
  \end{itemize}
  }
 
  \item Listing 4: Text und Animation
  {\small
  \lstset{language=SVG}
  \lstinputlisting{figure/text-and-animation.svg}
  } 
  \answer{
  \url{http://www.w3schools.com/svg/tryit.asp?filename=trysvg_animatemotion2}
  \begin{itemize}
    \item transform attribute, translate (=Translation/Verschiebung)
    \item set attribute visbility: verz�gerte Sichtbarkeit
    \item MotionPath: Verschieben der Gruppe entlang pfad mit dauer 5s
    \item animateTransform ver�ndert ein transformationsattribut �ber die Zeit
    \begin{itemize}
      \item rotate: rotiert
      \item scale: skaliert
    \end{itemize}
  \end{itemize}
  }
\end{enumerate}

\end{exercise}


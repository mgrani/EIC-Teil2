\documentclass{bs-reading}
\usepackage{todonotes}
\usepackage{animate}
\usepackage{movie15}
\usepackage{verbatim}
\usepackage{listings}

\usepackage{color}
\definecolor{gray}{rgb}{0.4,0.4,0.4}
\definecolor{darkblue}{rgb}{0.0,0.0,0.6}
\definecolor{cyan}{rgb}{0.0,0.6,0.6}

\lstset{
  basicstyle=\ttfamily,
  columns=fullflexible,
  showstringspaces=false,
  commentstyle=\color{gray}\upshape
}

\lstdefinelanguage{SVG}
{
  morestring=[b]",
  morestring=[s]{>}{<},
  morecomment=[s]{<?}{?>},
  stringstyle=\color{black},
  identifierstyle=\color{darkblue},
  keywordstyle=\color{red},
  morekeywords={transform,stroke,path,width,height,d}% list your attributes here
}
\lstdefinelanguage{XML}
{
  morestring=[b]",
  morestring=[s]{>}{<},
  morecomment=[s]{<?}{?>},
  stringstyle=\color{black},
  identifierstyle=\color{darkblue},
  keywordstyle=\color{red},
  morekeywords={xmlns,version,type}% list your attributes here
}


\bsauthor{GRANITZER}
\bsyear{2012}

\newcommand{\outline}{
\begin{bsslide}[\bsparthead]
\bspar1
\begin{bspartenumerate}[1]
\coloritem[\emcolor]
Medientechnik - Vektorgrafik
\begin{itemize}
\small
\coloritem Vektorgrafik Allgemein
\coloritem Codierung am Beispiel Scalable Vector Graphics (SVG)
\end{itemize}
\end{bspartenumerate}
\end{bsslide} 
} 

\setcounter{corollary}{0}
\setcounter{definition}{0} 
\setcounter{lemma}{0}
\setcounter{page}{0}
\setcounter{proof}{0}
\setcounter{theorem}{0}
%% TODO: Define new Environment Lernziele


\begin{document}

\bscollection{Medientechnik}
\bspartname{Vektorgrafik und SVG}
\setcounter{bspart}{5}

\setcounter{corollary}{0}
\setcounter{definition}{0} 
\setcounter{lemma}{0}
\setcounter{page}{0}
\setcounter{proof}{0}
\setcounter{theorem}{0}

\begin{bsslide}
\bspar9
\begin{center}
{\large\bfseries Medientechnik}
\bspar4
Vektorgraphik und SVG
\bspar4
Michael Granitzer
\bspar2
Professur f�r Medieninformatik
\bspar2 
Universit�t Passau
\end{center}
\end{bsslide}

\outline 
%%% DATE.  September, 1st, 2012

\clearpage
\bsauthor{GRANITZER}
\bsyear{2012}

%%% Unit statistics.
%%%
%%% corollary:   0
%%% definition:  0
%%% lemma:       0
%%% page:        0
%%% proof:       0
%%% theorem:     0


%%%%%%%%%%%%%%%%%%%%%%%%%%%%%%%%%%%%%%%%%%%%%%%%%%%%%%%%%%%%%%%%%%%%%%%%
%%% SOURCE. \cf[Kapitel 7.1 und 7.2]{Malak, Butz, Hu�mann - Einf�hrung Medieninformatik}
%%%%%%%%%%%%%%%%%%%%%%%%%%%%%%%%%%%%%%%%%%%%%%%%%%%%%%%%%%%%%%%%%%%%%%%%


\begin{bsslide}[Vektorgrafik Allgemein]
  \colortext{Lernziel}
  \bspar1
  \textbf{Unterthemen}
  \begin{itemize}
    \item Beschreibung Vektorgraphik/Unterschied zu Rastergrafiken
    \item Koordinatensystem, Punkte, Geraden
    \item �berblick Bezier-Kurven und Splines
    \item Rendering-Pipeline
    \begin{itemize}
      \item Szenegraph und Koordinatensysteme
      \item Clipping
      \item Rasterisierung
    \end{itemize}
  \end{itemize}
\end{bsslide}

\renewcommand{\textbufferB}{Grundlegende Beschreibung von 2D Vektorgraphiken}
  
  \begin{bsslide}[\textbufferB]
    \colortext{Bilder vs. Grafiken}
    \bspar1
    \textbf{Digitales Bild} besteht aus N Zeilen und je M Bildpunkten
    Anwendungsbereiche:
    \begin{itemize}
      \item Pixeln bzw. Picture Elements (Aufl�sung, Farbtiefe etc.)
      \item Bild kann aus der realen Welt kommen oder virtuell sein
      \item Wie beschreiben wir digital erstellte Grafiken?
    \end{itemize}
    \bspar1
    \textbf{(Vektor-)Grafiken}: durch grafische Primitive und ihre Attribute spezifiziert
    \begin{itemize}
      \item Primitive 2D Objekte: Linien, Rechtecke, Kreise, Ellipsen, Texte
      \item 2D Formate: SVG,PostScript, Windoes Metafile, CorelDraw, PDF $\ldots$
      \item Primitive 3D Objekte: Polyeder, Kugeln, $\ldots$
      \item Formate 3D: VRML, X3D
      \item Attribute: Stil der Linie, Breite, Farbe etc
    \end{itemize}
  \end{bsslide}
  
  
  \begin{bsslide}[\textbufferB]
    \colortext{Vektorgrafiken}
    \bspar1
    \begin{definition}[Vektorgrafik]
      Als Vektorgrafiken bezeichnet man \textbf{mathematisch und programmatisch} definierte Zeichenanweisungen in einem Koordinatensystem,
       aus welchen Rastergrafiken generiert, gespeichert und pr�sentiert werden k�nnen.
    \end{definition}
    \begin{minipage}{0.45\linewidth}
      \bsfigurecaption[0.6]{beispiel-vektorgraphic-logo.jpg}{\tiny Bildquelle zz-logo.de}
    \end{minipage}
    \begin{minipage}{0.45\linewidth}
      \small
      \textbf{Eigenschaften}
      \begin{itemize}
        \item Vektorgrafiken k�nnen einfach und exakt geometrisch transformiert werden
        \begin{itemize}
          \item Rotation, Skalierung, Verschiebung
          \item Separierung einzelner Bildelemente (Gruppierung, Ebenen)
          \item �nderung der Attribute einzelner Elemente (Farbe einer Fl�che etc)
          \item Im Grunde ein perfektes mathematisches Modell, welches jedoch in ein Rasterbild �berf�hrt werden muss
        \end{itemize}
      \end{itemize}
    \end{minipage}
  \end{bsslide}
  
    \begin{bsslide}[\textbufferB]
    \colortext{Bilder vs. Grafiken}
    \bspar1
     Unterschiede zu Rasterbild
    \begin{itemize}
      \item Erstellungsformat - Editor, Vektorgrafik-Programm
      \item Speicherformat - XML, properit�r (nicht in bin�ren Daten)
      \item Ausgabeformat - Rendering
      \item Falls keine Konvertierung in ein Standardformat, sind je Editorensystem eigene Previewer n�tig
      \item F�r einen Ausdruck sind (fast) immer Konvertierungen n�tig (z.B. nach Postscript)
    \end{itemize}
  \end{bsslide}
  
\renewcommand{\textbufferB}{Grundlegende Beschreibung von 2D Vektorgrafiken}
  
  \begin{bsslide}[\textbufferB]
    \colortext{Koordinatensystem}
    \bspar1
    \textbf{Koordinatensystem: }Grundlage jeder 2D Vektorgrafik ist ein zweidimensionaler Vektorraum
    \begin{itemize}
      \item X/Horizontal- und Y/Vertikalachse
      \item i.A. gleicher Abstand auf beiden Achsen
    \end{itemize}
    \bspar1
    \textbf{Elemente}
    \begin{itemize}
      \item Punkt beschrieben durch X/Y Koordinate
      \item Gerade beschrieben durch Start- und Endpunkt
      \item Polygon aus mehreren Geraden
      \begin{itemize}
        \item Geschlossenes Polygon (Fl�che)
        \item Offenes Polygon
      \end{itemize}
      \item Kreis - Radius und Mittelpunkt
    \end{itemize}
  \end{bsslide}
  
  \begin{bsslide}[\textbufferB]
    \colortext{Koordinatensystem}
    \bspar1
    \bsfigurecaption[0.4]{koordinatensystem.png}{\tiny Bildquelle \cite{Malaka et.al. 2009}}
    \bspar1
    \textbf{Canvas: }Die vom Koordinatensystem aufgespannte Zeichenfl�che wird auch Canvas bezeichnet
    \begin{itemize}
      \item Je nach Format liegt der Ursprung des Koordinatensystems woanders
      \item Java/SVG/Web: Links Oben; PostScript-Standard Links unten
    \end{itemize}
  \end{bsslide}
  
  \begin{bsslide}[\textbufferB]
    \colortext{Kurven Formen im Koordinatensystem}
    \bspar1
    Wie zeichnet man Kurven?
    \bspar1
    \textbf{Interpolationskurven oder Splines} werden durch \textbf{Kontroll- oder St�tzpunkte} beschrieben
    \bspar1
    \begin{minipage}{0.45\linewidth}
      \bsfigurecaption[0.4]{splines}{\tiny Bildquelle \cite{Malaka et.al. 2009}}
    \end{minipage}
    \begin{minipage}{0.45\linewidth}
      \small
      \begin{itemize}
        \item Eine Spline Kurve $n-ten$ Grades ist st�ckweise aus Polynomen maximal $n-ten$ Grades zusammen gesetzt
        \item Angabe der Randbedingungen f�r jedes St�ck (1.-3. Grad der Ableitung) bestehend aus \textbf{Steigung, Kr�mmung und Kr�mmungs�nderung}
        \item Randpunkte eralten somit die Gl�tte und Stetigkeit
        \item Angabe in Grafikprogrammen duch \textbf{Kontrollinien}
        \begin{itemize}
          \item Richtung = Steigung
          \item L�nge = Steifigkeit/Kr�mmung
        \end{itemize}
      \end{itemize}
    \end{minipage}
  \end{bsslide}
  
  \begin{bsslide}[\textbufferB]
    \colortext{Kurven Formen im Koordinatensystem}
    \bspar1\small
    Eine \textbf{Bezier-Kurven} $n-ten$ Grades ist eine spezielle Art der Interpolationskurve welche durch $n+1$ Kontrollpunkte beschrieben wird.
    \begin{itemize}
      \item Entwickelt von Bezier und Casteljau bei Renault bzw. Citroen zur Formgebung bei Autos
      \item Der Kurvenlauf l�sst sich mit dem Algoritmus von Casteljau ermitteln
    \end{itemize}
    \bspar2
    \begin{minipage}{0.45\linewidth}
      \bsfigurecaption[0.7]{berzier-kurven}{\tiny Bildquelle \cite{Malaka et.al. 2009}}
    \end{minipage}
    \begin{minipage}{0.45\linewidth}
      \small
      \textbf{Skizze des Algorithmus:}
      \begin{itemize}
        \item[1.] Gegeben: $n+1$ Kontrollpunkte $P$ einer Berzier Kurve $n-ten$ Grades
        \item[2.] Initalisiere Laufparameter $t\in [0:1]$ mit einem kleinen Wert
        \item[3.] Setze $P'=P$
        \item[4.] Teile die durch $P'$ definierten Geraden im Verh�ltnis $t$ 
        \item[5.] Verwende die Teilungspunkte als neue Kontrollpunkte $P'$
        \item[6.] Wenn $|P'|>1$ gehe zu 4
        \item[7.] Wenn $|P'|==1$ zeichne den Punkte an Position $P'$, erh�he $t$
        \item[8.] Solange $t<1$ gehe zu 3.
      \end{itemize}
    \end{minipage}
  \end{bsslide}
  
  \begin{bsslide}[\textbufferB]
    \colortext{Kurven Formen im Koordinatensystem}
    \bspar1
    Konstruktionsbeispiel f. Kurve erster Ordnung
    \bspar4
    \animategraphics{10}{berzier-1-}{0}{50}
    \bspar1\tiny
    \url{http://en.wikipedia.org/wiki/B%C3%A9zier_curve
    }
  \end{bsslide}
  
  \begin{bsslide}[\textbufferB]
    \colortext{Kurven Formen im Koordinatensystem}
    \bspar1
    Konstruktionsbeispiele f�r Kurve 2. Ordnung
    \animategraphics{10}{bezier-2-}{0}{50}
    \bspar1\tiny
    
    \url{http://en.wikipedia.org/wiki/B%C3%A9zier_curve
    }
  \end{bsslide}
  
  \begin{bsslide}[\textbufferB]
    \colortext{Kurven Formen im Koordinatensystem}
    \bspar1
    Konstruktionsbeispiele f�r Kurve 3. Ordnung
    \bspar1
    \animategraphics{10}{bezier-4-}{0}{50}
    \bspar1\tiny
        \url{http://en.wikipedia.org/wiki/B%C3%A9zier_curve
    }
  \end{bsslide}
  
  
%   \begin{bsslide}[\textbufferB]
%     \colortext{Kurven Formen im Koordinatensystem}
%     Konstruktionsbeispiele (Animation) (Quelle Wikipedia)
%     H�herer Ordnung
%    % \animategraphics{10}{bezier-complex-}{0}{48}
%     \bspar1\tiny
%     
%     \url{http://en.wikipedia.org/wiki/B%C3%A9zier_curve
%     }
%   \end{bsslide}
  
%   
%   \begin{bsslide}[\textbufferB]
%     \colortext{Kurven Formen im Koordinatensystem}
%     \bspar1
%     Aus einer Menge solcher Berzier-Kurven lassen sich nun beliebig lange harmonisch gebogene Spline Kurven, die sogenannten B-Splines, zusammensetzen.\bspar1
%     An den Grenzpunkten wird dabei sichergestellt, dass Steigung und Kr�mmung der beiden Kurvensegmente �bereinstimmen
%     \bspar4
%     $\Rightarrow$ Details nicht teil dieser Vorlesung.
%   \end{bsslide}
%   
  \begin{bsslide}[\textbufferB]
    \colortext{Geometrische Transformationen}
    \bspar1
    Wie bei den Bildoperationen, k�nnen Punkte im Koordinatensystem geometrisch transformiert werden.
    \begin{itemize}
      \item Translation
      \item Rotation
      \item Skalierung
      \item Scherung
    \end{itemize}
    \bspar4
    Sie VO Einheit Bildoperationen
  \end{bsslide}
  
  \begin{bsslide}[\textbufferB]
    \colortext{Formate und Beispiel}
    \bspar1
    Formate:
    \begin{itemize}
      \item Vektor-Grafik-Formate
      \item PostScript (.ps, .eps). PDF
      \item Windows Metafile (*.wmf, *.emf)
      \item Corel Draw (*.cdr)
      \item ScalableVector Graphics (*.svg)
      \item VRML (3D)
      \item u.a.m.
    \end{itemize}
    \begin{minipage}{0.45\linewidth}
      \bsfigurecaption[0.7]{beispiel-vektorgrapfik-msdraw.png}{}
    \end{minipage}
    \begin{minipage}{0.45\linewidth}
      vgl. z.b. Grafikerstellung mit MS Draw
      \begin{itemize}
        \item Einzelobjekt zusammengesetzt aus verschiedenen Grundprimitiven
        \item Objektorientierte-Sichtweise
        \item Gruppierungen m�ssen m�glich sein
        \item Typische Gr��e 10-100 KByte
      \end{itemize}
    \end{minipage}
    
  \end{bsslide}
  


\renewcommand{\textbufferB}{Rendering}
  \begin{bsslide}[\textbufferB]
    \colortext{Rendering}
    \bspar1
    Wie erfolgt das Zeichnen/Darstellen einer Vektorgrafik am Ausgabeger�t?
    \bspar1
    \bsfigurecaption[0.5]{beispiel-car}{\tiny Bildquelle \cite{Malaka et.al. 2009}}
    \bspar1
    \begin{definition}[Rendering]
      Als \textbf{Rendering} bezeichnet man die Darstellung/das Zeichen von Vektorgrafiken auf Repr�sentationsmedien (meist Bildschirm).
    \end{definition}
    \bspar1
    Rendering kann als Digitalisierungsprozess von einem idealen, mathematischen Modell in dessen Darstellung angesehen werden. D.h. es k�nnen die gleichen Effekte wie bei der Digitalisierung von Licht auftreten.
    \bspar1
    Unterschiede zwischen 2D und 3D Modellen. Wir betrachten nur 2D Rendering.
  \end{bsslide}
  
  \begin{bsslide}[\textbufferB]
    \colortext{Rendering}
    \bspar1
    \textbf{Rendering Prozess}: Etablierte Abfolge von Arbeitsschritten zur Darstellung der Vektorgrafik an Raster-basierten Pr�sentationsmedien
    \bspar1
    \bsfigurecaption{rendering-pipelin}{\tiny Bildquelle \cite{Malaka et.al. 2009}}
    \begin{itemize}
      \small
      \item\textbf{ Ausgangspunkt Szenegraph}: 2D Objekte als Gruppe primitiver 2D
        Objekte
      \item Transformation der Objekte in Weltkoordinaten
      \item \textbf{Clipping:} Beschneiden des Anzeigebereichs 
      \item Transformation von Weltkoordinaten in Bildschirmkoordinanten
      \item \textbf{Rasterisierung}: Zeichen von Linien und Punkten
    \end{itemize}
  \end{bsslide}
  
  \begin{bsslide}[\textbufferB]
    \colortext{Rendering Pipeline - Transformation Bild-/Weltkoordinaten}
    \bspar1
    Die zu zeichnenden Elemente sind in dem sogenannten \textbf{Szenengraph} definiert\bspar1
    Der Szenegraph ist ein gerichteter-azyklischer Graph von darzustellenden geometrischen Objekten und kann im einfachsten Fall als Baum aufgefasst werden.
    \begin{itemize}
      \small
      \item Blatt-Knoten definieren primitive geometrische Objekte (Linien, Polygone)
      \item Inner Knoten definieren 2D Transformationen auf abh�ngige Objekte
      \item Gerichtete Kanten spezifizieren Zusammensetzung von einfacheren Objekten/Primitives zu komplexeren Objekten
      \item Eigenschaften k�nnen vererbt werden (z.B. Farbe eines Objektes)
      \item Objekte im Szenengraphen sind anhand eines Objekt-lokalen Bezugskoordinatensystems (Objektkoordinaten) definiert (z.B. Mittelpunt des Rades ist Punkt 0,0) 
      \item Objektkoordinaten m�ssen beim darstellen der Szene in das Weltkoordinatensystem �berf�hrt werden (Position des Rades in der Szene) 
      \item Das Weltkoordinatensystem ist das Bezugskoordinatensystem der Szene in dem alle relevanten Objekte (inkl. Kamera, Lichter etc.) positioniert werden
    \end{itemize}
  \end{bsslide}
  
   \begin{bsslide}[\textbufferB]
    \colortext{Rendering Pipeline - Transformation Bild-/Weltkoordinaten}
    \bspar1
    \textbf{Beispiel eines Szenengraphen}
    \bsfigurecaption{szenegraph}{\tiny Bildquelle \cite{Malaka et.al. 2009}}
  \end{bsslide}

   \begin{bsslide}[\textbufferB]
    \colortext{Rendering Pipeline - Transformation Bild-/Weltkoordinaten}
    \bspar1
    \textbf{Beispiel eines Szenengraphen}
    \bsfigurecaption[0.5]{szenegraph-koordinaten}{\tiny Bildquelle \cite{Malaka et.al. 2009}}
  \end{bsslide}

  \begin{bsslide}[\textbufferB]
    \colortext{Rendering Pipeline - Transformation Bild-/Weltkoordinaten}
    \bspar1
    \textbf{Transformation in einem Knoten}
    \bsfigurecaption[0.7]{szenegraph-knoten}{}
    \bspar1
    Erinnerung: Transformationen sind assoziativ, d.h. eine Kette von Transformationen auf einem Objekt kann als eine Matrix ausgedr�ckt werden.
  \end{bsslide}
  
 \begin{bsslide}[\textbufferB]
    \colortext{Rendering Pipeline - Clipping}
    \bspar1
    Das Weltkoordinatensystem ist unendlich gro�. Die Darstellungsfl�che ist jedoch endlich.
    \bspar4
    \begin{minipage}{0.45\linewidth}
      Zwei Schritte sind notwendig
      \begin{itemize}
        \item Beschneiden des Szenegraphen auf den durch ein Fenster definierten sichtbaren Bereich (Clipping)
        \item Transformation in Bildschirmkoordinaten
      \end{itemize}
    \end{minipage}
    \begin{minipage}{0.45\linewidth}
      \begin{center}
        \bsfigure[0.8]{beispiel-car-clipping}
      \end{center}
    \end{minipage}
  \end{bsslide}
  
  \begin{bsslide}[\textbufferB]
    \colortext{Rendering Pipeline - Clipping}
    \bspar1
    \textbf{Beobachtungen}
    \begin{itemize}
      \item Die Szene besteht eigentlich nur aus Punkten oder Linien, d.h. es gen�gt wenn wir uns damit besch�ftigen wie wir Linien beschneiden (wir ignorieren Splines hier).
     \item Das Fenster kann Punkte ganz oder gar nicht beinhalten
      \item Das Fenster kann teilweise Linien vollst�ndig beinhalten, gar nicht oder teilweise
    \end{itemize}
    $\Rightarrow$ Clipping von Punkten\\
    $\Rightarrow$ Linienclipping nach Cohen und Sutherland
  \end{bsslide}
  
  \begin{bsslide}[\textbufferB]
    \colortext{Rendering Pipeline - Clipping}
    \bspar1
    \textbf{Clipping von Punkten}
    \bspar2
    \bsfigurecaption{clipping-punkt}{\tiny Bildquelle \url{http://www.cs.princeton.edu/courses/archive/fall99/cs426/lectures/pipeline/index.htm}}
  \end{bsslide}
  
  \begin{bsslide}[\textbufferB]
    \colortext{Rendering Pipeline - Clipping}
    \bspar1
    \textbf{Linien-Clipping nach Cohen und Shuterland}
    \bspar1
    Ausgangspunkt und Zielsetzung:
    \bsfigurecaption[0.6]{clipping-before-after}{\tiny Bildquelle \url{http://www.cs.princeton.edu/courses/archive/fall99/cs426/lectures/pipeline/index.htm}}
  \end{bsslide}
  
   \begin{bsslide}[\textbufferB]
    \colortext{Rendering Pipeline - Clipping}
   \bspar1 \small
      9 Segmente beschrieben durch 4-Bit Code
      \begin{itemize}
        \item ersten 2-Bit definieren horizontal-position (rechts=10,mitte=00,links=01)
        \item zweiten 2-Bit definieren vertikal-position (unten=10, mitte=00, oben=01)
      \end{itemize}
      \bspar1
    \begin{minipage}{0.45\linewidth}
      Logische Bitoperationen auf Linie ($P_1$,$P_2$)
      \begin{itemize}
        \item[1.] $P_1\;OR\;P_2=0000\Rightarrow$ Linie ist im Clipping Fenster
        \item[2.] $P_1\;AND\;P_2\neq 0000\Rightarrow$ Linie ist in den Randbereichen, d.h. nicht zu zeichnen
        \item[3.] Falls $P_1\neq 0000$ pr�fe Schnitt mit Rand (e.g. $0010$=rechter Rand)
        \item[4.] Falls $P_2\neq 0000$ pr�fe Schnitt mit Rand (e.g. $0010$=rechter Rand)
        \item[5.] Bei Schnitt w�hle Schnittpunkt $P_1'$ als neuen Linienpunkt.
      \end{itemize}
    \end{minipage}
    \hspace{0.5cm}
    \begin{minipage}{0.45\linewidth}
      \small
      \bsfigurecaption[0.55]{clipping-bit-regions}{\tiny Bildquelle \url{http://www.cs.princeton.edu/courses/archive/fall99/cs426/lectures/pipeline/index.htm}}
    \end{minipage}
   \end{bsslide}
   
   \begin{bsslide}[\textbufferB]
     \colortext{Rendering Pipeline - Clipping}
     \bspar1
     Schritte 1. und 2.: schnelle Klassifikation von Linien ohne Schnittpunkte mit Fenstergrenzen
     \bspar3
     \begin{minipage}{0.45\linewidth}
       \bsfigurecaption[0.5]{clipping-step1-AND}{\tiny Bildquelle \url{http://www.cs.princeton.edu/courses/archive/fall99/cs426/lectures/pipeline/index.htm}}
     \end{minipage}
     \hspace{0.5cm}
     \begin{minipage}{0.45\linewidth}
       \bsfigurecaption[0.5]{clipping-step1-AND-2}{\tiny Bildquelle \url{http://www.cs.princeton.edu/courses/archive/fall99/cs426/lectures/pipeline/index.htm}}
     \end{minipage}
   \end{bsslide}
   
   
   \begin{bsslide}[\textbufferB]
     \colortext{Rendering Pipeline - Clipping}
     \bspar1
     Schritte 3.-5.: ermitteln von Schnittpunkten f�r Linien mit Schnittpunkten
     \bspar3
     \begin{minipage}{0.45\linewidth}
       \bsfigurecaption[0.5]{clipping-step3-5-intersection}{\tiny Bildquelle \url{http://www.cs.princeton.edu/courses/archive/fall99/cs426/lectures/pipeline/index.htm}}
     \end{minipage}
     \hspace{0.5cm}
     \begin{minipage}{0.45\linewidth}
       \bsfigurecaption[0.5]{clipping-step3-5-intersection-2}{\tiny Bildquelle \url{http://www.cs.princeton.edu/courses/archive/fall99/cs426/lectures/pipeline/index.htm}}
     \end{minipage}
   \end{bsslide}
   
   \begin{bsslide}[\textbufferB]
     \colortext{Von Welt- nach Bildkoordinaten}
     \bspar1
     Clipping erfolgte noch im Weltkoordinatensystem, d.h wir ben�tigen ein Transformation in das Bildkoordinatensystem des Rasterbilds
    \bspar1 
    \bsfigurecaption[0.8]{welt-zu-bild-koordinate.png}{\tiny \cite{Malaka et.al. 2009}}
    \bspar2
    \begin{center}
      \small
      $x_{bild}=x0_{bild}+(x_{welt}-x0_{welt})*(x1_{bild}-x0_{bild})/(x1_{welt}-x0_{welt})$
      \bspar1
      $y_{bild}=y0_{bild}+(y_{welt}-y0_{welt})*(y1_{bild}-y0_{bild})/(y1_{welt}-y0_{welt})$
    \end{center}
   \end{bsslide}
   
  \begin{bsslide}[\textbufferB]
    \colortext{Rasterisierung}
    \bspar1
    Nach Bestimmung der darzustellenden Punkte und Linien m�ssen deren darzustellende Pixel ermittelt und gezeichnet werden (Ausnahme: Vektorgrafikger�ten wie Plotter, Laserprojektor, Fr�smaschinen)
    \bspar1
    \textbf{Naiver Ansatz zur Rasterisierung von Linien}
    \bspar1
    \begin{minipage}{0.45\linewidth}
    \begin{itemize}
      \small
        \item Annahme: Punkte bereits in Bildkoordinaten und X-Richtung ist l�nger als Y-Richtung
        \item Ermittle Steigung der Linie $k=(y_1-y_2)/(x_1-x_2)$
        \item Laufe in einer Schleife �ber all $x$ Werte zwischen $x_1$ und $x_2$
        \item Setze Pixel $y=round(k*(x-x_1))+y_1$ auf 1 (d.h. wir nehmen immer das n�chstgelegene Pixel)
      \end{itemize}  
      %TODO: Check algorithm
    \end{minipage}
    \begin{minipage}{0.45\linewidth}
      \bsfigurecaption[0.7]{rasterisierung-linie-naive}{\tiny \cite{Malaka et. al. 2009}}
    \end{minipage}
  \end{bsslide}
  
   \begin{bsslide}[\textbufferB]
    \colortext{Rasterisierung - Bresenham-Algorithmus}
    \bspar1
    \small
   Naiver Ansatz ist zeitintensive, da f�r jeden Punkt eine Multiplikation notwendig ist. Der Bresenham-Algorithmus kommt nur mit vergleichen, addieren und Bit-verschieben aus.\bspar4
   Grundidee:
   
    \begin{minipage}{0.35\linewidth}
\bsfigurecaption[0.4]{rasterrize-bresenham-skizze}{\tiny Bildquelle Wikipedia}
  \end{minipage}
  \begin{minipage}{0.55\linewidth}
    \begin{itemize}
      \small
      \item ermittle eine schnelle Richtung (Richtung in der die Koordinate schneller w�chst) und eine langsame Richtung
      \\{\tiny Bei einer Steigung $k<1$ w�chst $x$-Achse schneller als die $y$-Achse}
      \item Fehlerterm: Abweichung im Bereich $[-0.5:0.5]$ zwischen gezeichneten (gerundeten) Pixel zum wirklichen Linienwert
      \item Erh�hen den Fehlerterm mit jedem Schritt in die schnelle Richtung um die Steigung der Kurve $k=\frac{y_2-y_1}{x_2-x_1}$ 
      \item Liegt der Fehler �ber $0.5$, erh�he die langsame Richtung um 1 und reduziere den Fehler um $1.0$ 
    \end{itemize}
  \end{minipage}
   \end{bsslide}
   
   
%    \begin{bsslide}[\textbufferB]
%     \colortext{Rasterisierung - Bresenham-Algorithmus}
%     \bspar1
%     Algorithmus in Pseudo-Code
%     \bspar1
%     \bsfigurecaption[0.65]{rasterrize-bresenham}{\tiny Bildquelle Wikipedia}
%   \end{bsslide}

  
  \begin{bsslide}[\textbufferB]
    \colortext{Rasterisierung - Antialiasing mit Algorithmus von Wu}
    \bspar1
    Einbringen von Anti-Aliasing Techniken durch den Algorithmus von Wu {\tiny \url{http://en.wikipedia.org/wiki/Xiaolin_Wu%27s_line_algorithm
    }}
    \begin{itemize}
      \item F�r jeden ermittelten y-Wert einer Linie, setze mehrere Pixel
      \item F�rbe die Pixel entsprechend ihres Abstandes von der wahren Linie
    \end{itemize}
    \bspar1
    \bsfigurecaption[0.8]{rasterrize-wu}{\tiny Bildquelle Wikipedia}
  \end{bsslide}
  
  \begin{bsslide}[\textbufferB]
    \colortext{Rasterisierung von gef�llten Polygonen}
    \bspar1
    Neben effizienten Zeichnen von Linien sollte auch Polygone effizient bef�llt werden k�nnen.
    \bspar1
    \textbf{Painters Algorithmus:} Scanline basiertes Verfahren
    \bspar1
    \textbf{Beobachtung}: Die Anzahl der Schnittpunkte eines Polygons definiert, ob ein Punkt innerhalb oder au�erhalb des Polygons liegt
    \begin{itemize}
      \item Gerade Anzahl - Punkt liegt au�erhalb
      \item Ungerade Anzahl - Punkt liegt innerhalb
      \item gilt f�r beliebige Polygone
    \end{itemize}
    \bsfigurecaption[0.85]{polygone-und-paritat}{\tiny Bildquelle \cite{Malaka et. al. 2009}}
  \end{bsslide}
   
    
  \begin{bsslide}[\textbufferB]
    \colortext{Rasterisierung von gef�llten Polygonen}
    \bspar1
    \textbf{Painters Algorithmus} im �berblick
    \bspar1
    \begin{itemize}
      \small
      \item Scanline: Bestimme f�r jede Zeile von Pixeln all Schnittpunkte mit den Kanten des Polygons und sortiere sie aufsteigend nach X-Koordinate
      \item Ermittle f�r jedes Pixel innerhalb der Zeile seine Parit�t. Vor dem ersten Schnittpunkt haben alle Pixel die Parit�t null und bei jedem weiteren wird die Parit�t um eins erh�ht
      \item F�rbe alle Pixel mit ungerader Parit�t mit der F�llfarbe ein
    \end{itemize}
    \bsfigurecaption[0.8]{painters-algorithmus}{\tiny Bildquelle \cite{Malaka et. al. 2009}}
  \end{bsslide}
  
  \begin{bsslide}[\textbufferB]
    \colortext{Animationen}
    \bspar1
    \textbf{Computeranimationen} k�nnen in Vektorgrafiken einfach durch zeitlich Ver�nderung der Punkte und anschlie�endem Rendering definiert werden (i.e. Animtaions-transformationen auf Szenegraph)
    \bspar1
    \textbf{Keyframeanimation}
    \begin{itemize}
      \small
      \item Kontrollpunkte und primitive geometrische werden zu zwei Zeitpunkten (z.B. Sekunde 0 und Sekunde 10) bestimmt (die sogenannten Schl�sselbilder/Keyframes)
      \item \textbf{Keyframing:} Die restlichen Bilder dazwischen werden interpoliert
      \item Interpolation kann linear oder nicht-linear erfolgen (z.B. �ber Splines zwischen Kontrollpunkten)
      \item Zus�tzlich Interpolation der Farbe
      \item Ben�tigt entsprechende Designprogramme (z.B. Adobe Flash)
    \end{itemize}
    Weitere Formen der Interaktion:
    \begin{itemize}
      \small
      \item \textbf{Partikelsysteme}: Animation �ber physikalische Simulation
      \item \textbf{Scripting}: Animation �ber Programmcode
      \begin{itemize}
        \item Interaktionen k�nnen ber�cksichtigt werden
        \item Hyperlink Definition m�glich
        \item Beispielformate: Flash und SVG
      \end{itemize}
    \end{itemize}
  \end{bsslide}


\renewcommand{\textbufferB}{Zusammenfassung}
  \begin{bsslide}[\textbufferB]
    \colortext{Vektorgrafiken}
    \bspar1
    \begin{itemize}
      \item Vektorgrafiken bestehend aus Koordinatensystem und primitiven geometrischen Objekten
      \item Punkte, Geraden, Kreise und Interpolationskurven 
      \begin{itemize}
        \item Bezierkurven
      \end{itemize}
      \item Rendering: Szenengraph/Weltkoordinaten, Clipping, Bildschirmkoordinaten, Rasterisierung
      \begin{itemize}
        \item Bresenham Algorithmus zur schnellen Rasterisierung
        \item Antialisasing durch F�rbung
        \item Bef�llung von Polygonen - Painters Algorithmus
      \end{itemize} 
    \end{itemize}
  \end{bsslide}


\begin{bsslide}
  \begin{thebibliography}{9}

\bibitem{Malaka et.al. 2009} Malaka, Butz, Hussmann (2009) - Medieninformatik: Eine Einf�hrung (Pearson Studium - IT), Kapitel 7.1 und 7.2

\end{thebibliography}

\end{bsslide}
\outline 
%%% DATE.  September, 1st, 2012

\clearpage
\bsauthor{GRANITZER}
\bsyear{2012}

%%% Unit statistics.
%%%
%%% corollary:   0
%%% definition:  0
%%% lemma:       0
%%% page:        0
%%% proof:       0
%%% theorem:     0


%%%%%%%%%%%%%%%%%%%%%%%%%%%%%%%%%%%%%%%%%%%%%%%%%%%%%%%%%%%%%%%%%%%%%%%%
%%% SOURCE. \cf[Kapitel 7.1 und 7.2]{Malak, Butz, Hu�mann - Einf�hrung Medieninformatik}
%%%%%%%%%%%%%%%%%%%%%%%%%%%%%%%%%%%%%%%%%%%%%%%%%%%%%%%%%%%%%%%%%%%%%%%%


\begin{bsslide}[Codierung mittels Scalable Vektor Grafik]
  \colortext{Lernziel}
  \bspar1
  \textbf{Unterthemen}
  \begin{itemize}
    \item �berblick SVG und �hnliche Formate
    \item Statische SVG Bilder
    \item Animationen in SVG
    \item Beispiele
    \item Erstellungsprogramme
  \end{itemize}
  \bspar3
%   \textbf{Fragestellungen}
%   \begin{itemize}
%     \item Verstehen der Elemente des SVG Standards als Grundlage zum Umgang mit Vektorzeichenprogrammen
%   \end{itemize}
\end{bsslide}

\renewcommand{\textbufferB}{Kodierung von Vektorgrafiken}
  
  \begin{bsslide}[\textbufferB]
    \colortext{Grundelemente}
    \bspar1
    Zur \textbf{Kodierung} von Vektorgrafiken ist eine entsprechende Sprache zur Definition von Geometrie und Animation notwendig.
    \bspar1
    \textbf{Beispiel Turtle Grafik}
    \bspar1
    \begin{minipage}{0.45\linewidth}
      \begin{itemize}
        \item Bekannt durch Logo Programmiersprache aus den 1970iger Jahren zum Lernen von Programmierung
        \item Turtle: Position, Orientierung, Stift (Gr��e, Farbe etc.)
        \item Sprache beschreibt Weg: ``move forward 10 units''; ``lift pen''; ``turn left 90�''
        \item �hnlich dem Turtle Robot (physisch) aus der fr�hen Robotik Forschung 1960
      \end{itemize}
    \end{minipage}
    \begin{minipage}{0.45\linewidth}
      \bsfigurecaption[0.7]{turtleact.jpg}{\tiny Bildquelle \url{http://www.alancsmith.co.uk/logo/}}
    \end{minipage}
  \end{bsslide}
  
  \begin{bsslide}[\textbufferB]
    \colortext{Grundelemente}
    \bspar1
    \textbf{Post Script/Encapsulated PostScript (EPS)/Portable Document Format (PDF)}
    \bspar1
    \begin{minipage}{0.45\linewidth}
      \begin{itemize}
        \small
        \item PostScript entwickelt 1984 von Adobe zur ger�teunabh�ngigen Darstellung formatierter Texte
        \item Darstellung von Vektorgrafik + Rastergrafik
        \item Vollst�ndige Programmiersprache
        \item Wird meist von Druckern (e.g. Laserdrucker) implementiert
        \item Angabe von Punkten und Pfaden. Pfad wird dann mit Zeichenger�t (e.g. Stift, Pinsel) gezeichnet (siehe Beispiel)
        \item PDF als Nachfolger: Bessere Komprimierung, daf�r keine vollst�ndige Programmiersprache mehr
      \end{itemize}
    \end{minipage}
    \begin{minipage}{0.45\linewidth}
      \bsfigurecaption[0.7]{vektorgrafik-postscript}{\tiny Bildquelle \cite{Malaka et.al. 2009}}
    \end{minipage}
  \end{bsslide}
  


\renewcommand{\textbufferB}{Scalable Vektor Grafik - Grundlagen}
  
  \begin{bsslide}[\textbufferB]
    \colortext{SVG-�berblick}
    \bspar1
    Sprache f�r 2D-Graphik in XML, welche kombinierbar mit anderen Web-Standards
    \begin{minipage}[t]{0.40\linewidth}
      \textbf{Drei Typen grafischer Objekte}
      \begin{itemize}
        \item Shapes (Pfade aus Kurven und geraden Linien)
        \item Bilder (Raster-Graphik)
        \item Text
      \end{itemize}
      \textbf{Grafische Objekte k�nnen}
      \begin{itemize}
        \item gruppiert
        \item gestyled (CSS)
        \item transformiert
        \item zusammengesetzt werden
      \end{itemize}
    \end{minipage}
    \vspace{0.5cm}
    \begin{minipage}[t]{0.40\linewidth}
      \textbf{Vorteile}
      \begin{itemize}
        \item Zusammengesetzte Transformationen
        \item Clipping paths (Bilder flexibel zuschneiden)
        \item Alpha-Masken (Durchsichtigkeit von Objekten)
        \item Filter-Effekte
        \item Objektvorlagen
      \end{itemize}
      \textbf{SVG Zeichnungen sind potenziell}
      \begin{itemize}
        \item interaktiv und
        \item dynamisch
      \end{itemize}
    \end{minipage}
    \bspar2
    \centering{\small 
    W3C Standard \url{http://www.w3.org/Graphics/SVG/
    }
    }
  \end{bsslide}
  
  \begin{bsslide}[\textbufferB]
    \colortext{Grundelemente SVG}
    \bspar1
    \textbf{Koordinatensystem:}
    \begin{itemize}
      \item Koordinatensystem: $(0,0)$ links oben
      \item User Koordinaten System korrespondiert mit Bildschirmkoordinatensystem per Default (100 Pixel in SVG Datei entsprechen 100 Pixel am  Bildschirm)
      \item �nderungen am User Koordinatensystem m�glich
    \end{itemize}
    \bspar1
    \begin{itemize}
      \item Pfade (�hnlich der Turtle) als grundlegende Zeichenobjekt
      \item Erweiterung um geometrische Objekte (e.g. Kreis, Rechteck etc.)
      \item Definition von Attributen (e.g. Strichbreite) pro Pfad/Objekt
    \end{itemize}
    \bspar2
    \small
    $\Rightarrow$ Scalable Vector Graphics (SVG) 1.1 (Second Edition) \url{http://www.w3.org/TR/SVG/Overview.html}
  \end{bsslide}
  
  
  \begin{bsslide}[\textbufferB]
    \colortext{XML Grundstruktur SVG }
    \bspar4
    \bsfigurecaption[0.8]{svg-grundstruktur}{}
  \end{bsslide}
  
  \begin{bsslide}[\textbufferB]
    \colortext{XML Grundstruktur SVG }
    \bspar2
    \begin{itemize}
      \item \texttt{<svg>}
      \begin{itemize}
        \item[1.] Definitionen wieder verwendbarer Bestandteile
        \begin{itemize}
          \item Pfade
          \item Gradienten
          \item Filter
        \end{itemize}
        \item[2.] Zeichnen unter Verwendung der Definitionen und Grundoperationen
      \end{itemize}
      \item \texttt{</svg>}
    \end{itemize}
  \end{bsslide}
  
  \begin{bsslide}[\textbufferB]
    \colortext{Einfaches SVG Beispiel}
    \bspar1
    \footnotesize
    \lstset{language=SVG}
    \lstinputlisting{../code/duck-svg}
    \begin{minipage}{0.45\linewidth}
      \tiny Quelle: Prof. Butz, LMU 
    \end{minipage}
    \begin{minipage}{0.45\linewidth}
      \bsfigurecaption[0.4]{duck-svg}{}
    \end{minipage}
    \small
    \tiny Zum Ausprobieren: \url{http://www.w3schools.com/svg/tryit.asp?filename=trysvg_myfirst}
  \end{bsslide}
  
  \begin{bsslide}[\textbufferB]
    \colortext{SVG - Canvas Gr��e}
    \bspar1
    Bestimmung der Gr��e der Zeichenfl�che auf 2 Arten m�glich
    \bspar1
    \begin{minipage}[t]{0.45\linewidth}
      Absolute Gr��enangabe, d.h. Grafik wird bei Verkleinerung abgeschnitten
      \bspar1
      {\small
      \texttt{<svg width=''320'' height=''220''>}
      }
      \bspar5
      Angabe eines Sichtfensters, d.h. Gr��e wird bei �nderung des Fensters skaliert
      \bspar1
      \small
      \texttt{<svg viewBox=''0 0 320 200''>}
    \end{minipage}
    \begin{minipage}[t]{0.45\linewidth}
      \bsfigurecaption[0.5]{duck-view}{}
      \bspar4
      \bsfigurecaption[0.5]{duck-viewport}{}
    \end{minipage}
  \end{bsslide}
  
  \begin{bsslide}[\textbufferB]
    \colortext{SVG - Rendering Attribute}
    \bspar1
    Beeinflussung eines grafischen Objektes mit Attributen
    \bspar1
    Angabe der Attribute direkt in XML Tag, �ber \texttt{style} Definition in CSS2-Syntax oder �ber CSS2-Stylesheet
    \bspar1
    \begin{minipage}[t]{0.45\linewidth}
      \small
      \begin{itemize}
        \item F�llfarbe \texttt{fill}
        \item Transparenz \texttt{opacity}
        \item Linienfarbe und -st�rke \texttt{stroke} und \texttt{stroke-width}
        \item Linienenden  \texttt{stroke-linecap}
        \item Schriftfamilie und -gr��e \texttt{font-family} und  \texttt{font-size}
      \end{itemize}
    \end{minipage}
    \begin{minipage}[t]{0.45\linewidth}
      \bsfigurecaption[0.5]{duck-stroke-40}{\texttt{ <rect ...... stroke-width="20"/>}}
    \end{minipage}
  \end{bsslide}
  
  \begin{bsslide}[\textbufferB]
    \colortext{SVG mit Stylsheet}
    \bspar1
    \bsfigurecaption{svg-css}{\tiny Quelle Butz LMU}
  \end{bsslide}
  
  \begin{bsslide}[\textbufferB]
    \colortext{SVG - Pfad Syntax}
    \bspar1
    Pfade definieren eine Folge von Zeichenkommandos f�r einen virtuellen Zeichenstifft
    \begin{itemize}
      \item Syntax ist knapp gehalten, um Speicherplatz zu sparen
      \begin{itemize}
        \small
        \item Kommandos mit Zeichenl�nge 1, relative Koordinaten, keine Token Separatoren wenn m�glich, Berzier-Kurven Formulierung
        \item Zus�tzlich M�glichkeit der verlustfreien Kompression (z.B. Huffmann)
      \end{itemize}
      \item Kommandos
      \small
      \begin{itemize}
        \item \texttt{M X Y} Startpunkt auf Koordinate X,Y
        \item \texttt{L X Y} Linie nach X Y
        \item \texttt{Z} Gerade Linie zur�ck zum Startpunkt
        \item \texttt{H X} Horizontale Linie bis Koordiante X
        \item \texttt{V X} Vertikale Linie bis Koordiante X
        \item \texttt{Z} Gerade Linie zur�ck zum Startpunkt
        \item \texttt{Q cx cy x y} Quadratische Berzier-Kurve nach X,Y mit Kontrollpunkt cx,cy
        \item \texttt{C c1x c1y c2x c2y x y} Kubische Berzier-Kurve nach X,Y mit den beiden Kontrollpunkten (c1x,c1y) und (c2x,c2y)
        \item \texttt{A rx ry x-rot la-flag sweep-flag x y} Elliptische Kurve
        \item Kleinbuchstaben Versionen des Kommandos stehen f�r relative Koordinaten (e.g.\texttt{l X Y})
      \end{itemize}
    \end{itemize}
  \end{bsslide}
  
  \begin{bsslide}[\textbufferB]
    \colortext{SVG Bezier Kurven}
    \bspar1
    \begin{minipage}[t]{0.55\linewidth}
      Kubische Bezier Kurven
      \bspar1
      \bsfigurecaption[0.8]{bezier-cubic02}{\tiny \url{http://www.w3.org/TR/SVG/paths.html}}
    \end{minipage}
    \begin{minipage}[t]{0.35\linewidth}
      Quadratische Bezier Kurven
      \bspar1
      \bsfigurecaption[0.45]{bezier-quadratic}{\tiny \url{http://www.w3.org/TR/SVG/paths.html}}
    \end{minipage}
  \end{bsslide}
  
  \begin{bsslide}[\textbufferB]
    \colortext{SVG Bezier Pfad Beispiele}
    \bspar1
    \footnotesize
    \lstset{language=SVG}
    \lstinputlisting{../code/duck-svg-bezier}
    \begin{minipage}{0.45\linewidth}
      \tiny Quelle: \cite{Prof. Butz (LMU)}
    \end{minipage}
    \begin{minipage}{0.45\linewidth}
      \bsfigurecaption[0.9]{duck-svg-bezier}{}
    \end{minipage}
  \end{bsslide}
  
  
  \begin{bsslide}[\textbufferB]
    \colortext{SVG - F�lllregeln}
    \bspar1
    F�llen ben�tigt die Bestimmung, ob ein Punkt innen liegt oder nicht.
    \bspar1
    \textbf{nonzero:} F�r jeden Punkt sende einen Strahl ins unendliche. F�r jede links-rechts (rechts-links) gezeichnete Linie  erh�he (erniedrige) den Z�hler. Wenn Z�hler ungleich 0, dann liegt der Punkt innen
    \bspar3
    \bsfigurecaption{fillrule-nonzero}{\tiny Bildquelle \url{http://www.w3.org/TR/SVG/painting.html}}
    %TODO check beschreibung
  \end{bsslide}
  
  \begin{bsslide}[\textbufferB]
    \colortext{SVG - F�lllregeln}
    \bspar1
    Bestimme, ob ein Punkt innen liegt oder nicht
    \bspar1
    \textbf{evenodd:} F�r jeden Punkt sende einen Strahl ins unendliche. F�r jede links-rechts (rechts-links) gezeichnete Linie erh�he den Z�hler. Wenn Z�hler ungerade ist, dann liegt der Punkt innen
    \bspar3
    \bsfigurecaption{fillrule-evenodd}{\tiny Bildquelle \url{http://www.w3.org/TR/SVG/painting.html}}
  \end{bsslide}
  
  \begin{bsslide}[\textbufferB]
    \colortext{SVG - Text}
    \bspar1
    \texttt{<text>}
      \begin{itemize}
        \item Platzierung von Text auf der Leinwand
        \item Koordinaten-Attribute x und y: Linke untere Ecke des ersten Buchstabens
        \item Schrift, Gr��e etc. �ber Attribute oder Stylesheet
      \end{itemize}
      \bspar1
      \texttt{<tspan>}
      \begin{itemize}
        \item Untergruppe von Text in einem \texttt{<text>}-Element
        \item Einheitliche Formatierung (wie  \texttt{<span>} in HTML)
        \item Realtive Position zur aktuellen Textposition: Attribute \texttt{dx} und \texttt{dy}
      \end{itemize}
      \bspar1 
      Spezialeffekte
      \begin{itemize}
        \item Drehen einzelner Buchstaben (\texttt{rotate}-Attribut)
        \item Text entlang eines beliebigen Pfades (\texttt{<textpath}-Element)
      \end{itemize}
  \end{bsslide}
  
  \begin{bsslide}[\textbufferB]
    \colortext{SVG Text einfach}
    \bspar1
    \begin{minipage}{0.45\linewidth}
      \footnotesize
      \lstset{language=SVG}
      \lstinputlisting{../code/text-svg-normal.svg}
    \end{minipage}
    \begin{minipage}{0.45\linewidth}
      \vspace{4cm}
      \bsfigurecaption[0.8]{text-svg-normal}{}
    \end{minipage}
  \end{bsslide}
  
  \begin{bsslide}[\textbufferB]
    \colortext{SVG Text rotiert}
    \bspar1
    \begin{minipage}[h]{0.45\linewidth}
      \footnotesize
      \lstset{language=SVG}
      \lstinputlisting{../code/text-svg-rotiert.svg}
    \end{minipage}
    \begin{minipage}[h]{0.45\linewidth}
      \vspace{12cm}
      \bsfigurecaption[0.8]{text-svg-rotiert}{}
    \end{minipage}
  \end{bsslide}
  
  \begin{bsslide}[\textbufferB]
    \colortext{SVG - Geometrische Primitive und Transformationen}
    \bspar1
    Pfade definieren i.A. keine geschlossenen Objekte
    \bspar1
    SVG definiert auch Standard Objekte als geometrische Primitive
    \bspar1
    \bsfigurecaption{svg-geometrie-auflistung}{\tiny Quelle Prof. Butz, LMU}
    %TODO: Change figure to table. no time now
  \end{bsslide}
  
  
  \begin{bsslide}[\textbufferB]
    \colortext{SVG Geometrie Beispiel}
    \bspar1
    \begin{minipage}{0.45\linewidth}
      \footnotesize
      \lstset{language=SVG}
      \lstinputlisting{../code/geometrie.svg}
    \end{minipage}
    \begin{minipage}{0.45\linewidth}
      \vspace{8cm}
      \bsfigurecaption[0.6]{svg-geometrie}{}
    \end{minipage}
  \end{bsslide}
  
  \begin{bsslide}[\textbufferB]
    \colortext{SVG -Gruppen und Transformationen}
    \bspar1
    \textbf{Gruppen} - Geometrische Primitive k�nnen zu Gruppen kombiniert werden \texttt{<g>}
    \begin{itemize}
      \item Einheitliche Attributdefinition f�r element der Gruppe
      \item Manipulation der gesamten Gruppe (e.g. Transformation)
    \end{itemize}
    \bspar1
    \textbf{Transformationen}
    \bspar1
    \begin{itemize}
      \small
      \item Verschieben (translate), drehen (rotate), verzerren (skew) oder skalieren (scale)
      \item SVG Attribut \texttt{transform}
      \item Wert des Attributtes entspricht der Operation plus parameter
    \end{itemize}
  \end{bsslide}
  
  \begin{bsslide}[\textbufferB]
    \colortext{SVG Geometrie Beispiel}
    \bspar1
    \begin{minipage}{0.45\linewidth}
      \footnotesize
      \lstset{language=SVG}
      \lstinputlisting{../code/group-transform.svg}
    \end{minipage}
    \begin{minipage}{0.45\linewidth}
      \vspace{10cm}
      \bsfigurecaption[0.95]{group-transform}{}
    \end{minipage}
  \end{bsslide}
  
  \begin{bsslide}[\textbufferB]
    \colortext{SVG - Clipping}
    \bspar1
    \textbf{Clipping} - Ausschneiden von Pfaden aus Objekten
    \bspar1
    \begin{minipage}{0.45\linewidth}
      \footnotesize
      \lstset{language=SVG}
      \lstinputlisting{../code/clipping.svg}
    \end{minipage}
    \begin{minipage}{0.45\linewidth}
      \vspace{5cm}
      \bsfigurecaption[0.5]{clipping}{}
    \end{minipage}
  \end{bsslide}
  
  
  \begin{bsslide}[\textbufferB]
    \colortext{SVG- Alpha Composition, Masking}
    \bspar1
    \textbf{Opacity} - Transparenz von Objekten
    \bspar1
    \textbf{Masking} - Nutzung von beliebigen Objekten zur Maskierung (�berdeckung) eines anderen Objektes
    \bspar1
    \begin{minipage}{0.45\linewidth}
      \footnotesize
      \lstset{language=SVG}
      \lstinputlisting{../code/opacity.svg}
    \end{minipage}
    \begin{minipage}{0.45\linewidth}
      \bsfigurecaption[0.3]{opacity}{}
    \end{minipage}
  \end{bsslide}
  
  \begin{bsslide}[\textbufferB]
    \colortext{SVG - Links}
    \bspar1
    \textbf{Hyperlinks} k�nnen �ber den Xlink (XML Links) Standard \url{http://www.w3.org/1999/xlink} eingef�gt werden
    \begin{itemize}
      \item Der Namensraum muss im svg Tag spezifiziert werden
      \item Beispiel:
      \bspar1
      \lstset{language=SVG}
      \lstinputlisting{../code/xlink-namespace.tex}
    \end{itemize}
  \end{bsslide}
  
  \begin{bsslide}[\textbufferB]
    \colortext{SVG - Symbole}
    \bspar1
    \textbf{Symbole} k�nnen zur wiederholten Verwendung definiert werden
    \bspar3
    \begin{minipage}{0.45\linewidth}
      \footnotesize
      \lstset{language=SVG}
      \lstinputlisting{../code/symbol.tex}
    \end{minipage}
    \begin{minipage}{0.45\linewidth}
      \bsfigurecaption[0.9]{symbol}{}
    \end{minipage}
  \end{bsslide}
  
  \begin{bsslide}[\textbufferB]
    \colortext{SVG - Annimationen}
    \bspar1
    SVG-Objekte k�nnen zeitabh�ngig ver�ndert werden.
    \bspar1
    \textbf{Elemente}
    \begin{itemize}
      \item \texttt{animate} - Animation eines einzelnen Attributes �ber die Zeit. Beispiel f�r Ausblenden (opacity) �ber die Zeit:
      \bspar1
      \texttt{
      <rect>
      <animate attributeType=''CSS'' attributeName=''opacity''
      from=''1'' to=''0'' dur=''5s'' repeatCount=''indefinite'' />
      </rect>
      }
      \item \texttt{set}  - setzen eines Attributwertes f�r ein spezifische Zeit
      \item \texttt{animateMotion} - bewegt ein referenciertes Element entlang eines Pfades
      \item \texttt{animateColor} -Farbtransformation �ber die Zeit
      \item \texttt{animateTransform} - Transformation (Rotation, Skalierung etc.) �ber die Zeit
      \item Grundattribute: \texttt{from, by, to} zur Spezifikation der Startwerte, Schrittweite und Endwerte
      \item Abh�ngig v. d. Elementart weitere Attribute spezifizierbar
    \end{itemize}
  \end{bsslide}
  
  \begin{bsslide}[\textbufferB]
    \colortext{SVG - Annimationen}
    \bspar1
    \footnotesize
    \lstset{language=SVG}
    \lstinputlisting{../code/animate.svg}
    \bsfigurecaption[0.9]{animation}{}
  \end{bsslide}
  
  
  \begin{bsslide}[\textbufferB]
    \colortext{Software zur Darstellung und Erzeugung von SVG}
    \bspar1
    \begin{itemize}
      \item Direkte Browserunterst�tzung:
      \begin{itemize}
        \item Firefox, Safari, Opera, Chrome
        \item (derzeit) nicht in Internet Explorer
        \item Diverse Plugins f�r Internet Explorer, z.B.: Adobe SVG Viewer (nicht weiterentwickelt), Google Chrome Frame
      \end{itemize}
      \item Fr�her: Spezialsoftware (Standalone Viewer)
      \item Vektorgrafik-Editoren mit SVG-Import und Export z.B. Adobe Illustrator, CorelDraw
      \item SVG-orientierte Grafik-Editoren z.B. Inkscape (Open Source), Sketsa
      \item XML-Editoren, Keine Grafik-Unterst�tzung, nur Text-Syntax
    \end{itemize}
    Auch f�r 3D Grafiken gibt es solche Formate (z.B. VRML)
  \end{bsslide}


\renewcommand{\textbufferB}{Zusammenfassung}
  \begin{bsslide}[\textbufferB]
    \colortext{Scalable Vector Graphics}
    \bspar1
    \begin{itemize}
      \item Standardformat f�r Vektorgrafiken im WWW
      \item Kombinierbar mit anderen Web-Standards 
      \item Elemente: Pfade, geometrische Primitive, Bezier, Kreis, Elipse, 
      \item Text, Animationen, Transformationen und F�llungen
      \item Links, Symboldefinitionen etc.
       
    \end{itemize}
    \bspar4
    \centering
  \end{bsslide}


\begin{bsslide}
  \begin{thebibliography}{9}

\bibitem{Malaka et.al. 2009} Malaka, Butz, Hussmann (2009) - Medieninformatik: Eine Einf�hrung (Pearson Studium - IT), Kapitel 7.1 und 7.2

\end{thebibliography}

\end{bsslide}
\end{document}      

%%% DATE. November, 23rd, 2013

\clearpage
\bsauthor{GRANITZER}
\bsyear{2013}

%%% Unit statistics.
%%%
%%% corollary:   0
%%% definition:  0
%%% lemma:       0
%%% page:        0
%%% proof:       0
%%% theorem:     0
%%% theorem:     0
%%%%%%%%%%%%%%% METADATA %%%%%%%%%%%%%%%%
\bsunitname{JPEG Kompression}
\setcounter{bsunit}{3}
%%%%%%%%%%%%%%%%%%%%%%%%%%%%%%%%%%%%%%%%%%%%%%%%%%%%%%%%%%%%%%%%%%%%%%%%
%%% SOURCE. \cf[Kapitel 2]{Malak, Butz, Hu�mann - Einf�hrung Medieninformatik}
%%%%%%%%%%%%%%%%%%%%%%%%%%%%%%%%%%%%%%%%%%%%%%%%%%%%%%%%%%%%%%%%%%%%%%%%
\renewcommand{\textbufferB}{JPEG Kopmpression im \"Uberblick}
%%%%%%%%%%%%%%%%%%%%%%%%%%%%%%%%%%%%%%%%%%%%%%%%%%%%%%%%%%%%%%%%%%%%%%%%
  \begin{bsslide}[\textbufferB]
    \colortext{Verlustbehaftete Kompression}
    \begin{itemize}
      \item \textbf{Problem:} Theoretische und praktische Grenzen von verlustfreien Kompressionsverfahren
      \item \textbf{L�sung:} Zur Erh�hung der Kompressionsraten muss Wissen �ber die Daten eingebracht werden
      \item JPEG-Kompression ist ein mehrstufiges Verfahren, welches Wissen �ber visuelle Wahrnehmung nutzt
      \begin{itemize}
        \small
        \item Die Wahrnehmung wertet nicht alle Informationen des Bildes gleich gut aus
        \item Beispiel: Helligkeit vs. Farbigkeit
        \item Beispiel: Feinabstufung von Verl�ufen
        \item JPEG eignet sich daher am besten f�r Bilder realistischer Szenen (Photographie)
        \item Weniger gut bei Linienzeichnungen, Icons etc.
        \item Mehrmaliges editieren verschlechtert die Qualit�t
      \end{itemize}
      \item Interlaced Modus: JPEG Progressiv
    \end{itemize}
 	Hier: �berblick �ber das Verfahren. Details werden in Master-Vorlesungen behandelt.
  \end{bsslide}
  %%%%%%%%%%%%%%%%%%%%%%%%%%%%%%%%%%%%%%%%%%%%%%%%%%%%%%%%%%%%%%%%%%%%%%%%
  \begin{bsslide}[\textbufferB]
    \colortext{JPEG Kompression: Vergleich}
    \bspar1
    \bsfigurecaption[0.8]{jpeg-vergleich}{\tiny Bildquelle \url{http://www.mathematik.de/spudema/spudema_beitraege/beitraege/rooch/nkap04.html}}
  \end{bsslide}
  %%%%%%%%%%%%%%%%%%%%%%%%%%%%%%%%%%%%%%%%%%%%%%%%%%%%%%%%%%%%%%%%%%%%%%%%
  \begin{bsslide}[\textbufferB]
    \colortext{JPEG Kompression: Ablauf}
    \bspar1
   \bsfigurecaption{jpeg-overview}{}
  \end{bsslide}
  %%%%%%%%%%%%%%%%%%%%%%%%%%%%%%%%%%%%%%%%%%%%%%%%%%%%%%%%%%%%%%%%%%%%%%%%

   \begin{bsslide}[\textbufferB]
    \colortext{JPEG Kompression: Ablauf}
    \bspar1
    \textbf{Ablauf}
    \begin{itemize}
      \item[1.] Konvertierung von RGB nach $Y'C_BC_R$ (Luminanz + 2 Chroma Kan�le)
      \item[2.] \textbf{Chroma Subsampling:} Reduktion der Chroma Kan�le (Farbe)
      \item[3.] Split in 8x8 Pixel Bl�cke und \textbf{Discrete Cosinus Transformation} auf jeden Kanal
      \item[4.]  \textbf{Quantisierung} der Amplituden der Frequenzen (Hochfrequenz Bereiche ungenauer)
      \item[5.]  \textbf{Entropie-Kodierung}: ``Zick-Zack'' Laufl�ngenkodierung und Huffman Kodierung
    \end{itemize}
  \end{bsslide}
  %%%%%%%%%%%%%%%%%%%%%%%%%%%%%%%%%%%%%%%%%%%%%%%%%%%%%%%%%%%%%%%%%%%%%%%%
   \begin{bsslide}[\textbufferB]
    \colortext{JPEG Kompression: Effekte}
    \bspar1
    \small
    \begin{center}
      \begin{minipage}{0.45\linewidth}
        High Quality,  83k, 2.6:1 Ratio
          \bspar1
        \bsfigurecaption[0.6]{jpg-hq-83k}{\tiny Bildquelle Wikipedia}
        \bspar2
        Medium-High Quality,  9k, 23:1 Ratio
          \bspar1
        \bsfigurecaption[0.6]{jpg-q-25-9k}{\tiny Bildquelle Wikipedia}
      \end{minipage}
      \begin{minipage}{0.45\linewidth}
        Medium Quality,  15kk, 46:1 Ratio
          \bspar1
        \bsfigurecaption[0.6]{jpg-q-50-15k}{\tiny Bildquelle Wikipedia}
          \bspar2
        Low Quality,  4k, 144:1 Ration
          \bspar1
        \bsfigurecaption[0.6]{jpg-q-10-4k}{\tiny Bildquelle Wikipedia}
      \end{minipage}
    \end{center}
  \end{bsslide}


\begin{bsslide}[\textbufferB]
\colortext{Bibliographie}
\bspar4
\begin{itemize}
\item \textbf{Malaka et.al. 2009} Malaka, Butz, Hussmann (2009) - Medieninformatik: Eine Einf�hrung (Pearson Studium - IT), Kapitel 2.3

\item \textbf{Kerr} Chrominance Subsampling in Digital Images, \url{http://dougkerr.net/pumpkin/articles/Subsampling.pdf}

\item \textbf{Poynton 2008} Charles Poynton, Chroma subsampling notation, 2008 \url{http://scanline.ca/ycbcr/Chroma_subsampling_notation.pdf}

\end{itemize}

\end{bsslide}

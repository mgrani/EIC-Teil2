\documentclass{bs-reading}
\usepackage{xparse}
\usepackage{todo}
\usepackage{amsmath}
\usepackage{../../EIC-medien-generic/EIC-medien} %

\usepackage[latin1]{inputenc}
\bsauthor{GRANITZER M., KOSCH H.}
\bsyear{2014}


\usepackage{listings}

\usepackage{color}
\definecolor{gray}{rgb}{0.4,0.4,0.4}
\definecolor{darkblue}{rgb}{0.0,0.0,0.6}
\definecolor{cyan}{rgb}{0.0,0.6,0.6}

\lstset{
  basicstyle=\ttfamily,
  columns=fullflexible,
  showstringspaces=false,
  commentstyle=\color{gray}\upshape
}

\lstdefinelanguage{SVG}
{
  morestring=[b]",
  morestring=[s]{>}{<},
  morecomment=[s]{<?}{?>},
  stringstyle=\color{black},
  identifierstyle=\color{darkblue},
  keywordstyle=\color{red},
  morekeywords={transform,stroke,path,width,height,d}% list your attributes here
}
\lstdefinelanguage{XML}
{
  morestring=[b]",
  morestring=[s]{>}{<},
  morecomment=[s]{<?}{?>},
  stringstyle=\color{black},
  identifierstyle=\color{darkblue},
  keywordstyle=\color{red},
  morekeywords={xmlns,version,type}% list your attributes here
}


\renewcommand{\outline}{
\begin{bsslide}[\bsparthead]
\bspar1
\begin{bspartenumerate}[1]
\coloritem[\emcolor]
EIC Teil 2- Medium Bild
\begin{itemize}
\small
\coloritem
Farbwahrnehmung und Farbmodelle
\coloritem
Digitalisierung und Kodierung von Bildern
\coloritem
JPEG Kompression (\"Uberblick, nicht pr\"ufungsrelevant)
\coloritem
Vektorgrafiken am Beispiel SVG (Scalable Vector Graphics)
\end{itemize}
\end{bspartenumerate}
\end{bsslide}
}

%%%%%%%%%%%%%%%%%%%%%%%%%%%%%%%%%%%%%%%
%% Ressources
%% Malaka Book Chapter 3
%%
\begin{document}

\bscollection{Medientechnik}
\bspartname{Medium Bild}
\setcounter{bspart}{4}
\begin{bsslide}
\bspar9
\begin{center}
{\large\bfseries Einf\"uhrung Internet Computing Teil 2 \bspar1 Medien und Information}
\bspar4
 Kapitel 2: Medium Bild
\bspar4
\small 
Michael Granitzer
\bspar2
Harald Kosch
\bspar2
Universit�t Passau
\end{center}
\end{bsslide}


\outline
%%% DATE.  November, 23st, 2013

\clearpage
\bsauthor{GRANITZER}
\bsyear{2013}

%%% Unit statistics.
%%%
%%% corollary:   0
%%% definition:  0
%%% lemma:       0
%%% page:        0
%%% proof:       0
%%% theorem:     0
%%% theorem:     0
%%%%%%%%%%%%%%% METADATA %%%%%%%%%%%%%%%%
\bsunitname{Farbwahrnehmung und Farbmodelle}
\setcounter{bsunit}{1}
%%%%%%%%%%%%%%%%%%%%%%%%%%%%%%%%%%%%%%%%%%%%%%%%%%%%%%%%%%%%%%%%%%%%%%%%
%%% SOURCE. \cf[Kapitel 3]{Malak, Butz, Hu�mann - Einf�hrung
%%% Medieninformatik}
%%% Vermischung Wahrnehmung mit Farbmodellen nicht
%%% ideal. Z.B. Beispiel Chromatische Adaption passt nicht, da
%%% Farbmodelle ja lediglich Farbwerte umrechnen, aber in dem Sinn
%%% nichts mit Farbwahrnehmung zu tun haben.
%%%%%%%%%%%%%%%%%%%%%%%%%%%%%%%%%%%%%%%%%%%%%%%%%%%%%%%%%%%%%%%%%%%%%%%%
\begin{bsslide}[Farbwahrnehmung und Farbmodelle]
  \colortext{Lernziel}
  \bspar1
  \textbf{Themen}
  \begin{itemize}
    \item Licht
    \begin{itemize}
      \item Welche physikalischen Gr��en repr�sentieren Licht?
    \end{itemize}
    \item Farbwahrnehmung
    \begin{itemize}
      \item Wie nehmen wir Licht wahr?
      \item Wie wirkt sich die Wahrnehmung auf die Farbgestaltung aus?
    \end{itemize}
    \item Farbmodelle
    \begin{itemize}
      \item Was sind Farben?
      \item Wie kann man Farben charakterisieren?
      \item Sind Farben abh�ngig vom Ger�t und von der Person?
      \item Wie spezifizieren wir Farben in digitalen Medien?
    \end{itemize}
  \end{itemize}
\end{bsslide}
%%%%%%%%%%%%%%%%%%%%%%%%%%%%%%%%%%%%%%%%%%%%%%%%%%%%%%%%%%%%%%%%%%%%%%%%
\renewcommand{\textbufferB}{Licht}
\begin{bsslide}
  \bspar1
  \begin{center}
 \vspace{0.4\textheight}
    \large\textbufferB
  \end{center}
\end{bsslide}
%%%%%%%%%%%%%%%%%%%%%%%%%%%%%%%%%%%%%%%%%%%%%%%%%%%%%%%%%%%%%%%%%%%%%%%%
  \begin{bsslide}[\textbufferB]
    \colortext{Charakterisierung von Licht}
    \bspar1
    Licht als eine \textbf{elektromagnetische Welle} (oder als Teilchen)
    \bspar1
    \bsfigurecaption[0.5]{em-welle}{\tiny Quelle Wikipedia}
    \bspar1
    \small
    \textbf{Eigenschaften}
    \begin{itemize}
      \item Lineare Ausbreitungsrichtung
      \item Wellencharakter: Brechung, Beugung, Dispersion, Streuung
      \item Teilchen (Photonen): Absorption, Emission
      \item Polarisation (Ausbreitungsebene): linear, zirkular, elliptisch
      \item Charakterisiert durch Wellenl�nge $\lambda$, Frequenz $f$, Periodendauer $T$
      \item $f=1/T$ [Hz]
      \item $T=\lambda /c$ [s]
      \item $f=c/\lambda$ [Hz]
      \item $c=2,998 \cdot 10^8$ m/s
    \end{itemize}
  \end{bsslide}
%%%%%%%%%%%%%%%%%%%%%%%%%%%%%%%%%%%%%%%%%%%%%%%%%%%%%%%%%%%%%%%%%%%%%%%%
  \begin{bsslide}[\textbufferB]
    \colortext{Spektrum der elektromagnetischen Strahlung}
    \bspar1
    \bsfigurecaption[0.8]{em-spectrum}{\tiny Quelle Wikipedia}
    \bspar1
    \begin{itemize}
      \item Reale Strahlung i. A. �berlappung verschiedener Frequenzen
      \item Wie nehmen wir nun diese elektromagnetischen Strahlung wahr?
    \end{itemize}
  \end{bsslide}
%%%%%%%%%%%%%%%%%%%%%%%%%%%%%%%%%%%%%%%%%%%%%%%%%%%%%%%%%%%%%%%%%%%%%%%%
\renewcommand{\textbufferB}{Farbwahrnehmung}
\begin{bsslide}
  \bspar1
  \begin{center}
 \vspace{0.4\textheight}
    \large\textbufferB
  \end{center}
\end{bsslide}
%%%%%%%%%%%%%%%%%%%%%%%%%%%%%%%%%%%%%%%%%%%%%%%%%%%%%%%%%%%%%%%%%%%%%%%%
  %% Siehe Wikipedia http://de.wikipedia.org/wiki/Farbwahrnehmung
%%%%%%%%%%%%%%%%%%%%%%%%%%%%%%%%%%%%%%%%%%%%%%%%%%%%%%%%%%%%%%%%%%%%%%%%
  \begin{bsslide}[\textbufferB]
    \colortext{Wahrgenommene Frequenzbereiche}
    \bspar1\small
    \begin{itemize}
      \item St�bchen: Hell/Dunkelwahrnehmung (120 Millionen, skotopisches Sehen, Nachtsehen)
      \item Zapfen: Farbwahrnehmung (7 Million, photopisches Sehen/Tagessehen)
      \begin{itemize}
        \item 3 Typen: Rot, Gr�n, Blau
        \item Dunkeladaption: �nderung der Farbwahrnehmung bei Dunkelheit
        \item Chromatische Adaption: Wei�abgleich im Auge bei ge�nderter Farbtemperatur
      \end{itemize}
      \bsfigurecaption{wh-empfindlichkeit-zapfen}{\tiny Bildquelle
        Wikipedia}
    \item 3 Farbentheorie nach Young-Helmholtz
    \end{itemize}
  \end{bsslide}
  %%%%%%%%%%%%%%%%%%%%%%%%%%%%%%%%%%%%%%%%%%%%%%%%%%%%%%%%%%%%%%%%%%%%%%%%
    \begin{bsslide}[\textbufferB]
    \colortext{Wahrgenommene Frequenzbereiche}
    \bspar1\small
    \begin{itemize}
      \item Wahrgenommene Farbe hat keine ein-eindeutige Abbildung zum Frequenzspektrum
      \item Beispiel: Blaue und gr\"une  Rezeptorkurven ($R_{blue}$
        bzw. $R_{green}$) und zwei unterschiedliche Signale. Die
        Aktivierung beider Rezeptoren ist f\"ur beiden Signale $s_{1}$
        und $s_{2}$ gleich gross.
    \end{itemize}
    \bspar2
    \bsfigure{metamere}
  \end{bsslide}
 %%%%%%%%%%%%%%%%%%%%%%%%%%%%%%%%%%%%%%%%%%%%%%%%%%%%%%%%%%%%%%%%%%%%%%%%
  \begin{bsslide}[\textbufferB]
    \colortext{Farbsignalverarbeitung im menschlichen Gehirn}
    \bspar1\small
    Neuronale Verarbeitung der Signale R,G, B.
    \begin{itemize}
      \item Summensignal Helligkeit (Gelb) $Y=R+G$
      \item Differenzsignal Rot/Gr�n Unterscheidung $R-G$
      \item Differenzsignal Blau/Gelb Unterscheidung $Y-B$
      \item Y (Yellow) wird als Luminanzsignal bezeichnet und die
        Paare $(R,G)$ und $(Y,B)$ als Gegenfarben
       \item[$\Rightarrow$] Gegenfarbentheorie nach Hering
    \end{itemize}
    \bspar1
    Konsequenzen
    \begin{itemize}
      \item Gelb-Anteil ist wesentlich f�r Helligkeitswahrnehmung
      \item Blau-Anteil spielt keine Rolle bei Helligkeitswahrnehmung (k�hle Farbe)
      \item Farbkontraste Rot/Gr�n und Blau/Gelb besonders klar erkennbar
    \end{itemize}
    \bspar2
    \bsfigurecaption{wh-differenz-signale}{}
  \end{bsslide}
%%%%%%%%%%%%%%%%%%%%%%%%%%%%%%%%%%%%%%%%%%%%%%%%%%%%%%%%%%%%%%%%%%%%%%%%
  \begin{bsslide}[\textbufferB]
    \colortext{Farbsignalverarbeitung im menschlichen Gehirn}
    \bspar1
    Schematischer �berblick
    \bspar1
    \bsfigure{farbwahrnehmung-schema}
  \end{bsslide}
%%%%%%%%%%%%%%%%%%%%%%%%%%%%%%%%%%%%%%%%%%%%%%%%%%%%%%%%%%%%%%%%%%%%%%%%
  \begin{bsslide}[\textbufferB]
    \colortext{Anzahl wahrnehmbarer Farben}
    \bspar1
    Unterscheidung zwischen
    \begin{itemize}
      \item 128 verschiedenen Farbt\"onen (hues)
      \item 130 verschiedenen Farbs�ttigungen (Farbreinheit)
      \item 16 (im gelben) - 26 (im Blauen) verschiedene Helligkeitswerte
    \end{itemize}

    \bspar2 ca. 380 000 verschiedene Farben\\
    Sichere Unterscheidung gleichzeitig dargestellter Farben in Experimenten ca. bei 15 Farben
  \end{bsslide}
%%%%%%%%%%%%%%%%%%%%%%%%%%%%%%%%%%%%%%%%%%%%%%%%%%%%%%%%%%%%%%%%%%%%%%%%
\renewcommand{\textbufferB}{Farbmodelle}
\begin{bsslide}
  \bspar1
  \begin{center}
 \vspace{0.4\textheight}
    \large\textbufferB
  \end{center}
\end{bsslide}
  %%%%%%%%%%%%%%%%%%%%%%%%%%%%%%%%%%%%%%%%%%%%%%%%%%%%%%%%%%%%%%%%%%%%%%%%
  %% Ressources
  %http://neuronresearch.net/vision/files/metameres.htm
  % TODO:  Incorporate information from
  % http://dpbestflow.org/color/color-space-and-color-profiles
  % TODO: Incorporate information from
  % http://www.poynton.com/PDFs/coloureq.pdf
  % TODO http://compression.ru/download/articles/color_space/ch03.pdf
  %%%%%%%%%%%%%%%%%%%%%%%%%%%%%%%%%%%%%%%%%%%%%%%%%%%%%%%%%%%%%%%%%%%%%%%%
  \begin{bsslide}[\textbufferB]
    \colortext{Begriffsbestimmungen Farbe}
    \bspar1
    \textbf{Farbe} ist eine \textbf{individuelle} Wahrnehmung des Lichts durch das menschliche Auge
    \begin{definition}[Farbe]
      Farbe ist diejenige \textbf{Gesichtsempfindung} eines dem Auge des Menschen \textbf{strukturlos erscheinenden Teiles} des Gesichtsfeldes, durch die sich dieser Teil bei \textbf{ein�ugiger Beobachtung mit unbewegtem Auge} von einem gleichzeitig gesehenen, ebenfalls \textbf{strukturlosen angrenzenden Bezirk allein unterscheiden kann}. (Nach Din 5033)
    \end{definition}
    \bspar1
    \begin{itemize}
      \item Farbe entsteht durch Reizung der RGB Rezeptoren
      \item \textbf{Farbreiz} - Licht emittiert von einer Lichtquelle
        (unabh�ngig vom Betrachter)
        \begin{itemize}\small
        \item Schwarzk\"orperstrahlung; Spektrum einer Lichtquelle
        \end{itemize}
      \item \textbf{Farbvalenz} - Aufnahme des Farbreizes durch die
        Augen
        \begin{itemize}\small
        \item Rezeptorkurven multipliziert mit Farbreiz
        \end{itemize}
      \item \textbf{Farbempfindung} - Aufnahme der Farbvalenz im
        Gehirn
        \begin{itemize}\small
        \item Wahrnehmung; Neuronale Adaption
        \end{itemize}
    \end{itemize}
  \end{bsslide}
%%%%%%%%%%%%%%%%%%%%%%%%%%%%%%%%%%%%%%%%%%%%%%%%%%%%%%%%%%%%%%%%%%%%%%%%
  \begin{bsslide}[\textbufferB]
    \colortext{Subjektivit\"at der Farbwahrnehmung}
    \bspar1
    \begin{minipage}{0.55\linewidth}
      \bsfigurecaption[0.8]{chromatische-adaption}{\tiny Bildquelle Wikipedia}
    \end{minipage}
    \begin{minipage}{0.40\linewidth}
      Welche der Karten hat in beiden Bildern die gleiche Farbe?
    \end{minipage}
  \end{bsslide}
%%%%%%%%%%%%%%%%%%%%%%%%%%%%%%%%%%%%%%%%%%%%%%%%%%%%%%%%%%%%%%%%%%%%%%%%
  \begin{bsslide}[\textbufferB]
    \colortext{Subjektivit\"at der Farbwahrnehmung}
    \bspar1
    \begin{minipage}{0.55\linewidth}
      \bsfigurecaption[0.8]{chromatische-adaption}{\tiny Bildquelle Wikipedia}
    \end{minipage}
    \begin{minipage}{0.40\linewidth}
      \begin{itemize}
        \item Bilder mit unterschiedlicher Farbtemperatur (Annahme unterschiedlicher Lichtquellen)
        \item Die zweite Karte von Links hat in beiden Bildern die gleiche Farbe
        \item Die Wahrnehmung betr�gt uns, wegen dem h�heren Rotanteil
          im unteren Bild. Das Auge f�hrt automatisch eine
          Farbanpassung durch (bezeichnet als \textbf{Chromatische Adaption}).
      \end{itemize}
    \end{minipage}
  \end{bsslide}
  %%%%%%%%%%%%%%%%%%%%%%%%%%%%%%%%%%%%%%%%%%%%%%%%%%%%%%%%%%%%%%%%%%%%%%%%
  \begin{bsslide}[\textbufferB]
    \colortext{Motivation Farbmodelle}
    \bspar1  \small
    % http://colorusage.arc.nasa.gov/lum_and_chrom.php
    Betrachtung des Zusammenhangs zwischen den physikalischen Kenngr��en des Lichtes (Spektrum und St�rke) und der Wahrnehmung.\\
    \bspar1
    Physikalische Eigenschaften
    \begin{itemize}

      \item \textbf{Luminanz} - Helligkeit der Farbe (St�rke des Signals) $cd/m^2$
      \item \textbf{Chromatizit�t} - der aktuelle Farbwert der Farbe  (Frequenzspektrum)

    \end{itemize}
    \bspar1
    �berf�hrung in die menschliche Wahrnehmung:
    \begin{itemize}
      \small
      \item F�r Mensch sind \textbf{Luminanz plus 2 Chrominanzsignale}
        ausreichend (f�r z.B. Bienen nicht)
      \item Farbmischung durch �berlagerung der drei Grundfarben (Dreifarbentheorie/Tristimulustheorie nach Helmholtz und Young).
      \item Andere Parameter m�glich: Farbton (Hue), Helligkeit (Brightness), S�ttigung (Saturation)
    \end{itemize}
    \bspar1
    Mehrdeutigkeit der Wahrnehmung:
    \begin{itemize}
      \small
      \item Wahrenommene Luminanz eines Teilbildes ist abh�ngig von Luminanz UND
        Chromatizit\"t des gesamten Bildes!
      \item Metamere - Farbreize unterschiedlicher Spektralverteilung aber mit gleicher wahrgenommener Farbe
    \end{itemize}
    $\Rightarrow$ \textbf{�bersetzung durch Farbmodelle notwendig}
  \end{bsslide}
 %%%%%%%%%%%%%%%%%%%%%%%%%%%%%%%%%%%%%%%%%%%%%%%%%%%%%%%%%%%%%%%%%%%%%%%%
  \begin{bsslide}[\textbufferB]
    \colortext{Farbmodelle}
    \bspar1
    \begin{definition}[Farbmodell, Farbraum]
      Ein Farbmodell ist \textbf{ein mathematisches Modell}, im
      Allgemeinen basierend auf 3 Parametern, zur Beschreibung des
      durch den Menschen wahrgenommen Farbreizes. Die konkrete
      Instanzierung eines Farbmodells nennt man \textbf{Farbraum}. Die
      technische, numerische Realisierung nennt man \textbf{Farbprofil}.
    \end{definition}
    \bspar1
    \textbf{Eigenschaften}
    \begin{itemize}
      \item Abbildung von verschiedenen Ger\"ateeigenschaften:
        \begin{itemize}
        \item Wie repr\"asentiert sich die Farbe ``Rot'' am LCD
          Bildschirm, am R\"ohrenbildschirm?
        \item Das Auge als Ger\"at: Welchen Farbreiz nehmen wir als ``Rot'' wahr?
        \end{itemize}
      \item  Keine Behandlung wahrnehmungsspezifischer Eigenschaften
        wie chromatischer Adaption.
    \end{itemize}
  \end{bsslide}
 %%%%%%%%%%%%%%%%%%%%%%%%%%%%%%%%%%%%%%%%%%%%%%%%%%%%%%%%%%%%%%%%%%%%%%%%
  \begin{bsslide}[\textbufferB]
    \colortext{Kategorisierung von Farbmodellen}
    \bspar1
    \textbf{3 Klassen von Farbmodellen:}
    \bspar1
    \begin{itemize}
      \item \textbf{Allgemeine Farbmodelle:} CIE-Farbraum, CIE-L*a*b
        (kurz Lab)\\
      Vollst�ndige Abbildung aller wahrnehmbaren Farben; Referenzfarbraum f�r Transformationen
      \item \textbf{Pr�sentationsmedium bezogene Farbmodelle}: RGB, CMY, CMYK, YUV, YIQ\\
      Abbildung der durchs Medium abbildbaren Farben und entspr. Medieneigenschaften
      \item \textbf{Physiologisch orientierte Farbmodelle}: HLS, HSV\\
      Abbildung der Wahrnehmungseigenschaften von Farben
    \end{itemize}
    Standardisierungsgremium: CIE (Commision on Illumination)\\
    \small
    Applet f�r Experimente \url{http://dcssrv1.oit.uci.edu/~wiedeman/cspace/}
  \end{bsslide}

 %%%%%%%%%%%%%%%%%%%%%%%%%%%%%%%%%%%%%%%%%%%%%%%%%%%%%%%%%%%%%%%%%%%%%%%%
  \begin{bsslide}[\textbufferB]
    \colortext{Allgemeine Farbmodelle - CIE-Normvalenzsystem (CIE-xyY-Modell)}
    \bspar1
    \begin{minipage}{0.55\linewidth}
      \bsfigurecaption[0.55]{cie-normtafel.png}{CIE Normtafel, Bildquelle Wikipedia}
    \end{minipage}
    \begin{minipage}{0.4\linewidth}
      \small
      \begin{itemize}
        \item Allgemeines Farbmodell: experimentell entwickelter Zusammenhang zwischen Farbreiz und Farbwahrnemung
        \item Standardisiert durch CIE (Commision on Illumination)
        \item Hufeisenform der wahrnehmbaren Farben
        \item umgrenzt durch Spektralfarblinien und Purpurlinie
        \item \textbf{Wei�punkt} $(1/3,1/3,1/3)$
        \item \textbf{Black-Body-Kurve}: Kurve eines Schwarzen Strahlers
        \item \textbf{Farbton}: W-P gerade gleicher Farbt�ne bzw. W-Q als Komplement�rfarbe
        \item \textbf{S�ttigung}: Abstand Wei�punkt-Farbort (W-P = 100\%)
        \item \textbf{Gamut}: Farbunterraum eines Ger�ts repr�sentiert als Dreieck von X,Y,Z Werten des Ger�tes
      \end{itemize}
    \end{minipage}
  \end{bsslide}
 %%%%%%%%%%%%%%%%%%%%%%%%%%%%%%%%%%%%%%%%%%%%%%%%%%%%%%%%%%%%%%%%%%%%%%%%
  \begin{bsslide}[\textbufferB]
    \colortext{Allgemeine Farbmodelle - L*a*b*-Farbraum (DIN EN ISO 11664-4)}
    \bspar1
    Dreidimensionales Farbmodell der CIE entwickelt 1976
    \begin{itemize}
      \item Luminanz plus zwei Chrominanz-Werte (L= Luminanz, a=Gr�n/Rot, b=Blau/Gelb)
      \item 3D Darstellung: Gleiche Abst�nde entsprechen empfindungsgem�� gleichen Farbabst�nden (nichtlineare Darstellung)
      \item Keine einfache Umrechnung zu RGB Farbr�umen
    \end{itemize}
    \begin{minipage}{0.65\linewidth}
      \bsfigurecaption{cie-lab-color-space.jpg}{\tiny
        \url{http://www.colorcodehex.com/color-model.html}}
    \end{minipage}
    \begin{minipage}{0.30\linewidth}
      \begin{itemize}
     \small
      \item Verwendet z.B. intern in Adobe Produkten (z.B. Photoshop) zur
        Umrechnung von Farbr\"aumen
      \end{itemize}
    \end{minipage}

  \end{bsslide}
 %%%%%%%%%%%%%%%%%%%%%%%%%%%%%%%%%%%%%%%%%%%%%%%%%%%%%%%%%%%%%%%%%%%%%%%%
  \begin{bsslide}[\textbufferB]
    \colortext{Pr�sentationsmedium-spezifische Farbmodelle}
    \bspar1
    Betrachtung der Eigenschaften von Pr�sentationsmedien und der mit dem Medium verbundenen Farbwahrnehmung (e.g. Drucker, Bildschirm)
    \begin{itemize}
      \item Betrachten im Allgemeinen nur einen Ausschnitt allgemeiner Farbmodelle (\textbf{Gamut})
      \item Umrechnung zwischen Pr�sentationsmedium-spezifischen Farbmodellen i.a. nicht verlustfrei. Die Verwendung von allgemeinen Farbmodellen ist notwendig
      \item Ger�te spezifizieren entsprechende \textbf{Farbprofile
          (International Color Consortium - ICC)}
        \begin{itemize}
        \item ICC Profile definieren eine Abbildung zwischen
          Ger\"ate-spezifischen Farben v. Ein/Ausgabeger\"aten und den
          CIE LAB oder CIE xyY Farbraum.
        \end{itemize}
      \item Einbettung in Bilddaten m�glich (z.B. als EXIF Metadaten -
        siehe Digitalisierung)
    \end{itemize}
  \end{bsslide}
 %%%%%%%%%%%%%%%%%%%%%%%%%%%%%%%%%%%%%%%%%%%%%%%%%%%%%%%%%%%%%%%%%%%%%%%%
  \begin{bsslide}[\textbufferB]
    \colortext{Pr�sentationsmedium-spezifische Farbmodelle - RGB}
    \bspar1
    RGB Farbmodell - Additives Farbmodell
    \bspar1
    \begin{minipage}{0.30\linewidth}
      \bsfigurecaption[0.9]{rbg-gamut-xyz}{}
    \end{minipage}
    \begin{minipage}{0.30\linewidth}
      \bsfigurecaption[0.9]{rgb-farbraum-2d}{}
    \end{minipage}
    \begin{minipage}{0.30\linewidth}
      \bsfigurecaption[0.9]{rgb-farbraum-3d}{\tiny Quelle Butz}
    \end{minipage}
    \begin{itemize}
      \small
      \item Meist verwendetes Modell f�r aktive Lichterzeugende Ausgabemedien
      \item Spektrale Intensit�t der Komponenten werden addiert
      \item Gamut als Dreieck definiert die erzeugbaren Farben, welche
        (i) von den Prim\"arquellen abh\"angen und (ii) i.a. nicht
        alle wahrnehmbaren Farben abdecken.
    \end{itemize}
  \end{bsslide}
 %%%%%%%%%%%%%%%%%%%%%%%%%%%%%%%%%%%%%%%%%%%%%%%%%%%%%%%%%%%%%%%%%%%%%%%%
  \begin{bsslide}[\textbufferB]
    \colortext{Pr�sentationsmedium-spezifische Farbmodelle - RGB}
    \bspar1        \small
    Beispiele v. RGB Farbr\"aumen und Farbprofilen:
    \bspar1
    \begin{minipage}{0.40\linewidth}
      \begin{itemize}

      \item \textbf{sRGB:} Standardard Format definiert v. HP und
        Microsoft 1996. \\Kleiner Farbraum, der oft den kleinsten
        gemeinsamen Nenner zwischen Ger\"aten
        darstellt. Standardeinstellung bei den meisten Photo
      \item \textbf{Adobe RGB:} Umfangreicher als sRGB. Bilder
        erscheinen dunkler wenn keine Umrechnung in sRGB erfolgt.
      \item \textbf{Adobe wide RGB:} Umfangreicher als Adobe RGB mit
        puren Spektralfarben als Prim\"arfarben.
      \item \textbf{RGBA:} RGB mit Alpha Kanal. Kein eigentliches Farbmodell
      \item \textbf{UHDTV:} RGB Farbraum f. Ultra High Definition TV.
      \end{itemize}
    \end{minipage}
    \begin{minipage}{0.55\linewidth}
      \bsfigurecaption[0.7]{Adobe-RGB-vs-sRGB-gamut-diagram}{Adobe RGB
        Gamut vs. sRGB Gamut.\footnote{\tiny\url{http://photo.net/learn/digital-photography-workflow/advanced-photoshop-tutorials/using-lab-color-adjustments/}}}
    \end{minipage}
  \end{bsslide}
 %%%%%%%%%%%%%%%%%%%%%%%%%%%%%%%%%%%%%%%%%%%%%%%%%%%%%%%%%%%%%%%%%%%%%%%%
  \begin{bsslide}[\textbufferB]
    \colortext{Pr�sentationsmedium-spezifische Farbmodelle - CMY(K)}
    \bspar1
    CMY(K) Farbmodell - Subraktives Farbmodell
    \bspar1
    \begin{minipage}{0.40\linewidth}
      \bsfigurecaption[0.9]{cmy-farbraum-2d}{}
    \end{minipage}
    \begin{minipage}{0.40\linewidth}
      \bsfigurecaption[0.9]{cmy-farbraum-3d}{\tiny Quelle Butz}
    \end{minipage}
    \begin{itemize}
      \small
      \item Meist verwendetes Modell zur Ausgabe auf reflektierenden Ausgabemedien (z.B. Drucker)
      \item Einfallendes Licht wird gefiltert (e.g. Magenta filtert Gr�n)
      \item F�r Tintendrucker oft 4.Farbe K = blacK (aus rein praktischen Gr�nden)
    \end{itemize}
  \end{bsslide}
 %%%%%%%%%%%%%%%%%%%%%%%%%%%%%%%%%%%%%%%%%%%%%%%%%%%%%%%%%%%%%%%%%%%%%%%%
  \begin{bsslide}[\textbufferB]
    \colortext{Pr�sentationsmedium-spezifische Farbmodelle - CMY(K)/RGB}
    \bspar1
    \small
    RGB zu CMY transformation
    \bspar1
    \begin{equation*}
f_{RGB}=\begin{pmatrix}
f_R\\
f_G\\
f_B\\
\end{pmatrix}
=
\begin{pmatrix}
f_{max}-f_C\\
f_{max}-f_M\\
f_{max}-f_Y\\
\end{pmatrix}
=
\begin{pmatrix}
f_{max}\\
f_{max}\\
f_{max}\\
\end{pmatrix}
-
\begin{pmatrix}
f_C\\
f_M\\
f_Y\\
\end{pmatrix}
= white_{RBG}-f_{CMY}
\end{equation*}
\bspar2
Entsprechend gilt:
\begin{equation*}
f_{CMY}=\begin{pmatrix}
f_C\\
f_M\\
f_Y\\
\end{pmatrix}
=
white_{RGB}-
\begin{pmatrix}
f_R\\
f_G\\
f_B\\
\end{pmatrix}
= white_{RBG}-f_{RGB}
\end{equation*}
\bspar1
Beispiel bei 8 Bit Farbiefe, T�rkiston
\begin{equation*}
f_{CMY}
=
\begin{pmatrix}
255\\
255\\
255\\
\end{pmatrix}
-\begin{pmatrix}
80\\
200\\
130\\
\end{pmatrix}
= \begin{pmatrix}
175\\
55\\
125\\
\end{pmatrix}
\end{equation*}
  \end{bsslide}
 %%%%%%%%%%%%%%%%%%%%%%%%%%%%%%%%%%%%%%%%%%%%%%%%%%%%%%%%%%%%%%%%%%%%%%%%
  \begin{bsslide}[\textbufferB]
    \colortext{Pr�sentationsmedium-spezifische Farbmodelle - YCrCb}
    \bspar1
    YCrCB/YPbPr Farbmodell - Helligkeitssignal und zwei Chrominanz Signale
    \bspar1
    \begin{itemize}
      \item Y bezeichnet die Helligkeit
      \item CrCB/PbPr repr�sentieren Chrominanz Signale
      \item Entsteht aus einfacher linearer Transformation aus dem RGB Modell
      \item YCrCb Digitaltechnik
      \item YPbPr analoge Version von YCrCb
    \end{itemize}
     \bspar1
\centering
      RCA connector (Chinch) for Component Video.
      \bspar1
      \bsfigure{YPbPr-component_video}
  \end{bsslide}
 %%%%%%%%%%%%%%%%%%%%%%%%%%%%%%%%%%%%%%%%%%%%%%%%%%%%%%%%%%%%%%%%%%%%%%%%
  \begin{bsslide}[\textbufferB]
    \colortext{Pr�sentationsmedium-spezifische Farbmodelle - YCrCb}
    \bspar1
    YCrCB Konvertierung von RGB
    \bspar2
    \begin{equation*}
f_{YCbCr}
=
\begin{pmatrix}
f_Y\\
f_{Cb}\\
f_{Cr}\\
\end{pmatrix}
= \begin{pmatrix}
0\\
128\\
128\\
\end{pmatrix}
+
\begin{pmatrix}
0.299 & 0.587 & 0.114 \\
-0.168736 & -0.331264 & 0.5\\
0.5 & -0.418688 & -0.081312\\
\end{pmatrix}
\begin{pmatrix}
f_R\\
f_G\\
f_B\\
\end{pmatrix}
\end{equation*}
\bspar2
\begin{itemize}
  \item Y (Helligkeit) Kanal beinhaltet die meiste Information (nutzbar f�r Kompression oder effiziente Fernseh�bertragung)
  \item Entspricht eher dem nat�rlichen Farbempfinden
  \item Farbkan�le enthalten wenig Information (und daher haben wir ein angepasstes Sehsystem)
  \item Entspricht der Abbildung in der Natur: mehr Fl�chen gleicher Farbe im Vergleich zu kontrastreichen Licht/Schatten
\end{itemize}
  \end{bsslide}
  %%%%%%%%%%%%%%%%%%%%%%%%%%%%%%%%%%%%%%%%%%%%%%%%%%%%%%%%%%%%%%%%%%%%%%%%
  \begin{bsslide}[\textbufferB]
    \colortext{Pr�sentationsmedium-spezifische Farbmodelle - YCrCb}
    \bspar1
    Beispielvergleich YCrCB mit RGB
    \bsfigurecaption[0.7]{rgb-YCrCb}{\tiny Bildquelle Malaka, Butz, Hussmann}
  \end{bsslide}
 %%%%%%%%%%%%%%%%%%%%%%%%%%%%%%%%%%%%%%%%%%%%%%%%%%%%%%%%%%%%%%%%%%%%%%%%
  \begin{bsslide}[\textbufferB]
    \colortext{Physiologische Farbmodelle - HSV + HSL}
    \bspar1
    Physiologische Farbmodelle
    \begin{itemize}
      \item entsprechen menschlicher Wahrnehmung
      \item relativ einfache Selektion von Farben
      \begin{itemize}
        \item W�hle zuerst den Farbton
        \item Passe dann Helligkeit und S�ttigung an
      \end{itemize}
    \end{itemize}
    HSB, HSV, HSI:
    \begin{itemize}
      \item Hue, Saturation, Value bzw. Brightness (absolute Helligkeit)
      \item Hue, Saturation, Lightness (relative Helligkeit)
      \item Hue, Saturation, Intensity (Lichtintensit�t)
      \item Unterschied bezogen auf Anwendungsbereich (Phototechnik)
    \end{itemize}
  \end{bsslide}
  %%%%%%%%%%%%%%%%%%%%%%%%%%%%%%%%%%%%%%%%%%%%%%%%%%%%%%%%%%%%%%%%%%%%%%%%
  \begin{bsslide}[\textbufferB]
    \colortext{Physiologische Farbmodelle - HSV + HSL}
    \bspar1
    HSV Farbmodell im Detail
    \bspar1
    \begin{minipage}{0.45\linewidth}
      \bsfigurecaption[0.7]{hsv-cone}{\tiny HSV Kegel - Quelle Wikipedia}
      \bsfigurecaption[0.7]{hsv-hue-skala}{\tiny HSV Farbton Skala - Quelle Wikipedia}
    \end{minipage}
    \begin{minipage}{0.45\linewidth}
      \small
      \textbf{Polar-Koordinaten}
      \begin{itemize}
        \item Farbton als Farbwinkel H auf dem Farbkreis
        \begin{itemize}
          \item  Wertebereich: 0� Rot, 120� Gr�n, 240� Blau)
          \item Physikalisch Interpretation: dominante Wellenl�nge
        \end{itemize}
        \item S�ttigung in Prozent
        \begin{itemize}
          \item  Wertebereich: 0\% = Neutralgrau, 100\% = reine Farbe)
          \item  Physikalische Interpretation: Hinzumischung von Wei�
        \end{itemize}
        \item Hellwert V als Prozentwert
        \begin{itemize}
          \item  Wertebereich: 0\%=keine Helligkeit, 100\% = volle Helligkeit
          \item  Physikalische Interpretation - Gesamtenergiegehalt
        \end{itemize}
      \end{itemize}
    \end{minipage}
  \end{bsslide}
 %%%%%%%%%%%%%%%%%%%%%%%%%%%%%%%%%%%%%%%%%%%%%%%%%%%%%%%%%%%%%%%%%%%%%%%%
  \begin{bsslide}[\textbufferB]
    \colortext{Physiologische Farbmodelle - HSV + HSL}
    \bspar1
    \textbf{Visualisierung des Farbraums}
    \bspar1
    \begin{minipage}{0.3\linewidth}
      Als Kegel
      \bspar3
      \bsfigurecaption[0.7]{hsv-cone-}{\tiny Bildquelle Wikipedia}
    \end{minipage}
    \begin{minipage}{0.3\linewidth}
      Als Zylinder
      \bspar3
      \bsfigurecaption[0.7]{hsv-cylinder}{\tiny Bildquelle Wikipedia}
    \end{minipage}
    \begin{minipage}{0.3\linewidth}
      Farbw�hler f. Programme
      \bspar1
      \bsfigurecaption[0.6]{hsv-farbwaehler}{\tiny Bildquelle Wikipedia}
    \end{minipage}
  \end{bsslide}
 %%%%%%%%%%%%%%%%%%%%%%%%%%%%%%%%%%%%%%%%%%%%%%%%%%%%%%%%%%%%%%%%%%%%%%%%
  \begin{bsslide}[\textbufferB]
    \colortext{Farben in HTML}
    \bspar1
    Spezifikation von Farben im RGB Modell
    \begin{itemize}
      \item jeweils 8 bit, d.h. zweistellige Hexadezimalzahl: \texttt{\#rrggbb}
      \item Beispiel: ``Kiefer'' \texttt{\#006633}
    \end{itemize}
    Anbindung an HTML Tags durch CSS
    \begin{itemize}
      \item Hintergrundfarben, Farben f�r Pseudovermate (e.g. Verweise)
      \item Beispiel \texttt{<body style=\"background-color\":\#CCFFFF\">}
    \end{itemize}
    \bspar1
    Websichere Farben (mit festgelegter Palette f�r Ger�te mit geringerer Farbtiefe):
    \begin{itemize}
      \item Standardpalette von 216 RGB Farben
      \item RBG-Werte durch 51 teilbar
      \item Eingef�hrt von Netscape
    \end{itemize}

  \end{bsslide}
 %%%%%%%%%%%%%%%%%%%%%%%%%%%%%%%%%%%%%%%%%%%%%%%%%%%%%%%%%%%%%%%%%%%%%%%%
  \begin{bsslide}[\textbufferB]
    \colortext{Farben in HTML}
    \bspar1
    \bsfigurecaption{html-farbtabelle}{Farbtabelle und HTML Namen}
  \end{bsslide}
%%%%%%%%%%%%%%%%%%%%%%%%%%%%%%%%%%%%%%%%%%%%%%%%%%%%%%%%%%%%%%%%%%%%%%%%
%%%%%%%%%%%%%%%%%%%%%%%%%%%%%%%%%%%%%%%%%%%%%%%%%%%%%%%%%%%%%%%%%%%%%%%%
%%%%%%%%%%%%%%%%%%%%%%%%%%%%%%%%%%%%%%%%%%%%%%%%%%%%%%%%%%%%%%%%%%%%%%%%

\begin{bsslide}[Zusammenfassung]
  \colortext{Farbwahrnehmung und Farbmodelle}
  \bspar1
  \textbf{Licht}
  \begin{itemize}
    \item Elektromagnetische Welle mit physikalischen Kenngr��en
    \item Unabh�ngig von Wahrnehmung
  \end{itemize}
  \textbf{Farbwahrnehmung}
  \begin{itemize}
    \item Erfolgt �ber drei Zapfen-Rezeptortypen (RGB bzw. SML) und Mischung dieser
    \item Farben sind subjektiv, individuell und entstehen durch Mischung der 3 Grundfarben
  \end{itemize}
  \textbf{Farbmodell}
  \begin{itemize}
    \item Mathematisches Modell f�r die Definition von Farben
    \item Allgemeine Modelle (Alle Farben), Pr�sentationsmediums spezifische Modelle (Drucker, Monitor etc.), Physiologisch orientierte Farbmodelle (beruhen auf Wahrnehmung)
  \end{itemize}
  \bspar4
  \centering
\end{bsslide}
%%%%%%%%%%%%%%%%%%%%%%%%%%%%%%%%%%%%%%%%%%%%%%%%%%%%%%%%%%%%%%%%%%%%%%%%
\begin{bsslide}[Zusammenfassung]
\colortext[Bibliographie]
\bspar4
\begin{itemize}
\item \textbf{Malaka, Butz, Hussmann (2009)} -
  Medieninformatik: Eine Einf�hrung (Pearson Studium - IT)
\item \textbf{Kerr (2010)}- The CIE XYZ and xyY Color Spaces\\ \url{http://graphics.stanford.edu/courses/cs148-10-summer/docs/2010--kerr--cie_xyz.pdf}
\end{itemize}
\end{bsslide}

%%% Local Variables:
%%% mode: latex
%%% TeX-master: "part-de-medium-bild"
%%% End:

\outline
%%% DATE. November, 23rd, 2013

\clearpage

%%% Unit statistics.
%%%
%%% corollary:   0
%%% definition:  0
%%% lemma:       0
%%% page:        0
%%% proof:       0
%%% theorem:     0
%%% theorem:     0
%%%%%%%%%%%%%%% METADATA %%%%%%%%%%%%%%%%
\bsunitname{Digitalisierung und Kodierung digitaler Bilder}
\setcounter{bsunit}{2}
%%%%%%%%%%%%%%%%%%%%%%%%%%%%%%%%%%%%%%%%%%%%%%%%%%%%%%%%%%%%%%%%%%%%%%%%
%%% SOURCE. \cf[Kapitel 3]{Malak, Butz, Hu�mann - Einf�hrung Medieninformatik}
%%%%%%%%%%%%%%%%%%%%%%%%%%%%%%%%%%%%%%%%%%%%%%%%%%%%%%%%%%%%%%%%%%%%%%%
\begin{bsslide}[Digitalisierung und Kodierung digitaler Bilder]
  \colortext{Lernziel}
  \bspar1
  \textbf{Unterthemen}
  \begin{itemize}
    \item Digitalisierung - Grundlegende Eigenschaften von Bildern nach der Digitalisierung
    \item Kodierung - Bilddateiformate (Digital), Aufbau und Anwendungsbereich
  \end{itemize}
  \bspar3
  \textbf{Fragestellungen}
  \begin{itemize}
    \item Was sind die Eigenschaften digitaler Bilder?
    \item Was ist Rasterung und welche Farbtiefen werden verwendet?
    \item Was versteht man unter Dithering?
    \item Welche (wichtigen) Bildformate gibt es?
  \end{itemize}
\end{bsslide}
%%%%%%%%%%%%%%%%%%%%%%%%%%%%%%%%%%%%%%%%%%%%%%%%%%%%%%%%%%%%%%%%%%%%%%%%
\renewcommand{\textbufferB}{Digitalisierung}
\begin{bsslide}
  \bspar1
  \begin{center}
 \vspace{0.4\textheight}
    \large\textbufferB
  \end{center}
\end{bsslide}
%%%%%%%%%%%%%%%%%%%%%%%%%%%%%%%%%%%%%%%%%%%%%%%%%%%%%%%%%%%%%%%%%%%%%%%%
  \begin{bsslide}[\textbufferB]
    \colortext{Klassifikation von Bilddatenformaten}
    \bspar1
    \textbf{Rastergrafik (Bitmap)}
    \begin{itemize}
      \item Speicherung der Abtastung eines Bilds (Pixel-Bild/Rasterisierung).
      \item Kompression
      \begin{itemize}
        \item Verlustfreie Kompression: BMP, TIFF
        \item Verlustbehaftete Kompression: JPEG
        \item Erweiterte Bitmap-Format mit Zusatzfunktionalit�t
          (e.g. Animationen): GIF, PNG
      \end{itemize}
    \end{itemize}
    \textbf{Vektorgrafik}
    \begin{itemize}
      \item Beschreibung von Einzelobjekten (z.B. Linien, Kreis etc.)
      \item Beispiel SVG (Scalable Vector Graphics)
    \end{itemize}
    \textbf{Meta-Files}
    \begin{itemize}
      \item Kombination von Vektor- und Rastergrafik
      \item Beispiele: WMF (Windows Meta File), Macintosh PICT, PDF, EPS, RTF
    \end{itemize}
  \end{bsslide}
  %%%%%%%%%%%%%%%%%%%%%%%%%%%%%%%%%%%%%%%%%%%%%%%%%%%%%%%%%%%%%%%%%%%%%%%%
  \begin{bsslide}[\textbufferB]
    \colortext{R�umliche Aufl�sung: Bildgr��e und Aufl�sung}
    \bspar1
    \textbf{Pixel (Picture Element):} Kleinste Einheit eines Bildes; Bildpunkt
    \begin{itemize}
      \item tats�chliche Gr��e eines Pixels h�ngt vom Ausgabeger�t ab
      \item Seitenverh�ltnis muss nicht 1 sein
    \end{itemize}
    \bspar1
    \textbf{Bildgr��e}: Gr��enangabe eines Bildes in Pixel (100 x 200 Pixel)
    \bspar1
    \textbf{Aufl�sung}: Anzahl der Pixel die auf einer bestimmten Strecke zur Darstellung zu Verf�gung stehen
    \begin{itemize}
      \item Angabe in ppi (pixel per inch), Standardwert 72 ppi
      \item $Breite[pixel]=Breite[in]*Aufl�sung[ppi]$
    \end{itemize}
    \textbf{Skalierung:} Konversion des Bildes auf andere Aufl�sung (resampling)
    \begin{itemize}
      \item Abw�rtsskalierung durch Bildung von Mittelwerten
      \item Aufw�rtsskalierung durch Interpolation der Bildpunkte (e.g. ``Bi-kubische Interpolation); nur eingeschr�nkt automatisierbar
    \end{itemize}
  \end{bsslide}
  %%%%%%%%%%%%%%%%%%%%%%%%%%%%%%%%%%%%%%%%%%%%%%%%%%%%%%%%%%%%%%%%%%%%%%%%

  \begin{bsslide}[\textbufferB]
    \colortext{Rasterbild - R�umliche Aufl�sung}
    \bspar1
    \bsfigurecaption[0.8]{bild-beispiel}{}
  \end{bsslide}
  %%%%%%%%%%%%%%%%%%%%%%%%%%%%%%%%%%%%%%%%%%%%%%%%%%%%%%%%%%%%%%%%%%%%%%%%
  \begin{bsslide}[\textbufferB]
    \colortext{Rasterbild - R�umliche Aufl�sung}
    \bspar1
    \bsfigurecaption[0.8]{aufloesung}{}
  \end{bsslide}
  %%%%%%%%%%%%%%%%%%%%%%%%%%%%%%%%%%%%%%%%%%%%%%%%%%%%%%%%%%%%%%%%%%%%%%%%
  \begin{bsslide}[\textbufferB]
    \colortext{R�umliche Aufl�sung: Bildgr��e und Aufl�sung}
    \bspar1
    \textbf{Beispiel Anzahl Pixel}
    \bsfigurecaption[0.8]{bsp-anzahl-pixel}{\tiny Bildquelle Butz - Medieninformatik}
  \end{bsslide}
  %%%%%%%%%%%%%%%%%%%%%%%%%%%%%%%%%%%%%%%%%%%%%%%%%%%%%%%%%%%%%%%%%%%%%%%%
  \begin{bsslide}[\textbufferB]
    \colortext{Farbtiefen und Farbkan�le}
    \bspar1
    \textbf{Farbtiefe (color resolution):} Anzahl der Farben, die pro Pixel gespeichert werden k�nnen
    \begin{itemize}
      \item 2 Farben (1-bit) f�r schwarz-wei� Bild
      \item 16 Fraben (4-bit)
      \item 256 Farben (8-bit)
      \item 16,7 Millionen Farben (24 bit)
      \item 24 bit Farbtiefe (1 Byte je Grundfarbe RGB) - ``True Color'' ausreichend f�r menschliche Wahrnehmung
      \item Moderne Kameras liefern oft mehr 16 Bit/Kanal
      \item High-Dynamic-Range-Bilder: 32 Bit/Kanal
    \end{itemize}
  \end{bsslide}
  %%%%%%%%%%%%%%%%%%%%%%%%%%%%%%%%%%%%%%%%%%%%%%%%%%%%%%%%%%%%%%%%%%%%%%%%
  \begin{bsslide}[\textbufferB]
    \colortext{Farbtiefen und Farbkan�le}
    \bspar1
    \textbf{Farbkanal:} Teil der gespeicherten Information der sich auf eine der Prim�rkomponenten des gew�hlten Farbmodells bezieht
    \begin{itemize}
      \item Bei Rohdaten meist Rot, Gr�n und Blau (RGB-Modell)
      \item Bei Druckvorbereitung auch CMY bzw. CMYK (``Vierfarbendruck'')
    \end{itemize}
    \bspar2
    \textbf{Alpha Kanal} als zus�tzlicher Kanal: Spezifikation der Transparenz (Durchl�ssigkeit)
    \begin{itemize}
      \item Transparenzabstufung abh�ngig von Aufl�sung
      \begin{itemize}
        \item 1 Bit: Pixel ist transparent oder nicht
        \item 8 Bit: Grad der Transparenz
      \end{itemize}
      \item Formate: PNG, TIFF, PSD (8-bit und mehr), GIF definiert lediglich eine Farbe als Hintergrund (1-Bit Kanal)
    \end{itemize}
  \end{bsslide}
  %%%%%%%%%%%%%%%%%%%%%%%%%%%%%%%%%%%%%%%%%%%%%%%%%%%%%%%%%%%%%%%%%%%%%%%%
  \begin{bsslide}[\textbufferB]
    \colortext{Farbpaletten und indizierte Farben}
    \bspar1
    \textbf{Farbpalette} die Menge der in einem konkreten Bild tats�chlich enthaltenen Farben (meist Teilmenge aller m�glichen Farben)
    \bspar1
    \textbf{Indizierte Speicherung}
    \begin{itemize}
      \item Farbpalette (Tabelle) enth�lt die im Bild vorkommenden Farben
      \item Pro Pixel wird nur der Index in die Palettentabelle gespeichert
      \item �nderung der Farben bei �nderung der Palette
    \end{itemize}
    \bsfigurecaption{farbpalette.png}{\tiny Bildquelle Malaka - VO Digitale Medien}
  \end{bsslide}
  %%%%%%%%%%%%%%%%%%%%%%%%%%%%%%%%%%%%%%%%%%%%%%%%%%%%%%%%%%%%%%%%%%%%%%%%
  \begin{bsslide}[\textbufferB]
    \colortext{Beispiel Rastergrafik}
    \bspar1
    \bsfigurecaption[0.9]{rasterbild-beispiel}{\tiny Bildquelle Wikipedia}
  \end{bsslide}
  %%%%%%%%%%%%%%%%%%%%%%%%%%%%%%%%%%%%%%%%%%%%%%%%%%%%%%%%%%%%%%%%%%%%%%%%
  \begin{bsslide}[\textbufferB]
    \colortext{Beispiel Alphakanal}
    \bspar1
    \bsfigurecaption[0.5]{alphakanal}{\tiny Bildquelle Wikipedia}
  \end{bsslide}
  %%%%%%%%%%%%%%%%%%%%%%%%%%%%%%%%%%%%%%%%%%%%%%%%%%%%%%%%%%%%%%%%%%%%%%%%
  \begin{bsslide}[\textbufferB]
    \colortext{Dithering}
    \bspar1
    \begin{minipage}{0.55\linewidth}
      \textbf{Dithering (Fehlerdiffusion)} simuliert Farbverl�ufe durch bestimmte Pixel-Anordnung (Trade-off R�umliche Aufl�sung vs. Farbaufl�sung)
      \begin{itemize}
        \item  Bei zu geringer Farbtiefe lassen sich Farbverl�ufe schwer darstellen
        \item  Simulation einer Farbe �hnlich eines R�hrenmonitors
        \item Verschiedene Algorithmen
        \begin{itemize}
          \item Floyd-Steinberg
          \item Jarvis-Algorithmus
          \item Stucki
        \end{itemize}
        \item \textbf{Anti-Aliasing} ist die gegens\"atzliche
          Operation. Hier wird ein zu geringe r\"aumliche Aufl\"osung
          durch Verwendung von Farbeaufl\"osung kompensiert, indem
          benachbarte Pixel eine \"ahnliche Farbe bekommen.
        \bspar4

      \end{itemize}
    \end{minipage}
    \begin{minipage}{0.40\linewidth}
      \bsfigurecaption[0.7]{dithering-wikipedia}{Bildquelle Wikpedia}
    \end{minipage}
  \end{bsslide}
%%%%%%%%%%%%%%%%%%%%%%%%%%%%%%%%%%%%%%%%%%%%%%%%%%%%%%%%%%%%%%%%%%%%%%%%
\renewcommand{\textbufferB}{Kodierung}
\begin{bsslide}
  \bspar1
  \begin{center}
 \vspace{0.4\textheight}
    \large\textbufferB
  \end{center}
\end{bsslide}
%%%%%%%%%%%%%%%%%%%%%%%%%%%%%%%%%%%%%%%%%%%%%%%%%%%%%%%%%%%%%%%%%%%%%%%%
  \begin{bsslide}[\textbufferB]
    \colortext{�berblick}
    \bspar1 F�r die Speicherung von Bilddaten sind unterschiedliche Formate definiert, die unterschiedlichen Zwecken dienen.
    \begin{itemize}
      \item Unterschiedliche Einsatzbereiche/Eigenschaften (e.g. Kompression, Farbtiefe, Aufl�sung, Metadaten)
      \item Wir betrachten: TIFF, EXIF, BMP, PNG, GIF, JPG im Detail
    \end{itemize}
  \end{bsslide}
  %%%%%%%%%%%%%%%%%%%%%%%%%%%%%%%%%%%%%%%%%%%%%%%%%%%%%%%%%%%%%%%%%%%%%%%%
  \begin{bsslide}[\textbufferB]
    \colortext{Rasterformate - TIFF (Tagged Image File Format)}
    \bspar1
    \begin{itemize}
      \item Weit verbreitetes Format im Amateur und Profi Bereich f�r hohe Farbtiefe
      \item Entwickelt von Aldus (jetzt Adobe) 1986 als einheitliches Format f�r Scans; Derzeitige Version 6.0 (von 1992)
      \item Erm�glicht \textbf{flexible Kombination} unterschiedlicher Eigenschaften
      \begin{itemize}
        \item Verlustfreie (LZW) /verlustbehaftete Kompression (JPG)
        \item Einbettung von JPG
        \item Ebenen und Seiten
        \item Einbettung von Metadaten
      \end{itemize}
      \item Unterscheidung unterschiedlicher Auspr�gungen aufgrund der Flexibilit�t (Thousands of Incompatible File Formats)
      \begin{itemize}
        \item Baseline TIFF (max. 4 GB gro�e Files)
        \item TIFF Extensions
        \item BigTiff - 64-bit Variante
      \end{itemize}
      \item \textbf{Mime type:} \texttt{image/tiff}
    \end{itemize}
  \end{bsslide}


  %%%%%%%%%%%%%%%%%%%%%%%%%%%%%%%%%%%%%%%%%%%%%%%%%%%%%%%%%%%%%%%%%%%%%%%%
  \begin{bsslide}[\textbufferB]
    \colortext{Rasterformate - BMP (bitmap; Device Independent Bitmap)}
    \bspar1
    Format zur Ger�te-unabh�ngigen Speicherung von Rasterbildern
    \begin{itemize}
      \item Windows/MS DOS Standardformat
      \item Farbtiefen: 1, 4, 8 und 24 bit
      \item Verwendung einer Farbpaletten bei Farbiefe < 24 Bit
      \item RLE (verlustfreie) Kompression
    \end{itemize}
    \textbf{Format}
    \begin{itemize}
      \item File Header mit fixer Gr��e (Signatur, Gr��e, Offset zu Pixel Array)
      \item DIB (BMP) Header mit Metainformation zum Bild (Breite, H�he, Kompression, Farbpalette)
      \item Farbtabelle (optional)
      \item Datenbereich (Bitdaten)
      \item ICC Color Profile
    \end{itemize}
    \textbf{Mime-Type:} \texttt{image/bmp} oder \texttt{image/x-bmp}
  \end{bsslide}

\begin{bsslide}[\textbufferB]
  \colortext{Rasterformate - GIF (Graphics Interchange Format)}
  \bspar1
  \small
  \textbf{GIF-�berblick}
  \begin{itemize}
    \item Entwickelt 1987 von CompuServe (Version GIF87a; Erweiterung GIF89a )
    \item Limitierte Farbtiefe von 8-bit (256 Farben)
    \item Spezifikation der globalen/lokalen Farbpalette aus 24-Bit RGB Farben
    \item Kompression �ber LZW. Da LZW von UniSys patentiert war, gab es einen Patentstreit. Wegen der Lizenzforderungen wurde das Format PNG 1994 entwickelt
    \item GIF war eines von zwei der ersten Bildformate im WWW
    \item Annimationsf�higkeiten in Version GIF89a
    \item Interlacing zur schnelleren Darstellung im Web
  \end{itemize}
  \textbf{Format}
  \begin{itemize}
    \item Header mit Fixer Gr��e (Version, Logischer Darstellungsbereich)
    \item Globale Farbpalette
    \item $n$ Segmente welche jeweils den logischen Darstellungsbereich f�llen und die lokale Farbpalette spezifizieren
    \item Jedes Segment definiert ein Bild ($\Rightarrow$ Animation)
    \item Definition einer Hintergrundfarbe (1-Bit Alphakanal) in
      globaler Farbtabelle m�glich (keine echte Transparenz/Halo Effekt)
  \end{itemize}
\end{bsslide}
%%%%%%%%%%%%%%%%%%%%%%%%%%%%%%%%%%%%%%%%%%%%%%%%%%%%%%%%%%%%%%%%%%%%%%%%
\begin{bsslide}[\textbufferB]
  \colortext{Rasterformate - GIF }
  \bspar1
  Interlacing in GIF
  \begin{itemize}
    \item Ziel: K�rzere empfundene Ladezeit f�r Betrachter
    \item Ablauf: Das Bild wird schrittweise in Zeilen aufgebaut
    \begin{itemize}
      \item 1. Durchlauf: Jede 8. Zeile beginnend mit Zeile 0
      \item 2. Durchlauf: Jede 8. Zeile beginnend in Zeile 4
      \item 3. Durchlauf: Jede 4. Zeile beginnend in Zeile 2
      \item 4. Durchlauf: jede 2. Zeile beginnend in Zeile 1
    \end{itemize}
  \end{itemize}
  \bspar3

  \bsfigurecaption[0.5]{interlacing}{\tiny Bildquelle \url{http://www.schaik.com/png/adam7.html}}
\end{bsslide}
%%%%%%%%%%%%%%%%%%%%%%%%%%%%%%%%%%%%%%%%%%%%%%%%%%%%%%%%%%%%%%%%%%%%%%%%

\begin{bsslide}[\textbufferB]
  \colortext{Rasterformate - PNG }
  \bspar1
  \textbf{Geschichte}
  \begin{itemize}
    \item Entwickelt aufgrund GIF Lizenzforderungen f�r GIF Format (1994)
    \item Erste Version 1996, Standardisierung M�rz 2004: ISO/IEC 15948:2004
  \end{itemize}
  \bspar1
  \textbf{Eigenschaften}
  \bspar1
  \begin{itemize}
    \item Verlustfreie Kompression (DEFLATE/zlib)
    \item ``Full Color'' oder Paletten-basiert (24-Bit RGB oder 32-bit RGBA, Graubilder mit od. ohne Alphakanal)
    \item Fokus Bilddaten im Internet, keine Unterst�tzung nicht RGB basierter Farbr�ume wie CMYK
    \item Verbessertes Interlacing (7-pass vertikal und horizontales Interlacing. Adam7-algorithm)\\
    {\tiny  Vergleich PNG (Adam7) zu GIF Interlacing: \url{http://www.schaik.com/png/adam7.html}}
    \item Animationen nur mit MNG (Multiple Image Network Graphics)
  \end{itemize}
  \bspar1
  \textbf{Mime Typ:}\texttt{image/png}
\end{bsslide}
%%%%%%%%%%%%%%%%%%%%%%%%%%%%%%%%%%%%%%%%%%%%%%%%%%%%%%%%%%%%%%%%%%%%%%%%
\begin{bsslide}[\textbufferB]
  \colortext{Rasterformate - PNG }
  \bspar1
  \textbf{Format}
  \begin{itemize}
    \item Header mit fixer Gr��e
    \item Danach eine Menge von Chunks (L�nge, Typ des Chunks als 4-byte ASCII, Data und CRC Code)
    \bspar1
    \bsfigurecaption[0.6]{png-chunk}{}
    \begin{itemize}
      \item Critical Chunks:
      \begin{itemize}
        \small
        \item Metadaten (Typ=IHDR)
        \item Paletten (Typ=PLTE)
        \item Bilddaten (Typ=Data)
        \item Bildende (Typ=IEND)
      \end{itemize}
      \item Ancillary Chunks:
      \begin{itemize}
        \small
        \item ICC Profile (iCCP)
        \item Gamma (gAMA)
        \item Text/Metadaten (tEXT)
        \item $\ldots$
      \end{itemize}
    \end{itemize}
  \end{itemize}
\end{bsslide}
%%%%%%%%%%%%%%%%%%%%%%%%%%%%%%%%%%%%%%%%%%%%%%%%%%%%%%%%%%%%%%%%%%%%%%%%
\begin{bsslide}[\textbufferB]
  \colortext{Rasterformate - PNG }
  \bspar1
  \textbf{Alpha Kanal}
  \begin{itemize}
    \item Alpha Werte mit jedem Pixel gespeichert
    \item 4 Bytes pro Pixel: ``RGBA''-Farbmodell erm�glicht elegante Schatten und �berg�nge
    \item Vermeidet Wechselwirkung zwischen Anti-Aliasing und Transparenzfarbe, wie dies in GIFs bei einem 1-Bit Alphakanal auftreten kann:\\
    \begin{itemize}
      \item Anit-Aliasing interpoliert Pixel zur Aufl�sung von ``Unstetigkeiten''
      \item Farbwert entspricht nicht der Hintergrundfarbe
      \item[$\Rightarrow$] Randeffekte (Halo Effekt)
    \end{itemize}
  \end{itemize}
  \bsfigurecaption{png-halo}{\tiny \url{http://www.lunaloca.com/tutorials/antialiasing/}}
\end{bsslide}
%%%%%%%%%%%%%%%%%%%%%%%%%%%%%%%%%%%%%%%%%%%%%%%%%%%%%%%%%%%%%%%%%%%%%%%%

\begin{bsslide}[\textbufferB]
  \colortext{Rasterformate - PNG - Kompression}
  \bspar1
  2 Schritte verlustfreier Kompressionsprozess
  \begin{itemize}
    \item Vorfilter (prediction): \\
    �berf�hrung der Daten in einen besser komprimierbaren Bin�r-Stream
    \item compression: \\
    \begin{itemize}
      \item DEFLATE (zlib library unter Linux/Mac)
      \item Patentfreier, verlustfreier Kompressionsalgorithmus
      \begin{itemize}
        \item Basiert auf LZ77 Algorithmus (Vorg�nger von LZW) kombiniert mit Huffman Kodierung
      \end{itemize}
    \end{itemize}
  \end{itemize}
\end{bsslide}
%%%%%%%%%%%%%%%%%%%%%%%%%%%%%%%%%%%%%%%%%%%%%%%%%%%%%%%%%%%%%%%%%%%%%%%%
\begin{bsslide}[\textbufferB]
  \colortext{Rasterformate - PNG}
  \bspar1\small
  \textbf{Filter}
  \begin{itemize}
    \item Komprimiere nicht die Rohwerte, sondern die Differenz sequentiell aufeinander folgender Pixelwerte, z.B. $<1,2,3,4,5,6,7,9,11>\Rightarrow<1,1,1,1,1,1,1,2,2>$
  \end{itemize}
  \bspar1
  \begin{minipage}{0.45\linewidth}
    \begin{itemize}
      \item Heuristische Selektion aus unterschiedlichen Filter je nach Bildzeile:
      \begin{itemize}
        \item Aktuelles Pixel ist X
        \item \textbf{Sub} -Differenz zu  Pixel A
        \item \textbf{Up} - Differenz zu Pixel B
        \item \textbf{Average} - Differenz zu $\frac{B+A}{2}$
        \item \textbf{Path} - Differenz zu $Px=A+B-C$
      \end{itemize}
    \end{itemize}
  \end{minipage}
  \begin{minipage}{0.45\linewidth}
    \bspar 1

    \bsfigurecaption[0.7]{png-filter}{\tiny Bildquelle Wikipedia}
  \end{minipage}
  \bspar3
  Beispiel: PNG Bild mit 256 Farben (oben) komprimiert auf 251 Byte durch Vorfilterung (unten)
  \bsfigurecaption{png-vorfilter-beispiel}{\tiny Bildquelle Wikipedia}
\end{bsslide}
%%%%%%%%%%%%%%%%%%%%%%%%%%%%%%%%%%%%%%%%%%%%%%%%%%%%%%%%%%%%%%%%%%%%%%%%
\begin{bsslide}[\textbufferB]
  \colortext{Rasterformate - JPEG/JFIF}
  \bspar1\small
  \textbf{Geschichte}
  \begin{itemize}
    \item Entwickelt von der Joint Photographic Experts Group 1992
    \item ISO/IE 10918-1 (CCITT) (1994)
    \item Standard f�r Digitalbilder
  \end{itemize}
  \bspar1
  \textbf{Eigenschaften}
  \bspar1
  \begin{itemize}
    \item JPEG - Beschreibt die Kompressions CODECS (Codierung/Decodierung. Details sp�ter)
    \item Kombination verlustfreier und verlustbehafteter Kompression bis zu 10:1 ohne sehbaren Verlust
    \item JPEG File Interchange Format (JFIF) beschreibt das Fileformat (Weitere Varianten: SPIFF, JNG)
    \item Maximale Aufl�sung 65535x65365
    \item Farbraum YCbCr
    \item Format: Header + Segmente (�quivalent zu Tags in TIFFs)
  \end{itemize}
  \bspar1
  \textbf{Mime Typ:}\texttt{image/jpeg}
\end{bsslide}
%%%%%%%%%%%%%%%%%%%%%%%%%%%%%%%%%%%%%%%%%%%%%%%%%%%%%%%%%%%%%%%%%%%%%%%%
\begin{bsslide}[\textbufferB]
  \colortext{Bildmetadaten - EXIF (Exchangeable Image File Format)}
  \bspar1
  Dateiformat zur Speicherung von Metadaten �ber digitale Bilder
  \begin{itemize}
    \item Standardisiert durch Japan Electronic and Information Technology Industries Association
    \item EXIF Daten werden direkt in JPEG oder TIFF im Header verwendet
    \item Nahezu all Digitalkameras unterst�tzen das Format
  \end{itemize}
  \bspar1

  \begin{minipage}{0.55\linewidth}
    \textbf{Metdaten}
    \begin{itemize}
      \item Datum/Uhrzeit
      \item Orientierung (Hoch/Quer), Brennweite, Belichtungszeit, Blendeinstellung
      \item Belichtungsprogramm, ISO-Wert, GPS Koordinaten, Thumbnail
      \item IPTC-Daten (Kommentar, Name)
    \end{itemize}
  \end{minipage}
  \begin{minipage}{0.40\linewidth}
    \vspace{1cm}
    \bsfigurecaption[0.55]{exif}{\tiny Quelle Wikipedia}
  \end{minipage}
\end{bsslide}
%%%%%%%%%%%%%%%%%%%%%%%%%%%%%%%%%%%%%%%%%%%%%%%%%%%%%%%%%%%%%%%%%%%%%%%%
\begin{bsslide}[\textbufferB]
  \colortext{Anwendungsbereiche}
  \bspar1
  {Web-Grafik} (klein, geringe Farbanzahl) $\rightarrow$ GIF od. PNG
  \bspar2
  {Scanner/Bilderzeugung}  $\rightarrow$ TIFF
  \bspar2
  {Austausch zwischen Ger�ten}  $\rightarrow$ TIFF
  \bspar2
  {Hochaufl�sende Bilder mit vielen Farben} $\rightarrow$
  \begin{itemize}
    \item JPEG - bessere Kompression
    \item PNG - bessere Qualit�t, gute Kompression bei grossen einheitlichen Farbfl�chen
  \end{itemize}
\end{bsslide}
%%%%%%%%%%%%%%%%%%%%%%%%%%%%%%%%%%%%%%%%%%%%%%%%%%%%%%%%%%%%%%%%%%%%%%%%
\begin{bsslide}[\textbufferB]
  \colortext{Anwendungsbereiche - Vergleich Kompression}
  \bspar1
  \bsfigurecaption[0.6]{jpeg-n-png-123-photo}{\tiny \url{http://rightyaleft.com/others/png-or-jpeg-or-gif-which-is-best-way-to-choose-your-image-file/}}
  \bspar1

  \bsfigurecaption[0.4]{jpeg-n-png-1}{\tiny \url{http://rightyaleft.com/others/png-or-jpeg-or-gif-which-is-best-way-to-choose-your-image-file/}}
  \bspar1
\end{bsslide}
%%%%%%%%%%%%%%%%%%%%%%%%%%%%%%%%%%%%%%%%%%%%%%%%%%%%%%%%%%%%%%%%%%%%%%%%%
%    \begin{bsslide}[\textbufferB]
%     \colortext{Anwendungsbereiche - JavaScript und Verwendung im Web}
%     \bspar1
%     \todo{HTML und Java Script tags einf�gen}
%   \bspar1
%   \end{bsslide}
%
%%%%%%%%%%%%%%%%%%%%%%%%%%%%%%%%%%%%%%%%%%%%%%%%%%%%%%%%%%%%%%%%%%%%%%%%
\begin{bsslide}[Zusammenfassung]
  \colortext{Digitalisierung und Kodierung}
  \bspar1
  \textbf{Digitalisierung}
  \begin{itemize}
    \item Rastergrafik, Vektorgrafik, Meta-Files
    \item Pixel, Bildgr��e, Aufl�sung, Skalierung, Farbtiefe, Farbkan�le, Paletten
    \item Anti-Aliasing, Dithering
  \end{itemize}
  \textbf{Kodierung}
  \begin{itemize}
    \item Generelle Formate: TIFF, BMP
    \item  Web-optimierte Formate: GIF, PNG, JPEG
    \item Metadaten: EXIF
    \item Kompression: verlustbehaftet oder verlustfrei
  \end{itemize}
\end{bsslide}

\begin{bsslide}[Zusammenfassung]
\colortext[Bibliographie]
\bspar4
\begin{itemize}
\item \textbf{Malaka, Butz, Hussmann (2009)} -
  Medieninformatik: Eine Einf�hrung (Pearson Studium - IT)
\item \textbf{Butz (2011)}-  Prof. Andreas Butz, Vorlesung Digitalen Medien - LMU M�nchen, 2011
\end{itemize}
\end{bsslide}


\outline
%%% DATE. November, 23rd, 2013

\clearpage
\bsauthor{GRANITZER}
\bsyear{2013}

%%% Unit statistics.
%%%
%%% corollary:   0
%%% definition:  0
%%% lemma:       0
%%% page:        0
%%% proof:       0
%%% theorem:     0
%%% theorem:     0
%%%%%%%%%%%%%%% METADATA %%%%%%%%%%%%%%%%
\bsunitname{JPEG Kompression}
\setcounter{bsunit}{3}
%%%%%%%%%%%%%%%%%%%%%%%%%%%%%%%%%%%%%%%%%%%%%%%%%%%%%%%%%%%%%%%%%%%%%%%%
%%% SOURCE. \cf[Kapitel 2]{Malak, Butz, Hu�mann - Einf�hrung Medieninformatik}
%%%%%%%%%%%%%%%%%%%%%%%%%%%%%%%%%%%%%%%%%%%%%%%%%%%%%%%%%%%%%%%%%%%%%%%%
\renewcommand{\textbufferB}{JPEG Kopmpression im \"Uberblick}
%%%%%%%%%%%%%%%%%%%%%%%%%%%%%%%%%%%%%%%%%%%%%%%%%%%%%%%%%%%%%%%%%%%%%%%%
  \begin{bsslide}[\textbufferB]
    \colortext{Verlustbehaftete Kompression}
    \begin{itemize}
      \item \textbf{Problem:} Theoretische und praktische Grenzen von verlustfreien Kompressionsverfahren
      \item \textbf{L�sung:} Zur Erh�hung der Kompressionsraten muss Wissen �ber die Daten eingebracht werden
      \item JPEG-Kompression ist ein mehrstufiges Verfahren, welches Wissen �ber visuelle Wahrnehmung nutzt
      \begin{itemize}
        \small
        \item Die Wahrnehmung wertet nicht alle Informationen des Bildes gleich gut aus
        \item Beispiel: Helligkeit vs. Farbigkeit
        \item Beispiel: Feinabstufung von Verl�ufen
        \item JPEG eignet sich daher am besten f�r Bilder realistischer Szenen (Photographie)
        \item Weniger gut bei Linienzeichnungen, Icons etc.
        \item Mehrmaliges editieren verschlechtert die Qualit�t
      \end{itemize}
      \item Interlaced Modus: JPEG Progressiv
    \end{itemize}
 	Hier: �berblick �ber das Verfahren. Details werden in Master-Vorlesungen behandelt.
  \end{bsslide}
  %%%%%%%%%%%%%%%%%%%%%%%%%%%%%%%%%%%%%%%%%%%%%%%%%%%%%%%%%%%%%%%%%%%%%%%%
  \begin{bsslide}[\textbufferB]
    \colortext{JPEG Kompression: Vergleich}
    \bspar1
    \bsfigurecaption[0.8]{jpeg-vergleich}{\tiny Bildquelle \url{http://www.mathematik.de/spudema/spudema_beitraege/beitraege/rooch/nkap04.html}}
  \end{bsslide}
  %%%%%%%%%%%%%%%%%%%%%%%%%%%%%%%%%%%%%%%%%%%%%%%%%%%%%%%%%%%%%%%%%%%%%%%%
  \begin{bsslide}[\textbufferB]
    \colortext{JPEG Kompression: Ablauf}
    \bspar1
   \bsfigurecaption{jpeg-overview}{}
  \end{bsslide}
  %%%%%%%%%%%%%%%%%%%%%%%%%%%%%%%%%%%%%%%%%%%%%%%%%%%%%%%%%%%%%%%%%%%%%%%%

   \begin{bsslide}[\textbufferB]
    \colortext{JPEG Kompression: Ablauf}
    \bspar1
    \textbf{Ablauf}
    \begin{itemize}
      \item[1.] Konvertierung von RGB nach $Y'C_BC_R$ (Luminanz + 2 Chroma Kan�le)
      \item[2.] \textbf{Chroma Subsampling:} Reduktion der Chroma Kan�le (Farbe)
      \item[3.] Split in 8x8 Pixel Bl�cke und \textbf{Discrete Cosinus Transformation} auf jeden Kanal
      \item[4.]  \textbf{Quantisierung} der Amplituden der Frequenzen (Hochfrequenz Bereiche ungenauer)
      \item[5.]  \textbf{Entropie-Kodierung}: ``Zick-Zack'' Laufl�ngenkodierung und Huffman Kodierung
    \end{itemize}
  \end{bsslide}
  %%%%%%%%%%%%%%%%%%%%%%%%%%%%%%%%%%%%%%%%%%%%%%%%%%%%%%%%%%%%%%%%%%%%%%%%
   \begin{bsslide}[\textbufferB]
    \colortext{JPEG Kompression: Effekte}
    \bspar1
    \small
    \begin{center}
      \begin{minipage}{0.45\linewidth}
        High Quality,  83k, 2.6:1 Ratio
          \bspar1
        \bsfigurecaption[0.6]{jpg-hq-83k}{\tiny Bildquelle Wikipedia}
        \bspar2
        Medium-High Quality,  9k, 23:1 Ratio
          \bspar1
        \bsfigurecaption[0.6]{jpg-q-25-9k}{\tiny Bildquelle Wikipedia}
      \end{minipage}
      \begin{minipage}{0.45\linewidth}
        Medium Quality,  15kk, 46:1 Ratio
          \bspar1
        \bsfigurecaption[0.6]{jpg-q-50-15k}{\tiny Bildquelle Wikipedia}
          \bspar2
        Low Quality,  4k, 144:1 Ration
          \bspar1
        \bsfigurecaption[0.6]{jpg-q-10-4k}{\tiny Bildquelle Wikipedia}
      \end{minipage}
    \end{center}
  \end{bsslide}


\begin{bsslide}[\textbufferB]
\colortext{Bibliographie}
\bspar4
\begin{itemize}
\item \textbf{Malaka et.al. 2009} Malaka, Butz, Hussmann (2009) - Medieninformatik: Eine Einf�hrung (Pearson Studium - IT), Kapitel 2.3

\item \textbf{Kerr} Chrominance Subsampling in Digital Images, \url{http://dougkerr.net/pumpkin/articles/Subsampling.pdf}

\item \textbf{Poynton 2008} Charles Poynton, Chroma subsampling notation, 2008 \url{http://scanline.ca/ycbcr/Chroma_subsampling_notation.pdf}

\end{itemize}

\end{bsslide}

\outline
%%% DATE.  September, 1st, 2012

\clearpage
\bsauthor{GRANITZER}
\bsyear{2012}

%%% Unit statistics.
%%%
%%% corollary:   0
%%% definition:  0
%%% lemma:       0
%%% page:        0
%%% proof:       0
%%% theorem:     0


%%%%%%%%%%%%%%%%%%%%%%%%%%%%%%%%%%%%%%%%%%%%%%%%%%%%%%%%%%%%%%%%%%%%%%%%
%%% SOURCE. \cf[Kapitel 7.1 und 7.2]{Malak, Butz, Hu�mann - Einf�hrung Medieninformatik}
%%%%%%%%%%%%%%%%%%%%%%%%%%%%%%%%%%%%%%%%%%%%%%%%%%%%%%%%%%%%%%%%%%%%%%%%


\begin{bsslide}[Codierung mittels Scalable Vektor Grafik]
  \colortext{Lernziel}
  \bspar1
  \textbf{Unterthemen}
  \begin{itemize}
    \item �berblick SVG und �hnliche Formate
    \item Statische SVG Bilder
    \item Animationen in SVG
    \item Beispiele
    \item Erstellungsprogramme
  \end{itemize}
  \bspar3
%   \textbf{Fragestellungen}
%   \begin{itemize}
%     \item Verstehen der Elemente des SVG Standards als Grundlage zum Umgang mit Vektorzeichenprogrammen
%   \end{itemize}
\end{bsslide}

\renewcommand{\textbufferB}{Kodierung von Vektorgrafiken}
  
  \begin{bsslide}[\textbufferB]
    \colortext{Grundelemente}
    \bspar1
    Zur \textbf{Kodierung} von Vektorgrafiken ist eine entsprechende Sprache zur Definition von Geometrie und Animation notwendig.
    \bspar1
    \textbf{Beispiel Turtle Grafik}
    \bspar1
    \begin{minipage}{0.45\linewidth}
      \begin{itemize}
        \item Bekannt durch Logo Programmiersprache aus den 1970iger Jahren zum Lernen von Programmierung
        \item Turtle: Position, Orientierung, Stift (Gr��e, Farbe etc.)
        \item Sprache beschreibt Weg: ``move forward 10 units''; ``lift pen''; ``turn left 90�''
        \item �hnlich dem Turtle Robot (physisch) aus der fr�hen Robotik Forschung 1960
      \end{itemize}
    \end{minipage}
    \begin{minipage}{0.45\linewidth}
      \bsfigurecaption[0.7]{turtleact.jpg}{\tiny Bildquelle \url{http://www.alancsmith.co.uk/logo/}}
    \end{minipage}
  \end{bsslide}
  
  \begin{bsslide}[\textbufferB]
    \colortext{Grundelemente}
    \bspar1
    \textbf{Post Script/Encapsulated PostScript (EPS)/Portable Document Format (PDF)}
    \bspar1
    \begin{minipage}{0.45\linewidth}
      \begin{itemize}
        \small
        \item PostScript entwickelt 1984 von Adobe zur ger�teunabh�ngigen Darstellung formatierter Texte
        \item Darstellung von Vektorgrafik + Rastergrafik
        \item Vollst�ndige Programmiersprache
        \item Wird meist von Druckern (e.g. Laserdrucker) implementiert
        \item Angabe von Punkten und Pfaden. Pfad wird dann mit Zeichenger�t (e.g. Stift, Pinsel) gezeichnet (siehe Beispiel)
        \item PDF als Nachfolger: Bessere Komprimierung, daf�r keine vollst�ndige Programmiersprache mehr
      \end{itemize}
    \end{minipage}
    \begin{minipage}{0.45\linewidth}
      \bsfigurecaption[0.7]{vektorgrafik-postscript}{\tiny Bildquelle \cite{Malaka et.al. 2009}}
    \end{minipage}
  \end{bsslide}
  


\renewcommand{\textbufferB}{Scalable Vektor Grafik - Grundlagen}
  
  \begin{bsslide}[\textbufferB]
    \colortext{SVG-�berblick}
    \bspar1
    Sprache f�r 2D-Graphik in XML, welche kombinierbar mit anderen Web-Standards
    \begin{minipage}[t]{0.40\linewidth}
      \textbf{Drei Typen grafischer Objekte}
      \begin{itemize}
        \item Shapes (Pfade aus Kurven und geraden Linien)
        \item Bilder (Raster-Graphik)
        \item Text
      \end{itemize}
      \textbf{Grafische Objekte k�nnen}
      \begin{itemize}
        \item gruppiert
        \item gestyled (CSS)
        \item transformiert
        \item zusammengesetzt werden
      \end{itemize}
    \end{minipage}
    \vspace{0.5cm}
    \begin{minipage}[t]{0.40\linewidth}
      \textbf{Vorteile}
      \begin{itemize}
        \item Zusammengesetzte Transformationen
        \item Clipping paths (Bilder flexibel zuschneiden)
        \item Alpha-Masken (Durchsichtigkeit von Objekten)
        \item Filter-Effekte
        \item Objektvorlagen
      \end{itemize}
      \textbf{SVG Zeichnungen sind potenziell}
      \begin{itemize}
        \item interaktiv und
        \item dynamisch
      \end{itemize}
    \end{minipage}
    \bspar2
    \centering{\small 
    W3C Standard \url{http://www.w3.org/Graphics/SVG/
    }
    }
  \end{bsslide}
  
  \begin{bsslide}[\textbufferB]
    \colortext{Grundelemente SVG}
    \bspar1
    \textbf{Koordinatensystem:}
    \begin{itemize}
      \item Koordinatensystem: $(0,0)$ links oben
      \item User Koordinaten System korrespondiert mit Bildschirmkoordinatensystem per Default (100 Pixel in SVG Datei entsprechen 100 Pixel am  Bildschirm)
      \item �nderungen am User Koordinatensystem m�glich
    \end{itemize}
    \bspar1
    \begin{itemize}
      \item Pfade (�hnlich der Turtle) als grundlegende Zeichenobjekt
      \item Erweiterung um geometrische Objekte (e.g. Kreis, Rechteck etc.)
      \item Definition von Attributen (e.g. Strichbreite) pro Pfad/Objekt
    \end{itemize}
    \bspar2
    \small
    $\Rightarrow$ Scalable Vector Graphics (SVG) 1.1 (Second Edition) \url{http://www.w3.org/TR/SVG/Overview.html}
  \end{bsslide}
  
  
  \begin{bsslide}[\textbufferB]
    \colortext{XML Grundstruktur SVG }
    \bspar4
    \bsfigurecaption[0.8]{svg-grundstruktur}{}
  \end{bsslide}
  
  \begin{bsslide}[\textbufferB]
    \colortext{XML Grundstruktur SVG }
    \bspar2
    \begin{itemize}
      \item \texttt{<svg>}
      \begin{itemize}
        \item[1.] Definitionen wieder verwendbarer Bestandteile
        \begin{itemize}
          \item Pfade
          \item Gradienten
          \item Filter
        \end{itemize}
        \item[2.] Zeichnen unter Verwendung der Definitionen und Grundoperationen
      \end{itemize}
      \item \texttt{</svg>}
    \end{itemize}
  \end{bsslide}
  
  \begin{bsslide}[\textbufferB]
    \colortext{Einfaches SVG Beispiel}
    \bspar1
    \footnotesize
    \lstset{language=SVG}
    \lstinputlisting{../code/duck-svg}
    \begin{minipage}{0.45\linewidth}
      \tiny Quelle: Prof. Butz, LMU 
    \end{minipage}
    \begin{minipage}{0.45\linewidth}
      \bsfigurecaption[0.4]{duck-svg}{}
    \end{minipage}
    \small
    \tiny Zum Ausprobieren: \url{http://www.w3schools.com/svg/tryit.asp?filename=trysvg_myfirst}
  \end{bsslide}
  
  \begin{bsslide}[\textbufferB]
    \colortext{SVG - Canvas Gr��e}
    \bspar1
    Bestimmung der Gr��e der Zeichenfl�che auf 2 Arten m�glich
    \bspar1
    \begin{minipage}[t]{0.45\linewidth}
      Absolute Gr��enangabe, d.h. Grafik wird bei Verkleinerung abgeschnitten
      \bspar1
      {\small
      \texttt{<svg width=''320'' height=''220''>}
      }
      \bspar5
      Angabe eines Sichtfensters, d.h. Gr��e wird bei �nderung des Fensters skaliert
      \bspar1
      \small
      \texttt{<svg viewBox=''0 0 320 200''>}
    \end{minipage}
    \begin{minipage}[t]{0.45\linewidth}
      \bsfigurecaption[0.5]{duck-view}{}
      \bspar4
      \bsfigurecaption[0.5]{duck-viewport}{}
    \end{minipage}
  \end{bsslide}
  
  \begin{bsslide}[\textbufferB]
    \colortext{SVG - Rendering Attribute}
    \bspar1
    Beeinflussung eines grafischen Objektes mit Attributen
    \bspar1
    Angabe der Attribute direkt in XML Tag, �ber \texttt{style} Definition in CSS2-Syntax oder �ber CSS2-Stylesheet
    \bspar1
    \begin{minipage}[t]{0.45\linewidth}
      \small
      \begin{itemize}
        \item F�llfarbe \texttt{fill}
        \item Transparenz \texttt{opacity}
        \item Linienfarbe und -st�rke \texttt{stroke} und \texttt{stroke-width}
        \item Linienenden  \texttt{stroke-linecap}
        \item Schriftfamilie und -gr��e \texttt{font-family} und  \texttt{font-size}
      \end{itemize}
    \end{minipage}
    \begin{minipage}[t]{0.45\linewidth}
      \bsfigurecaption[0.5]{duck-stroke-40}{\texttt{ <rect ...... stroke-width="20"/>}}
    \end{minipage}
  \end{bsslide}
  
  \begin{bsslide}[\textbufferB]
    \colortext{SVG mit Stylsheet}
    \bspar1
    \bsfigurecaption{svg-css}{\tiny Quelle Butz LMU}
  \end{bsslide}
  
  \begin{bsslide}[\textbufferB]
    \colortext{SVG - Pfad Syntax}
    \bspar1
    Pfade definieren eine Folge von Zeichenkommandos f�r einen virtuellen Zeichenstifft
    \begin{itemize}
      \item Syntax ist knapp gehalten, um Speicherplatz zu sparen
      \begin{itemize}
        \small
        \item Kommandos mit Zeichenl�nge 1, relative Koordinaten, keine Token Separatoren wenn m�glich, Berzier-Kurven Formulierung
        \item Zus�tzlich M�glichkeit der verlustfreien Kompression (z.B. Huffmann)
      \end{itemize}
      \item Kommandos
      \small
      \begin{itemize}
        \item \texttt{M X Y} Startpunkt auf Koordinate X,Y
        \item \texttt{L X Y} Linie nach X Y
        \item \texttt{Z} Gerade Linie zur�ck zum Startpunkt
        \item \texttt{H X} Horizontale Linie bis Koordiante X
        \item \texttt{V X} Vertikale Linie bis Koordiante X
        \item \texttt{Z} Gerade Linie zur�ck zum Startpunkt
        \item \texttt{Q cx cy x y} Quadratische Berzier-Kurve nach X,Y mit Kontrollpunkt cx,cy
        \item \texttt{C c1x c1y c2x c2y x y} Kubische Berzier-Kurve nach X,Y mit den beiden Kontrollpunkten (c1x,c1y) und (c2x,c2y)
        \item \texttt{A rx ry x-rot la-flag sweep-flag x y} Elliptische Kurve
        \item Kleinbuchstaben Versionen des Kommandos stehen f�r relative Koordinaten (e.g.\texttt{l X Y})
      \end{itemize}
    \end{itemize}
  \end{bsslide}
  
  \begin{bsslide}[\textbufferB]
    \colortext{SVG Bezier Kurven}
    \bspar1
    \begin{minipage}[t]{0.55\linewidth}
      Kubische Bezier Kurven
      \bspar1
      \bsfigurecaption[0.8]{bezier-cubic02}{\tiny \url{http://www.w3.org/TR/SVG/paths.html}}
    \end{minipage}
    \begin{minipage}[t]{0.35\linewidth}
      Quadratische Bezier Kurven
      \bspar1
      \bsfigurecaption[0.45]{bezier-quadratic}{\tiny \url{http://www.w3.org/TR/SVG/paths.html}}
    \end{minipage}
  \end{bsslide}
  
  \begin{bsslide}[\textbufferB]
    \colortext{SVG Bezier Pfad Beispiele}
    \bspar1
    \footnotesize
    \lstset{language=SVG}
    \lstinputlisting{../code/duck-svg-bezier}
    \begin{minipage}{0.45\linewidth}
      \tiny Quelle: \cite{Prof. Butz (LMU)}
    \end{minipage}
    \begin{minipage}{0.45\linewidth}
      \bsfigurecaption[0.9]{duck-svg-bezier}{}
    \end{minipage}
  \end{bsslide}
  
  
  \begin{bsslide}[\textbufferB]
    \colortext{SVG - F�lllregeln}
    \bspar1
    F�llen ben�tigt die Bestimmung, ob ein Punkt innen liegt oder nicht.
    \bspar1
    \textbf{nonzero:} F�r jeden Punkt sende einen Strahl ins unendliche. F�r jede links-rechts (rechts-links) gezeichnete Linie  erh�he (erniedrige) den Z�hler. Wenn Z�hler ungleich 0, dann liegt der Punkt innen
    \bspar3
    \bsfigurecaption{fillrule-nonzero}{\tiny Bildquelle \url{http://www.w3.org/TR/SVG/painting.html}}
    %TODO check beschreibung
  \end{bsslide}
  
  \begin{bsslide}[\textbufferB]
    \colortext{SVG - F�lllregeln}
    \bspar1
    Bestimme, ob ein Punkt innen liegt oder nicht
    \bspar1
    \textbf{evenodd:} F�r jeden Punkt sende einen Strahl ins unendliche. F�r jede links-rechts (rechts-links) gezeichnete Linie erh�he den Z�hler. Wenn Z�hler ungerade ist, dann liegt der Punkt innen
    \bspar3
    \bsfigurecaption{fillrule-evenodd}{\tiny Bildquelle \url{http://www.w3.org/TR/SVG/painting.html}}
  \end{bsslide}
  
  \begin{bsslide}[\textbufferB]
    \colortext{SVG - Text}
    \bspar1
    \texttt{<text>}
      \begin{itemize}
        \item Platzierung von Text auf der Leinwand
        \item Koordinaten-Attribute x und y: Linke untere Ecke des ersten Buchstabens
        \item Schrift, Gr��e etc. �ber Attribute oder Stylesheet
      \end{itemize}
      \bspar1
      \texttt{<tspan>}
      \begin{itemize}
        \item Untergruppe von Text in einem \texttt{<text>}-Element
        \item Einheitliche Formatierung (wie  \texttt{<span>} in HTML)
        \item Realtive Position zur aktuellen Textposition: Attribute \texttt{dx} und \texttt{dy}
      \end{itemize}
      \bspar1 
      Spezialeffekte
      \begin{itemize}
        \item Drehen einzelner Buchstaben (\texttt{rotate}-Attribut)
        \item Text entlang eines beliebigen Pfades (\texttt{<textpath}-Element)
      \end{itemize}
  \end{bsslide}
  
  \begin{bsslide}[\textbufferB]
    \colortext{SVG Text einfach}
    \bspar1
    \begin{minipage}{0.45\linewidth}
      \footnotesize
      \lstset{language=SVG}
      \lstinputlisting{../code/text-svg-normal.svg}
    \end{minipage}
    \begin{minipage}{0.45\linewidth}
      \vspace{4cm}
      \bsfigurecaption[0.8]{text-svg-normal}{}
    \end{minipage}
  \end{bsslide}
  
  \begin{bsslide}[\textbufferB]
    \colortext{SVG Text rotiert}
    \bspar1
    \begin{minipage}[h]{0.45\linewidth}
      \footnotesize
      \lstset{language=SVG}
      \lstinputlisting{../code/text-svg-rotiert.svg}
    \end{minipage}
    \begin{minipage}[h]{0.45\linewidth}
      \vspace{12cm}
      \bsfigurecaption[0.8]{text-svg-rotiert}{}
    \end{minipage}
  \end{bsslide}
  
  \begin{bsslide}[\textbufferB]
    \colortext{SVG - Geometrische Primitive und Transformationen}
    \bspar1
    Pfade definieren i.A. keine geschlossenen Objekte
    \bspar1
    SVG definiert auch Standard Objekte als geometrische Primitive
    \bspar1
    \bsfigurecaption{svg-geometrie-auflistung}{\tiny Quelle Prof. Butz, LMU}
    %TODO: Change figure to table. no time now
  \end{bsslide}
  
  
  \begin{bsslide}[\textbufferB]
    \colortext{SVG Geometrie Beispiel}
    \bspar1
    \begin{minipage}{0.45\linewidth}
      \footnotesize
      \lstset{language=SVG}
      \lstinputlisting{../code/geometrie.svg}
    \end{minipage}
    \begin{minipage}{0.45\linewidth}
      \vspace{8cm}
      \bsfigurecaption[0.6]{svg-geometrie}{}
    \end{minipage}
  \end{bsslide}
  
  \begin{bsslide}[\textbufferB]
    \colortext{SVG -Gruppen und Transformationen}
    \bspar1
    \textbf{Gruppen} - Geometrische Primitive k�nnen zu Gruppen kombiniert werden \texttt{<g>}
    \begin{itemize}
      \item Einheitliche Attributdefinition f�r element der Gruppe
      \item Manipulation der gesamten Gruppe (e.g. Transformation)
    \end{itemize}
    \bspar1
    \textbf{Transformationen}
    \bspar1
    \begin{itemize}
      \small
      \item Verschieben (translate), drehen (rotate), verzerren (skew) oder skalieren (scale)
      \item SVG Attribut \texttt{transform}
      \item Wert des Attributtes entspricht der Operation plus parameter
    \end{itemize}
  \end{bsslide}
  
  \begin{bsslide}[\textbufferB]
    \colortext{SVG Geometrie Beispiel}
    \bspar1
    \begin{minipage}{0.45\linewidth}
      \footnotesize
      \lstset{language=SVG}
      \lstinputlisting{../code/group-transform.svg}
    \end{minipage}
    \begin{minipage}{0.45\linewidth}
      \vspace{10cm}
      \bsfigurecaption[0.95]{group-transform}{}
    \end{minipage}
  \end{bsslide}
  
  \begin{bsslide}[\textbufferB]
    \colortext{SVG - Clipping}
    \bspar1
    \textbf{Clipping} - Ausschneiden von Pfaden aus Objekten
    \bspar1
    \begin{minipage}{0.45\linewidth}
      \footnotesize
      \lstset{language=SVG}
      \lstinputlisting{../code/clipping.svg}
    \end{minipage}
    \begin{minipage}{0.45\linewidth}
      \vspace{5cm}
      \bsfigurecaption[0.5]{clipping}{}
    \end{minipage}
  \end{bsslide}
  
  
  \begin{bsslide}[\textbufferB]
    \colortext{SVG- Alpha Composition, Masking}
    \bspar1
    \textbf{Opacity} - Transparenz von Objekten
    \bspar1
    \textbf{Masking} - Nutzung von beliebigen Objekten zur Maskierung (�berdeckung) eines anderen Objektes
    \bspar1
    \begin{minipage}{0.45\linewidth}
      \footnotesize
      \lstset{language=SVG}
      \lstinputlisting{../code/opacity.svg}
    \end{minipage}
    \begin{minipage}{0.45\linewidth}
      \bsfigurecaption[0.3]{opacity}{}
    \end{minipage}
  \end{bsslide}
  
  \begin{bsslide}[\textbufferB]
    \colortext{SVG - Links}
    \bspar1
    \textbf{Hyperlinks} k�nnen �ber den Xlink (XML Links) Standard \url{http://www.w3.org/1999/xlink} eingef�gt werden
    \begin{itemize}
      \item Der Namensraum muss im svg Tag spezifiziert werden
      \item Beispiel:
      \bspar1
      \lstset{language=SVG}
      \lstinputlisting{../code/xlink-namespace.tex}
    \end{itemize}
  \end{bsslide}
  
  \begin{bsslide}[\textbufferB]
    \colortext{SVG - Symbole}
    \bspar1
    \textbf{Symbole} k�nnen zur wiederholten Verwendung definiert werden
    \bspar3
    \begin{minipage}{0.45\linewidth}
      \footnotesize
      \lstset{language=SVG}
      \lstinputlisting{../code/symbol.tex}
    \end{minipage}
    \begin{minipage}{0.45\linewidth}
      \bsfigurecaption[0.9]{symbol}{}
    \end{minipage}
  \end{bsslide}
  
  \begin{bsslide}[\textbufferB]
    \colortext{SVG - Annimationen}
    \bspar1
    SVG-Objekte k�nnen zeitabh�ngig ver�ndert werden.
    \bspar1
    \textbf{Elemente}
    \begin{itemize}
      \item \texttt{animate} - Animation eines einzelnen Attributes �ber die Zeit. Beispiel f�r Ausblenden (opacity) �ber die Zeit:
      \bspar1
      \texttt{
      <rect>
      <animate attributeType=''CSS'' attributeName=''opacity''
      from=''1'' to=''0'' dur=''5s'' repeatCount=''indefinite'' />
      </rect>
      }
      \item \texttt{set}  - setzen eines Attributwertes f�r ein spezifische Zeit
      \item \texttt{animateMotion} - bewegt ein referenciertes Element entlang eines Pfades
      \item \texttt{animateColor} -Farbtransformation �ber die Zeit
      \item \texttt{animateTransform} - Transformation (Rotation, Skalierung etc.) �ber die Zeit
      \item Grundattribute: \texttt{from, by, to} zur Spezifikation der Startwerte, Schrittweite und Endwerte
      \item Abh�ngig v. d. Elementart weitere Attribute spezifizierbar
    \end{itemize}
  \end{bsslide}
  
  \begin{bsslide}[\textbufferB]
    \colortext{SVG - Annimationen}
    \bspar1
    \footnotesize
    \lstset{language=SVG}
    \lstinputlisting{../code/animate.svg}
    \bsfigurecaption[0.9]{animation}{}
  \end{bsslide}
  
  
  \begin{bsslide}[\textbufferB]
    \colortext{Software zur Darstellung und Erzeugung von SVG}
    \bspar1
    \begin{itemize}
      \item Direkte Browserunterst�tzung:
      \begin{itemize}
        \item Firefox, Safari, Opera, Chrome
        \item (derzeit) nicht in Internet Explorer
        \item Diverse Plugins f�r Internet Explorer, z.B.: Adobe SVG Viewer (nicht weiterentwickelt), Google Chrome Frame
      \end{itemize}
      \item Fr�her: Spezialsoftware (Standalone Viewer)
      \item Vektorgrafik-Editoren mit SVG-Import und Export z.B. Adobe Illustrator, CorelDraw
      \item SVG-orientierte Grafik-Editoren z.B. Inkscape (Open Source), Sketsa
      \item XML-Editoren, Keine Grafik-Unterst�tzung, nur Text-Syntax
    \end{itemize}
    Auch f�r 3D Grafiken gibt es solche Formate (z.B. VRML)
  \end{bsslide}


\renewcommand{\textbufferB}{Zusammenfassung}
  \begin{bsslide}[\textbufferB]
    \colortext{Scalable Vector Graphics}
    \bspar1
    \begin{itemize}
      \item Standardformat f�r Vektorgrafiken im WWW
      \item Kombinierbar mit anderen Web-Standards 
      \item Elemente: Pfade, geometrische Primitive, Bezier, Kreis, Elipse, 
      \item Text, Animationen, Transformationen und F�llungen
      \item Links, Symboldefinitionen etc.
       
    \end{itemize}
    \bspar4
    \centering
  \end{bsslide}


\begin{bsslide}
  \begin{thebibliography}{9}

\bibitem{Malaka et.al. 2009} Malaka, Butz, Hussmann (2009) - Medieninformatik: Eine Einf�hrung (Pearson Studium - IT), Kapitel 7.1 und 7.2

\end{thebibliography}

\end{bsslide}

\end{document}
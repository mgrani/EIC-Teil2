

\begin{exercise}{Histogrammmanipulationen}
\label{ex-de-mt-histogram}
Gegeben sei folgendes 3-Bit Graustufenbild:
\[
\begin{pmatrix}
	4 & 4 & 4 & 4 & 4\\
	3 & 4 & 5 & 4 & 3\\
	3 & 5 & 5 & 5 & 3\\
	3 & 4 & 5 & 4 & 3\\
	4 & 4 & 4 & 4 & 4\\
\end{pmatrix}
\]
\begin{itemize}
  \item Bilden Sie das Grauwerthistogramm.
  \item F�hren sie eine Histogrammeinebnung am obigen Graustufenbild durch. Zeigen Sie das entstehende kumulative Histogramm sowie das transformierte Bild.
  \answer{
%see digital image processing by jayaraman\\

\begin{tabular}{l|cccccccc}
Graustufen 						 & 0  & 1 & 2 & 3 & 4 & 5 & 6 & 7\\
\# Pixel (Histogram)  			 & 0  & 0 & 0 & 6 & 14 & 5 & 0 & 0\\
Kumulative Summe (kum. Hist.) 	 & 0  & 0 & 0 & 6 & 20 & 25 & 25 & 25\\
Spreizungshistogramm 
($B^{neu}(x,y)=\frac{B(x,y)-3}{2}*7$)	 & 6  & 0 & 0 & 0 & 14 & 0 & 0 & 5 \\

Neuer Grauwert bei Einebnung 
($G^{neu}(i)=\frac{7}{25}*H_{kum}(i)$) & 0  & 0 & 0 & 2 & 6 & 7 & 7 &
7 \\
%Achtung: die Zeile mit Einebnung zeigt die neuen Grauwerte!

\end{tabular}

Gespreizte Matrix:
\[
\begin{pmatrix}
	4 & 4 & 4 & 4 & 4\\
	0 & 4 & 7 & 4 & 0\\
	0 & 7 & 7 & 7 & 0\\
	0 & 4 & 7 & 4 & 0\\
	4 & 4 & 4 & 4 & 4\\
\end{pmatrix}
\]


Einebnung d. Matrix:
\[
\begin{pmatrix}
	6 & 6 & 6 & 6 & 6\\
	2 & 6 & 7 & 6 & 2\\
	2 & 7 & 7 & 7 & 2\\
	2 & 6 & 7 & 6 & 2\\
	6 & 6 & 6 & 6 & 6\\
\end{pmatrix}
\]
 }
  \item F�hren Sie eine Histogrammspreizung am obigen Graustufenbild durch.  Zeigen Sie das entstehende Histogramm sowie das transformierte Bild.
  \item Zeigen sie den Unterschied zwischen einer Histogrammspreizung (Streching) und einer Histogrammeinebnung (Equalization) mit der Bildbearbeitungssoftware GIMP.
  
 \answer{
Take image on slide Spreizung  and use GIMP Functions Color Auto Equalize and Color => Stretch Contrast and compre the Histogramm
 }  
\end{itemize}




\end{exercise}


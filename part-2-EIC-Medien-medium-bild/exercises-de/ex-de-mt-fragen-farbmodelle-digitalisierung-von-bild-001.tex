

\begin{exercise}{Farbmodelle und Digitalisierung von Bildern - Allgemeine Fragen}
\label{ex-de-mt-informationstheorie-fragen}
\begin{enumerate}
\item Was sind Farbmodelle und wie stehen sie in Zusammenhang mit der menschlichen Wahrnehmung?
\item Erkl�ren Sie das CIE-XYZ Farbmodell. Welche Farben sind damit abbildbar und was spezifiziert die Black-Body Kurve?
\item Erkl�ren Sie das RGB und das CYMK Farbmodell und vergleichen sie es mit dem YCrCb/YPbPr Farbmodell.
\item Wie kann das HSV bzw. HSL Farbmodell als Zylinder dargestellt werden und worin liegt der Unterschied zum RGB Modell? 
\item Beschreiben sie den Digitalisierungsprozess bei Bildern sowie die damit verbundenen wichtigsten Kenngr��en?
\item Was bedeutet Dithering?
\item Definieren sie den Begriff Gamma-Korrektur sowie dessen Anwendungsbereich.
\item Beschreiben sie den grundlegenden Aufbau von TIFF, BMP, GIF und PNG.
\item Worin unterscheiden sich PNG und GIFs?
\item Diskutieren sie die verschiedenen Anwendungsszenarien von TIFF,
  BMP, GIF, PNG, JPEG und EXIF.
\item Erkl\"aren Sie die JPEG Kompression. Wo entstehen Verluste?
  Erkl\"aren Sie den Zusammenhang zur menschlichen Wahrnehmung.
\answer{[Siehe Vorlesungsfolien}
\end{enumerate}
\end{exercise}


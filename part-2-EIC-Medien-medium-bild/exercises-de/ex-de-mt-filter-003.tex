

\begin{exercise}{Filteroperationen}
\label{ex-de-mt-filter}
Gegeben ist folgendes 8-Bit Graustufenbild:
\[
\begin{pmatrix}
	0 & 0 & 0 & 0 & 0\\
	0 & 0 & 255 & 0 & 0\\
	0 & 0 & 255 & 255 & 255\\
	0 & 0 & 255 & 255 & 255\\
	0 & 0 & 255 & 255 & 255\\
\end{pmatrix}
\]


\answer{
Als Bild: 
\includegraphics[width=.1\textwidth]{figure/binary-image-2}
}


Wenden sie folgende Filter auf dieses Bild an:
\begin{itemize}
  \item $3\times 3$ Medianfilter
  
  
  \answer{
  \[
\begin{pmatrix}
	0 & 0 & 0 & 0 & 0\\
	0 & 0 & \mathbf{0} & 0 & 0\\
	0 & 0 & 255 & 255 & 255\\
	0 & 0 & 255 & 255 & 255\\
	0 & 0 & 255 & 255 & 255\\
\end{pmatrix}
\]
Als Bild:\\
\includegraphics[width=.1\textwidth]{figure/binary-image-median}
  }
  
  \item $3\times 3$- Boost Filter
  
    \answer{
 
 
 
    Boost-Filter:\\
    \[
    B^{neu}(x,y) = \begin{pmatrix}
    -1  & -1 & -1 \\
    -1 & 9   & -1 \\
    -1  & -1  & -1 \\
    \end{pmatrix}
    * N_8(x,y)
    \] 
    Bild nach Anwendung:
     \[
\begin{pmatrix}
	0 & -255   & -255  & -255   & 0\\
	0 & -2*255 & 7*255 & -4*255 & -3*255\\
	0 & -3*255 & 5*255 & 3*255 & 4*255\\
	0 & -3*255 & 4*255 & 255 & 255\\
	0 & -3*255 & 4*255 & 255 & 255\\
\end{pmatrix}
\]
Die Werte sind au�erhalb des darstellbaren Bereichs. Es gibt zwei M�glichkeiten damit umzugehen:
\begin{itemize}
  \item Clipping:


\[
 B(x,y) = \begin{cases}
        0  & B(x,y)<0 \\
        255  & B(x,y)>255 \\
        B(x,y)  & sonst
        \end{cases}
\]

  \item Skalierung $B(x,y)=\frac{B(x,y)-B_{min}}{B_{max}-B_{min}}$
\end{itemize}
Unter Anwendung von Clipping ergibt sich:
  \[
\begin{pmatrix}
	0 & 0 & 0 & 0 & 0\\
	0 & 0 & 255 & 0 & 0\\
	0 & 0 & 255 & 255 & 255\\
	0 & 0 & 255 & 255 & 255\\
	0 & 0 & 255 & 255 & 255\\
\end{pmatrix}
\]

 }
  \item $3\times 3$- Laplace Filter (ohne 45� Geradendetektion, d.h. 4er Nachberschaft)
  
  \answer{
  Laplace-Filter:\\
    \[
    B^{neu}(x,y) = \begin{pmatrix}
    0  & -1 & 0 \\
    -1 & 4   & -1 \\
    0  & -1  & 0 \\
    \end{pmatrix}
    * N_8(x,y)
    \] 
    Bild nach Anwendung:
     \[
\begin{pmatrix}
	0 & 0    & -255  &  0     & 0\\
	0 & -255 & 3*255 & -2*255 & -255\\
	0 & -255 & 255   & 255    & 255\\
	0 & -255 & 255   & 0      & 0\\
	0 & -255 & 255   & 0      & 0\\
\end{pmatrix}
\]
Die Werte sind au�erhalb des darstellbaren Bereichs
Unter Anwendung von Clipping ergibt sich:
  \[
\begin{pmatrix}
	0 & 0 & 0 & 0 & 0\\
	0 & 0 & 255 & 0 & 0\\
	0 & 0 & 255 & 255 & 255\\
	0 & 0 & 255 & 0 & 0\\
	0 & 0 & 255 & 0 & 0\\
\end{pmatrix}
\]
Als Bild:\\

\includegraphics[width=.1\textwidth]{figure/binary-image-laplace}
  }
  \item $3\times 3$- Gaussfilter
   \answer{
  Gau�-Filter:\\
    \[
    B^{neu}(x,y) = 1/16*\begin{pmatrix}
    1 & 2 & 1 \\
    2 & 4 & 2 \\
    1  & 2  & 1 \\
    \end{pmatrix}
    * N_8(x,y)
    \] 
    Bild nach Anwendung:
     \[
\begin{pmatrix}
	0 & 255/16   & 2/16*255   & 1/16*255 & 0\\
	0 & 3/16*255 & 7/16*255   & 6/16*255 & 0\\
	0 & 4/16*255 & 11/16*255  & 13/16*255   & 12/16*255\\
	0 & 4/16*255 & 12/16*255  & 16/16*255   & 16/16*255\\
	0 & 4/16*255 & 12/16*255  & 16/16*255   & 16/16*255\\
\end{pmatrix}
\]
Als Bild:
\includegraphics[width=.1\textwidth]{figure/binary-image-gauss}
  }
  \item (Optional, 1-Pkt.) Verwenden Sie ein beliebiges RGB Bild um in der Bildbearbeitungssoftware GIMP den Effekt der oben genannten Filter zu zeigen (Men�punkt Filter, Generic, Convolution Matrix). Erl�utern Sie die Effekte der Normalisierung sowie der Offset Funktion. 
\end{itemize}

Hinweis: Setzen sie das Bild an den R�ndern einfach fort

\end{exercise}


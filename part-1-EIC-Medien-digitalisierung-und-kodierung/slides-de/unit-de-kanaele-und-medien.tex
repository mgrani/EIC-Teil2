%%% DATE.  January, 4th, 2014

\clearpage


%%%%%%%%%%%%%%% METADATA %%%%%%%%%%%%%%%%
\bsunitname{Begriffsbestimmung}
\setcounter{bsunit}{2}

%%% Unit statistics.
%%%
%%% corollary:   0
%%% definition:  0
%%% lemma:       0
%%% page:        0
%%% proof:       0
%%% theorem:     0


%%%%%%%%%%%%%%%%%%%%%%%%%%%%%%%%%%%%%%%%%%%%%%%%%%%%%%%%%%%%%%%%%%%%%%%%
%%% SOURCE. \cf[Kapitel 2.1]{Malak, Butz, Hussmann - Einfuehrung Medieninformatik}
%%%%%%%%%%%%%%%%%%%%%%%%%%%%%%%%%%%%%%%%%%%%%%%%%%%%%%%%%%%%%%%%%%%%%%%%
\renewcommand{\textbufferB}{Begriffsbestimmung - Digitale Medien}
\begin{bsslide}
  \bspar1
  \begin{center}
 \vspace{0.4\textheight}
    \large\textbufferB
  \end{center}
\end{bsslide}
%%%%%%%%%%%%%%%%%%%%%%%%%%%%%%%%%%%%%%%%%%%%%%%%%%%%%%%%%%%%%%%%%%%%%%%%
%%%%%%%%%%%%%%%%%%%%%%%%%%%%%%%%%%%%%%%%%%%%%%%%%%%%%%%%%%%%%%%%%%%%%%%%
%%%%%%%%%%%%%%%%%%%%%%%%%%%%%%%%%%%%%%%%%%%%%%%%%%%%%%%%%%%%%%%%%%%%%%%%


\begin{bsslide}[Begriffsbestimmung Digitale Medien und Multimedia]
  \colortext{Was sind Digitale Medien?}
  \bspar1
  \begin{definition}[Digitale Medien]
    Digitale Medien sind Medien, welche in digitaler Form
    gespeichert, verarbeitet oder manipuliert werden.
  \end{definition}
  \bspar3
  Typische Beispiele digitaler Medien und Computer-gest\"utzter Funktionen:
  \begin{itemize}
    \item\textbf{ Datentr\"ager/Darstellungstechniken}: eBooks, digitales
    Fernsehen, digitale Bilder
    \item \textbf{Neue Medien und Interaktionsparadigmen}:  World Wide Web, Hypertext,
    Ubiquitous Computing
    \item \textbf{Erstellung/Manipulation von Inhalten}: Rechner-basierte Authoring
    Tools und Programme zur Medienbearbeitung
  \end{itemize}
  \end{bsslide}
  
\begin{bsslide}[Begriffsbestimmung Digitale Medien  und Multimedia]
    \colortext{Charakterisierung medialer Angebote nach ISO/IEC-Standard MHEG}
    \bspar1
    Wie kann man Digitale Medien unterscheiden?
    \bspar2
    \bsfigure{charakterisierung-von-medien}
    \bspar1 \small
    MHEG = Multimedia und Hypermedia Expert Group
  \end{bsslide}
  
  \begin{bsslide}[Begriffsbestimmung Digitale Medien und Multimedia]
    \colortext{Charakterisierung medialer Angebote nach ISO/IEC-Standard MHEG}
    \bspar1
    Charakterisierung medialer Angebote nach ISO/IEC Standard "`MHEG"'
    \begin{itemize}
      \item \textbf{Pr\"asentationsmedium}: Ger\"ate zur Ein- und Ausgabe von Information (e.g. Beamer, Drucker, Kamera etc.)
      \item \textbf{Repr\"aentationsmedium}: Kodierung von Information
      \begin{itemize}
        \item Kodierung im Sinne der Informatik: Text als ASCII, Bild als JPEG
        \item Kodierung im Sinne der Medienpsychologie: Symbolsysteme (Gr\"un als Farbe f\"ur Hoffnung etc.)
      \end{itemize}
      \item \textbf{Perzeptionsmedium:} Modalit\"at (Sinneskanal) der von  Mensch f\"ur die Informationsaufnahme genutzt wird
      \begin{itemize}
        \item Sehen, H\"oren, F\"uhlen$^*$, Riechen$^*$, Schmecken$^*$
        \item Die Modalit\"at ist \textbf{unabh\"angig} von der Kodierung\\
        z.B. Noten  als Zahlen repr\"asentiert, k\"onnen von mir in der VO vorgelesen oder per E-Mail versendet werden
        \item Auf semiotischer Ebene (d.h. die Interpretation der Zeichen) k\"onnen unterschiedliche Zusatzinformationen pro Modalit\"at erfasst werden (e.g. Sympathie/Antipathie f\"ur einen Nachrichtensprecher)
      \end{itemize}
    \end{itemize}
  \end{bsslide}
  
  \begin{bsslide}[Begriffsbestimmung Digitale Medien und Multimedia]
    \colortext{Charakterisierung medialer Angebote nach ISO/IEC-Standard MHEG}
    \bspar1
    \begin{itemize}
      \item \textbf{Speichermedium} - Worauf wird Information gespeichert?
      \begin{itemize}
        \item Digitales Speichermedium: USB Stick, Festplatte, DVD
        \item Analgoes Medium:  Schallplatte, Papier, Photo-Negativ
      \end{itemize}
      \item \textbf{\"Ubertragungsmedium} - Wor\"uber wird Information \"ubertragen
      \begin{itemize}
        \item Kupferkabel, Glasfaser, Luft 
        \item Im Detail: Protokolle, Modulationsverfahren, Bandbreite, Frequenz etc.
      \end{itemize}(Radiowellen, UMTS, Kabel)
      \item \textbf{Informationsaustauschmedium} - Welcher Tr\"ager wird f\"ur den Austausch von Information zwischen verschiedenen Orten verwendet
      \begin{itemize}
        \item Oberbegriff von bestimmten Speicher- und \"Ubertragungsmedien
      \end{itemize}
    \end{itemize}
  \end{bsslide}

 \begin{bsslide}[Begriffsbestimmung Digitale Medien und Multimedia]
    \colortext{Charakterisierung medialer Angebote nach ISO/IEC-Standard MHEG}
    \bspar1
    \bsfigure{charakterisierung-von-medien-2}
    \bspar1
  \end{bsslide}

%%%%%%%%%%%%%%%%%%%%%%%%%%%%%%%%%%%%%%%%%%%%%%%%%%%%%%%%%%%%%%%%%%%%%%%%
%%%%%%%%%%%%%%%%%%%%%%%%%%%%%%%%%%%%%%%%%%%%%%%%%%%%%%%%%%%%%%%%%%%%%%%%
%%%%%%%%%%%%%%%%%%%%%%%%%%%%%%%%%%%%%%%%%%%%%%%%%%%%%%%%%%%%%%%%%%%%%%%%
\renewcommand{\textbufferB}{Begriffsbestimmung - Multimedia}
\begin{bsslide}
  \bspar1
  \begin{center}
 \vspace{0.4\textheight}
    \large\textbufferB
  \end{center}
\end{bsslide}
%%%%%%%%%%%%%%%%%%%%%%%%%%%%%%%%%%%%%%%%%%%%%%%%%%%%%%%%%%%%%%%%%%%%%%%%
%%%%%%%%%%%%%%%%%%%%%%%%%%%%%%%%%%%%%%%%%%%%%%%%%%%%%%%%%%%%%%%%%%%%%%%%
%%%%%%%%%%%%%%%%%%%%%%%%%%%%%%%%%%%%%%%%%%%%%%%%%%%%%%%%%%%%%%%%%%%%%%%%


   \begin{bsslide}[Begriffsbestimmung Digitale Medien und Multimedia]
     \colortext{Was bedeutet Multimedia?}
     \bspar1
     Multimedia = "`Viele Vermittler"' als Definition unzureichend 
     \bspar2
     Mehrer Definitionen aus der Literatur:
     \begin{itemize}
       \item "`Der Begriff Multimedia bezeichnet Inhalte und Werke, die aus mehreren, meist digitalen Medien bestehen: Text, Fotografie, Grafik, Animation, Audio und Video."' (de.wikipedia.org, 17.10.2010)
       \item "`Multimedia is media and content that uses a combination of different content forms." (en.wikipedia.org, 08.10.2009)
       \item Multimedia ist der Trend, die verschiedenen Kommunikationskan\"ale des Menschen mit den Mitteln der Informationswissenschaft \"uber alle Quellen zu integrieren und als Gesamtheit f\"ur die Kommunikation zu nutzen. (sinngem. nach P. Henning) 
     \end{itemize}
     \bspar2
     Definition alleine i.A. nicht ausreichend, daher ist es wichtig die \textbf{Charakterisierung} des Begriffs zu betrachten
  \end{bsslide}
  
  
  \begin{bsslide}[Begriffsbestimmung Digitale Medien und Multimedia]
    \colortext{Was bedeutet Multimedia?}
    \bspar1
    \small
    Mediale Angebote k\"onnen als \textbf{monomedial} oder \textbf{multimedial} charakterisiert werden
    \begin{itemize}
      \item \textbf{Pr\"asentationsmedium:} behandelt das physikalische Ausgabemedium (die Ausgabeger\"ate)
      \begin{itemize}
        \item Monomedial: nur ein Ausgabemedium (e.g Radio, Buch)
        \item Multimedia: mehrere Ausgabemedien (e.g. Fernsehger\"at, Computer)
      \end{itemize}
      \bspar1
      \item \textbf{Repr\"asentationsmedium:} behandelt die Kodierung von Information
      \begin{itemize}
        \item Monocodal: nur eine Repr\"asentationsart (e.g. Bedienungsanleitung nur durch Text oder nur durch grafische Elemente)
        \item Multicodal: mehrere Repr\"asentationsarten (e.g. Text und grafische Elemente in einer Bedienungsanleitung)
      \end{itemize}
      \bspar1
      \item Die \textbf{Perzeptionsebene} unterscheidet nach der Anzahl der bedienten menschlichen Sinneskan\"ale
      \begin{itemize}
        \item Monomodal: nur eine Sinneskanal wird angesprochen (e.g. Bedienungsanleitung)
        \item Multimodal: mehrere Sinneskan\"ale werden angesprochen (e.g. Video mit Bedienungsanweisungen)
      \end{itemize}
    \end{itemize}
    \small Einteilung nach Weidemann
  \end{bsslide}
  
  \begin{bsslide}[Begriffsbestimmung Digitale Medien und Multimedia]
    \colortext{Was bedeutet Multimedia?}
    \bspar1
    \textbf{Zeit} ist eine wichtige Dimension in der Medienrepr\"asentation und klassifiziert ein Medium in
    \begin{itemize}
      \item Zeit-unabh\"angig (Diskret): z.B. Text, Bild
      \item Zeit-abh\"angig (Kontinuierlich): Audio, Video
    \end{itemize}
    Diskret/Kontinuierlich beziehen sich auf die Wahrnehmung durch den Benutzer, nicht die interne Repr\"asentation
    \bspar1
    Interaktion stellt ein weiteres zentrales Kennzeichen von Multimedia Medien dar und f\"uhrt zu folgender Charakterisierung:
    \begin{itemize}
      \item \textbf{Lineare Pr\"asentation}: Es gibt keine Interaktion zwischen dem Benutzer und dem Medium (z.B. Kinofilm, Radio)
      \item \textbf{Nicht linear Pr\"asentation}: der Benutzer hat die Navigationskontrolle \"uber das Medium (z.B. Spiele, Interaktive CD)
    \end{itemize}
  \end{bsslide}
  
  \begin{bsslide}[Begriffsbestimmung Digitale Medien und Multimedia]
    \colortext{Zusammenfassung}
    \bspar1
    
    \begin{itemize}
      \item  Unterscheidung Digitaler Medien nach Medientypen
      \item Charakterisierung von Multimedia auf verschiedenen technischen Ebenen (Pr\"asentation, Kodierung, Wahrnehmung) und entlang zeitlicher Dimension (diskret/kontinuierlich und linear/nicht-linear)
    \end{itemize}
    \bspar2
    Verarbeitung von Medien ben\"otigt Flexibilit\"at. Im Grunde m\"ussen Signale unterschiedlicher Modalit\"aten verarbeitet werden. Durch Digitalisierung kann dies "`einfach"' realisiert werden. \\
    \bspar4
    
    \centering \textbf{Was bedeutet Digitalisierung und wie k\"onnen
      Signale kodiert werden?}
    
  \end{bsslide}

\begin{bsslide}[Begriffsbestimmung Digitale Medien und Multimedia]
    \colortext{Literatur}
    \begin{itemize}
      \item Malaka, Butz, Hussmann (2009) - Medieninformatik: Eine
        Einf\"uhrung (Pearson Studium - IT), Kapitel 2.1
      \item Weidenmann, B. (1995). \emph{Multimedia, Multicodierung
          und Multimodalit\"at}. In L. J. Issing \& P. Klimsa (Hrsg.),
        Information und Lernen mit Multimedia (S. 65-84). Weinheim:
        PVU.

      \end{itemize}
\end{bsslide}

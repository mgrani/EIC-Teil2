%%% DATE.  November, 17th, 2013

\clearpage

%%%%%%%%%%%%%%% METADATA %%%%%%%%%%%%%%%%
\bsunitname{Informationstheorie}
\setcounter{bsunit}{3}
%%%%%%%%%%%%%%%%%%%%%%%%%%%%%%%%%%%%%%%%%%%%%%%%%%%%%%%%%%%%%%%%%%%%%%%%
%%% SOURCE. \cf[Kapitel 2]{Malak, Butz, Hu�mann - Einf�hrung Medieninformatik}
%%%%%%%%%%%%%%%%%%%%%%%%%%%%%%%%%%%%%%%%%%%%%%%%%%%%%%%%%%%%%%%%%%%%%%%%
\renewcommand{\textbufferB}{Informationstheorie - Ein einf\"uhrendes Beispiel}
\begin{bsslide}
  \bspar1
  \begin{center}
 \vspace{0.4\textheight}
    \large\textbufferB
  \end{center}
\end{bsslide}
%%%%%%%%%%%%%%%%%%%%%%%%%%%%%%%%%%%%%%%%%%%%%%%%%%%%%%%%%%%%%%%%%%%%%%%%
\begin{bsslide}[\textbufferB]
  \colortext{Problemstellung}
  \bspar1
  Wir haben unterschiedliche Kodierungen mit verschiedenen
  Eigenschaften kennen gelernt.
  \bspar2
  Doch wann ist eine Kodierung optimial, d.h. existiert eine Kodierung die mit
  weniger Bit die gleiche Information \"ubermitteln kann?
  \bspar2
  Wie kann man die Menge der in einer Nachricht erhaltenen Information
  quantifizieren?
  \bspar2
  Wie kann man den Informationsgehalt einer Nachrichtenquellen quantifizieren?
\end{bsslide}
%%%%%%%%%%%%%%%%%%%%%%%%%%%%%%%%%%%%%%%%%%%%%%%%%%%%%%%%%%%%%%%%%%%%%%%%
\renewcommand{\textbufferA}{\bspar1
  Nehmen wir an sie haben ein Kartenspiel bestehend aus 4 Karten,
  $K=\{1,2,3,4\}$.  Person A zieht eine Karte. Person B muss \"uber
  m\"oglichst wenige ``Ja/Nein'' Antworten herausfinden, um welche
  Karte es sich handelt.}
\begin{bsslide}[\textbufferB]
  \colortext{Beispiel Kartenspiel}
  \textbufferA
\end{bsslide}
%%%%%%%%%%%%%%%%%%%%%%%%%%%%%%%%%%%%%%%%%%%%%%%%%%%%%%%%%%%%%%%%%%%%%%%%
\begin{bsslide}[\textbufferB]
  \colortext{Beispiel Kartenspiel}
  \textbufferA
  \bspar1
   \bsfigurecaption[1.0]{Codierungstheorie-Beispiel-Code-A}{Beispiel Fragebaum A}
\end{bsslide}
%%%%%%%%%%%%%%%%%%%%%%%%%%%%%%%%%%%%%%%%%%%%%%%%%%%%%%%%%%%%%%%%%%%%%%%%
\begin{bsslide}[\textbufferB]
  \colortext{Beispiel Kartenspiel}
  \textbufferA
  \bspar1
   \bsfigurecaption[1.0]{Codierungstheorie-Beispiel-Code-B}{Beispiel Fragebaum B}
\end{bsslide}
%%%%%%%%%%%%%%%%%%%%%%%%%%%%%%%%%%%%%%%%%%%%%%%%%%%%%%%%%%%%%%%%%%%%%%%%
\begin{bsslide}[\textbufferB]
  \colortext{Beispiel Kartenspiel}
\bspar1
  Jede ``Ja/Nein'' Antwort kann als Bit (``0/1'') verstanden werden,
  wodurch sich folgender Bin\"arcode ergibt:
  \begin{center}
    \begin{tabular}{lrrc}
      \textbf{Karte} &\textbf{Code A} &\textbf{Code B} &\textbf{Informationsgehalt}\\
      \hline
      1                    &                          1&   00  & 2\\
      2                    &                        10&   01  & 2\\
      3                    &                      100&   10  & 2\\
      4                    &                    1000&   11  & 2\\
    \end{tabular}
  \end{center}
  \bspar1
  Beobachtungen:
  \begin{itemize}
  \item  Code B scheint effizienter, da er im Schnitt weniger Fragen (Bits)
    ben\"otigt.
  \item  Code B scheint optimal zu sein, da wir f\"ur 4 Zust\"ande 2
    Bits ben\"otigen ($N = 2^{\#bits}$ bzw. $\#bits = \log_2 (N)$ mit $N$
    als Anzahl der Karten)
  \item Alternativ: In Code B reduziert jede Frage die Unsicherheit um 50 \%
    d.h. die Anzahl der verbleibenden M\"oglichkeiten wird um die
    H\"alfte reduziert
  \item Der Informationsgehalt einer Karte kann als minimale
    Anzahl von  Ja/Nein Fragen aufgefasst werden.
  \end{itemize}
\end{bsslide}
%%%%%%%%%%%%%%%%%%%%%%%%%%%%%%%%%%%%%%%%%%%%%%%%%%%%%%%%%%%%%%%%%%%%%%%%
\renewcommand{\textbufferA}{\bspar1
  Ein neues Spiel hat folgende 4 Karten: $K=\{1,1,2,3\}$, d.h. Karte 1
  kommt 2x vor. Wie \"andert sich der Code?}
%%%%%%%%%%%%%%%%%%%%%%%%%%%%%%%%%%%%%%%%%%%%%%%%%%%%%%%%%%%%%%%%%%%%%%%%
\begin{bsslide}[\textbufferB]
  \colortext{Beispiel Kartenspiel}
  \textbufferA
\end{bsslide}
%%%%%%%%%%%%%%%%%%%%%%%%%%%%%%%%%%%%%%%%%%%%%%%%%%%%%%%%%%%%%%%%%%%%%%%%
\renewcommand{\textbufferA}{\bspar1
  Ein neues Spiel hat folgende 4 Karten: $K=\{1,1,2,3\}$, d.h. Karte 1
  kommt 2x vor.
  \begin{center}
    \begin{tabular}{lrrc}
      \textbf{Karte} &\textbf{Code B} &\textbf{Code C}  &\textbf{Informationsgehalt}\\
      \hline
      1                    &                    00&   0  & 1\\
      1                    &                    01&   0  & 1\\
      3                    &                    10&   01  & 2\\
      4                    &                    11&   11  & 2\\
    \end{tabular}
  \end{center}}
%%%%%%%%%%%%%%%%%%%%%%%%%%%%%%%%%%%%%%%%%%%%%%%%%%%%%%%%%%%%%%%%%%%%%%%%
\begin{bsslide}[\textbufferB]
  \colortext{Beispiel Kartenspiel}
  \textbufferA
  \bspar1
  \textbf{Fragen}:
  \begin{itemize}
  \item \textbf{Bit 0}: Ist es nicht 1?
  \item \textbf{Bit 1}: Ist es 4?
    \end{itemize}
  Der Code ist optimal, da beide Fragen die Unsicherheit um 50\%
  reduzieren, d.h. die Anzahl der verbleibenden M\"oglichkeiten auf
  die H\"alfte reduzieren.
\end{bsslide}
%%%%%%%%%%%%%%%%%%%%%%%%%%%%%%%%%%%%%%%%%%%%%%%%%%%%%%%%%%%%%%%%%%%%%%%%
\begin{bsslide}[\textbufferB]
  \colortext{Beispiel Kartenspiel}
  \textbufferA
  \bspar1
  \textbf{Beobachtungen}:
  \begin{itemize}
  \item  Code C scheint effizienter, da er im Schnitt weniger Fragen (Bits)
    ben\"otigt.
  \item 2 Karten haben dieselbe Bedeutung.
    \begin{itemize}
    \item Da es wahrscheinlicher
    ist, Karte 1 zu ziehen, ist der Informationsgehalt von Karte 1
    jedoch geringer.
    \item Alternativ: Da Karte 1 \"ofters gezogen wird, wollen wir der
      Karte einen k\"urzeren Code geben.
 \end{itemize}
 \end{itemize}
\end{bsslide}
%%%%%%%%%%%%%%%%%%%%%%%%%%%%%%%%%%%%%%%%%%%%%%%%%%%%%%%%%%%%%%%%%%%%%%%%
\begin{bsslide}[\textbufferB]
  \colortext{Beispiel Kartenspiel}
  \textbufferA
  \bspar1
  Wie hoch ist der Informationsgehalt von Karte 1?
  \begin{itemize}
  \item F\"ur 4 Karten benotigen wir 2 bit ($\log_2 (4)$).
  \item Karte 1 w\"urde dabei 2 Codes erhalten, die jedoch  die
    gleiche Bedeutung haben und somit redundant sind. D.h. 1 Frage/Bit ($\log_2(2)$) ist redundant.
  \item $IG(Karte_1) = \log_2 (4)-\log_2 (2)=\log_2 (\frac{4}{2})=\log_2
    (\frac{1}{p(Karte_1)})$\\
    wobei $p(Karte_1)$ die Wahrscheinlichkeit darstellt, dass $Karte_1$
    gezogen wird.
\item Analog f. Karte 3: $IG(Karte_3) = \log_2 (4)-\log_2 (1)=\log_2 (\frac{4}{1})=\log_2
    (\frac{1}{p(Karte_3)})$
  \end{itemize}
\end{bsslide}
%%%%%%%%%%%%%%%%%%%%%%%%%%%%%%%%%%%%%%%%%%%%%%%%%%%%%%%%%%%%%%%%%%%%%%%%
\renewcommand{\textbufferA}{\bspar1
  Nehmen wir an sie haben ein Kartenspiel bestehend aus 4 Karten,
  $K=\{1,2,3,4\}$.  Person A zieht eine Karte. Person B muss \"uber
  m\"oglichst wenige ``Ja/Nein'' Antworten herausfinden, um welche
  Karte es sich handelt.}

%%%%%%%%%%%%%%%%%%%%%%%%%%%%%%%%%%%%%%%%%%%%%%%%%%%%%%%%%%%%%%%%%%%%%%%%
\renewcommand{\textbufferB}{Informationstheorie - Formale Definition}
\begin{bsslide}
  \bspar1
  \begin{center}
 \vspace{0.4\textheight}
    \large\textbufferB
  \end{center}
\end{bsslide}
%%%%%%%%%%%%%%%%%%%%%%%%%%%%%%%%%%%%%%%%%%%%%%%%%%%%%%%%%%%%%%%%%%%%%%%%
\begin{bsslide}[\textbufferB]
  \colortext{�berblick}
  \bspar1
  \begin{itemize}
  \item Die Informationstheorie nach Shannon wurde 1948 von \textbf{Claude
      Shannon} entwickelt.
  \item Sie stellt ein \textbf{mathematisches Modell} f\"ur
    Kommunikationssysteme dar.
  \item  Dabei werden Nachrichten als
    \textbf{stochastische Ereignisse}, die von einer
    Nachrichtenquelle, dem Sender, ausgesendet und von einem
    Empf\"anger empfangen werden. \\[-1ex]
    \begin{itemize}
    \item Stochastische Prozesse sind zeitlich geordnete, zuf�llige
      Prozesse die mit Hilfe der Statistik und
      Wahrscheinlichkeitsrechnung beschrieben werden
    \item Nachrichten bestehen somit aus einer endlichen Anzahl
      diskreter Zeichen, welche mit einer
      gewissen Wahrscheinlichkeit auftreten.
    \end{itemize}
  \end{itemize}
\end{bsslide}
%%%%%%%%%%%%%%%%%%%%%%%%%%%%%%%%%%%%%%%%%%%%%%%%%%%%%%%%%%%%%%%%%%%%%%%%
\begin{bsslide}[\textbufferB]
  \colortext{�berblick}
  \bspar1
  Kernelemente der Informationstheorie:
  \begin{itemize}
  \item Definition einer \textbf{Nachrichtenquellen} und deren Eigenschaften
  \item Definition des \textbf{Informationsgehaltes bzw. der Information eines Zeichens, $I(x)$}
  \item Definition der \textbf{Entropie $H(X)$}, d.h. der Informationsdichte bzw. des
    durchschnittlichen Informationsgehaltes einer Nachrichtenquelle
   \item Definition von \textbf{optimalen Kodierungen }und darauf aufbauender \textbf{Redundanz}
  \end{itemize}
\end{bsslide}

%%%%%%%%%%%%%%%%%%%%%%%%%%%%%%%%%%%%%%%%%%%%%%%%%%%%%%%%%%%%%%%%%%%%%%%%
  \begin{bsslide}[\textbufferB]
    \colortext{Nachrichtenquelle}
    \bspar1
    \begin{definition}[Diskrete, ged\"achtnislose Nachrichtenquelle]
      Eine diskrete, ged\"achtnislose Nachrichtenquelle setzt in jedem
      Zeittakt ein Zeichen $a_i \in A$ aus einem Zeichenvorrat $A$ mit
      der Wahrscheinlicheit $P(a_i)=p_i$ ab. Die Auswahl der Zeichen
      ist unabh\"angig von bereits emitierten Zeichen in der Vergangenheit.
    \end{definition}
    \bspar1
    Beispiel mit 3 Quellen, Zeichenvorrat $\{A,B,C,D\}$:
      \begin{center}
      \begin{tabular}{|l|c|c|c|c|}
        \hline
        Zeichen $a$  	&  A 	& B		& C		&D		\\
\hline
Wahrscheinlichkeit $p_a$ in Quelle 1 & 1.0 & 0.0 & 0.0 & 0.0 \\
Wahrscheinlichkeit $p_a$ in Quelle 2 & 0.25 & 0.25 & 0.25 & 0.25 \\
Wahrscheinlichkeit $p_a$ in Quelle 3 & 0.5 & 0.25 & 0.125 & 0.125 \\
\hline
      \end{tabular}
    \end{center}
      \bspar1
      \begin{itemize}
      \item Welche Quelle �bertr�gt die meiste Information in einer
        Nachricht?
      \item  Wieviel Bit ben�tigen wir zur Kodierung eines Zeichens einer
        Quelle?
       \item Wie hoch ist die Information (bzw. der
         Informationsgehalt) eines Zeichens?
      \end{itemize}
  \end{bsslide}
 %%%%%%%%%%%%%%%%%%%%%%%%%%%%%%%%%%%%%%%%%%%%%%%%%%%%%%%%%%%%%%%%%%%%%%%%
  \begin{bsslide}[\textbufferB]
    \colortext{Informationsgehalt eines Zeichen}
    \bspar1
    \textbf{Axiomatische Definition}\footnote{Axiome sind die Grunds\"atze
      bzw. fixen Annahmen einer Theorie. In einer Axiomatisch
      Definition werden dies a-priori festgelegt. Aus den Axiomen
      werden anschlie\ss end deduktiv weitere Eigenschaften
      abgeleitet.} des Informationsgehalts \"uber 3 Axiome
    (grunds\"atzliche Eigenschaften):
    \begin{enumerate}
    \item[(I)] Der Informationsgehalt eines Zeichens $a\in A$ mit der
      Wahrscheinlicheit $p_i$ ist ein nicht-negatives Ma\ss, welches nur
      von der Wahrscheinlichkeit des Zeichens abh\"angt: $I(p_i)\geq
      0$
     \item[(II)] Der Informationsgehalt zweier voneinander unabh\"angiger
       Zeichen $a_i$ und $a_j$ mit den Wahrscheinlichkeiten $p_i$ und
       $p_j$ addiert sich: $I(p_i, p_j)=I(p_i) + I (p_j)$
     \item[(III)] Der Informationsgehalt ist eine stetige Funktion der
       Wahrscheinlichkeiten der Zeichen.
    \end{enumerate}
  \end{bsslide}

 %%%%%%%%%%%%%%%%%%%%%%%%%%%%%%%%%%%%%%%%%%%%%%%%%%%%%%%%%%%%%%%%%%%%%%%%
\renewcommand{\textbufferA}{\bspar1
    \begin{theorem}[Informationsgehalt/Entscheidungsgehalt eines Zeichens]
    Sei $p_i$ die Wahrscheinlichkeit f�r das Auftreten von Zeichen $a_i\in A$. Der Entscheidungsgehalt $x_i$  f�r das Zeichen $a_i\in A$ ist definiert als
      \[2^{x_a}=1/p_a \Rightarrow x_a = \log_2(1/p_a)\]
    Die Einheit ist bit (basic indissoluble\footnote{unaufl�slich} information unit) pro Zeichen. n bit entsprechen dabei mindestens n Bit (Binary Digits)
    \end{theorem}}

 \begin{bsslide}[\textbufferB]
    \colortext{Informationsgehalt eines Zeichen}
    \textbufferA
    \bspar1
     \begin{center}
      \begin{tabular}{|l|c|c|c|c|}
        \hline
        Zeichen $a$  	&  A 	& B		& C		&D		\\
\hline
$p_a$ in Quelle 1 & 1.0 & 0.0 & 0.0 & 0.0 \\
$x_a$ (in bit) in Quelle 1 & 0 & undef. & undef. & undef. \\
\hline
$p_a$ in Quelle 2 & 0.25 & 0.25 & 0.25 & 0.25 \\
$x_a$ (in bit)in Quelle 2 & 2 & 2 & 2 & 2 \\
\hline
$p_a$ in Quelle 3 & 0.5 & 0.25 & 0.125 & 0.125 \\
$x_a$ (in bit)in Quelle  & 1 & 2 & 3 & 3 \\
\hline
      \end{tabular}
     \end{center}
  \end{bsslide}
%%%%%%%%%%%%%%%%%%%%%%%%%%%%%%%%%%%%%%%%%%%%%%%%%%%%%%%%%%%%%%%%%%%%%%%%
 \begin{bsslide}[\textbufferB]
    \colortext{Informationsgehalt eines Zeichen}
    \textbufferA
    \bspar1\small
    \textbf{Eigenschaften:}
    \begin{itemize}
    \item Ein sicheres Ereignis ($p_i=1.0$) besitzt keine Information.
    \item Unwahrscheinliche Ereignisse besitzen viel Information.
    \item Der Informationsgehalt spiegelt die Axiome wider, wobei der
      Logarithmus f\"ur die Additivit\"at (Axiom II) sorgt
    \item Die Basis des Logarithmus definiert die Einheit. Bei einem
      $\log_{10}$ w\"uerden Zeichen in einem Dezimalcode kodiert
      werden.
   \item Ein bit definiert die kleinste Einheit und kann als
     ``Ja/Nein'' Entscheidung verstanden werden.

    \end{itemize}
  \end{bsslide}
 %%%%%%%%%%%%%%%%%%%%%%%%%%%%%%%%%%%%%%%%%%%%%%%%%%%%%%%%%%%%%%%%%%%%%%%%
  \begin{bsslide}[\textbufferB]
    \colortext{Durchschnittliche Informationsgehalt einer Quelle}
    \bspar1
    \begin{definition}[Entropie (durchschnittlichen Entscheidungsgehalt)]
      Sei $A$ ein Zeichenvorrat einer diskreten, ged\"achtnislosen
      Quelle und sei $p_i$ die Wahrscheinlichkeit f�r das Auftreten
      von Zeichen $a_i\in A$. Dann definiert die Entropie $H$ den \emph{durchschnittlichen
        Informationsgehalt der Quelle} als:\\
      \centering{$H=\sum_{a\in A}p_a*\log_2(\frac{1}{p_a})$}
    \end{definition}
    \bspar2\small
    \textbf{Bemerkungen}
    \begin{itemize}
      \item $p_a*\log_2(\frac{1}{p_a})=p_a*x_a$ bedeutet das der
        Entscheidungsgehalt eines Zeichens noch mit dessen
        Auftrittswahrscheinlichkeit gewichtet wird
        \begin{itemize}
        \item Seltene Zeichen besitzen hohe Information, werden jedoch
          nicht oft gesendet.
       \item H\"aufige Zeichen besitzen niedrige Information, werden
         jedoch h\"aufig gesendet.
        \end{itemize}
      \item Die Entropie gibt die minimale Anzahl von bit's an die
        notwendig sind, um
        Nachrichten einer diskreten, ged\"achtnislosen Quelle zu
        kodieren. Es gibt keine Kodierung mit weniger bit's.
      \item Entropie kann im physikalischen Sinn der Thermodynamik auch als
        Unordnung eines Systems verstanden werden.
      \item Alternative Darstellung: $\mathbf{-}p_a*\mathbf{\log_2{p_a}}$
    \end{itemize}
    \textbf{Frage:} Wann ist die Entropie maximal?
  \end{bsslide}
  %%%%%%%%%%%%%%%%%%%%%%%%%%%%%%%%%%%%%%%%%%%%%%%%%%%%%%%%%%%%%%%%%%%%%%%%
  \begin{bsslide}[\textbufferB]
    \colortext{Informationsgehalt einer Quelle}
    \bspar1
    \begin{theorem}[Maximalit�t der Entropie]
      Sei $A$ ein Zeichenvorrat einer Quelle und $p_a$ die Wahrscheinlichkeit f�r das Auftreten von Zeichen $a\in A$. Die \textbf{Entropie-Funktion ist maximal} wenn gilt\\
      \begin{center}$\forall_{i,j\in A}| p_i=p_j$\\\end{center}
      d.h. die \textbf{Auftrittswahrscheinlichkeit f�r alle Zeichen gleich gro� ist.}
    \end{theorem}
    \bspar1
    Darstellung f�r den Fall von zwei Zeichen $A,B$ ($p_a=1-p_b$):
    \bsfigure[scale=0.6]{entropie}
  \end{bsslide}
 %%%%%%%%%%%%%%%%%%%%%%%%%%%%%%%%%%%%%%%%%%%%%%%%%%%%%%%%%%%%%%%%%%%%%%%%
  \begin{bsslide}[\textbufferB]
    \colortext{Beispiel Informationsgehalt einer Quelle}
    \bspar1
    \begin{center}
      \begin{tabular}{|l|c|c|c|c|}
        \hline
        Zeichen $a$  	&  A 	& B		& C		&D		\\
\hline
 $p_a$ in Quelle 1 & 1.0 & 0.0 & 0.0 & 0.0  \\
$x_a$ (in bit) in Quelle 1 & 0 & undef. & undef. & undef. \\
\hline
 $p_a$ in Quelle 2 & 0.25 & 0.25 & 0.25 & 0.25 \\
$x_a$ (in bit)in Quelle 2 & 2 & 2 & 2 & 2 \\
\hline
 $p_a$ in Quelle 3 & 0.5 & 0.25 & 0.125 & 0.125 \\
$x_a$ (in bit)in Quelle 3 & 1 & 2 & 3 & 3 \\
\hline
      \end{tabular}
    \end{center}
    Entropie:
    \begin{itemize}
      \item $H_{Quelle 1}=0$
      \item $H_{Quelle 2}=2$
      \item $H_{Quelle 3}=1.75$
    \end{itemize}
  \end{bsslide}
%%%%%%%%%%%%%%%%%%%%%%%%%%%%%%%%%%%%%%%%%%%%%%%%%%%%%%%%%%%%%%%%%%%%%%%%
\renewcommand{\textbufferB}{Redundanz}
\begin{bsslide}
  \bspar1
  \begin{center}
 \vspace{0.4\textheight}
    \large\textbufferB
  \end{center}
\end{bsslide}
%%%%%%%%%%%%%%%%%%%%%%%%%%%%%%%%%%%%%%%%%%%%%%%%%%%%%%%%%%%%%%%%%%%%%%%%
  \begin{bsslide}[\textbufferB]
    \colortext{Redundanz einer Kodierung}
    \bspar1
    \textbf{Was bedeutet Redundanz?}
    \begin{itemize}
      \item Redundanz bedeutet, da� es �berfl�ssige Nachrichtenteile gibt
      \item Redundante Kodierungen sind nicht optimal. Dies kann
        gewollt sein (z.B. zur Fehlerkorrektur) oder ungewollt,
        wodurch mehr Speicher als n\"otig verwendet wird.
    \end{itemize}
    Wie hoch ist die Redundanz einer gegebenen Kodierung $c$ f�r eine Nachrichtenquelle?
    \bspar1
    Intuitive Vorgehenswei�e:
    \begin{itemize}
      \item Ermittle die durchschnittliche (Wort)L�nge einer Kodierung in Bits ($L$)
      \item Vergleiche diese mit dem Informationsgehalt (Entropie) einer Quelle ($H$)
    \end{itemize}
  \end{bsslide}
%%%%%%%%%%%%%%%%%%%%%%%%%%%%%%%%%%%%%%%%%%%%%%%%%%%%%%%%%%%%%%%%%%%%%%%%
  \begin{bsslide}[\textbufferB]
    \colortext{Wortl�nge}
    \bspar1
    Was ist die durchschnittliche L�nge einer Kodierung in Bits?
    \bspar2
    \begin{definition}[Wortl�nge (Nachrichtenl�nge)]
      Sie $A^*$ die Menge aller endlichen Sequenzen/Worte (Nachrichten) aus einem Zeichenvorrat $A$. \\
      F�r ein Wort (Nachricht) $w\in A^*$ ist die L�nge die Anzahl der in dem Wort (der Nachricht) enthaltenen Zeichen, abgek�rzt durch $|w|$. \\
      Wenn eine Kodierung $c$ einem Zeichen $a\in A$ ein Wort $c(a)\in B^*$ zuweist, dann ist $|c(a)|$ die \textbf{Wortl�nge der Kodierung} des Zeichens $a$
    \end{definition}
    \bspar1
    Beispiele
    \begin{itemize}
      \item $w_1=<010101> \Rightarrow |w| = 6$
      \item $c_1(w_1)=c_1(010101)=<ab> \Rightarrow |w| = 2$
      \item $c_2(w_1)=c_2(010101)=<Medien> \Rightarrow |w| = 6$
    \end{itemize}
    Die Einheit der Wortl�nge eines Bin\"arcodes ist Bit.
  \end{bsslide}
%%%%%%%%%%%%%%%%%%%%%%%%%%%%%%%%%%%%%%%%%%%%%%%%%%%%%%%%%%%%%%%%%%%%%%%%
   \begin{bsslide}[\textbufferB]
    \colortext{Wortl�nge}
    \bspar1
    \begin{definition}[Durchschnittliche Wortl�nge]
      Bei einer Kodierung $c$ einer Nachrichtenqulle ist die \textbf{durchschnittliche Wortl�nge} $L$ die nach Auftrittswahrscheinlichkeit gewichtete Summe der Wortl�ngen aller Kodierungen der Einzelzeichen, als Formel:\\
\begin{center}$L=\sum_{a_\in A}p_a*|c(a)|$\end{center}
    \end{definition}
    \bspar2
    Zusammenhang zur Entropie/Informationsgehalt:
    \begin{itemize}
  \item $|c(a)|$ ist in Bit angegeben (wenn wir von einem Bin�rcode ausgehen)
  \item $\log_{2}(\frac{1}{p_a})$ ist der Informationsgehalt eines Zeichens, ebenfalls in Bit angegeben.
  \item Der Informationsgehalt stellt die Bits eines idealen Code dar, w�hrend $|c(a)|$ die Bitl�nge eines echten Codes darstellt
\end{itemize}
  \end{bsslide}
 %%%%%%%%%%%%%%%%%%%%%%%%%%%%%%%%%%%%%%%%%%%%%%%%%%%%%%%%%%%%%%%%%%%%%%%%
  \begin{bsslide}[\textbufferB]
    \colortext{Beispiel Redundanz}
    \bspar1
    Beispiel f\"ur zwei Kodierungen $c_{1}$ und $c_{2}$
    \bspar1
    \begin{center}
      \begin{tabular}{lcccc}
        \textbf{Zeichen a}   &   \textbf{A}     & \textbf{B}    & \textbf{C}   & \textbf{D}   \\
        \hline
        Kodierung $c_{1}$     &      $00$          &      $01$      &      $10$     & $11$ \\
        Kodierung $c_{2}$     &      $0$           &      $10$      &      $110$     & $111$ \\
        \hline
      \end{tabular}
    \end{center}

    \bspar2
     Beispiel f\"ur eine redundante Kodierungen $c_{1}$
     \bspar1
     \begin{center}
       \begin{tabular}{lcccc}
         \textbf{Zeichen a}   &   \textbf{A}     & \textbf{B}    & \textbf{C}   & \textbf{D}   \\
         \hline
         Wahrscheinlichkeit $p_{a}$ in Quelle $3$    &      $0.5$          &      $0.25$      &      $0.125$     & $0.125$ \\
         Kodierung $c_{1}$     &      $00$          &      $01$      &      $10$     & $11$ \\
         Wortl\"ange             &      $2$            &      $2$        &      $2$     & $2$ \\
         \hline
       \end{tabular}
     \end{center}

    \bspar1
     Durchschnittliche Wortl\"ange $L_{c_{1}}=0.5*2+0.25*2+0.125*2+0.125*2=2$
\bspar1
     Informationsgehalt:  $H_{3}=0.5*1+0.25*2+0.125*3+0.125*3=1.75$
  \end{bsslide}
%%%%%%%%%%%%%%%%%%%%%%%%%%%%%%%%%%%%%%%%%%%%%%%%%%%%%%%%%%%%%%%%%%%%%%%%
  \begin{bsslide}[\textbufferB]
    \colortext{Definition Redundanz}
    \bspar1\small
    \begin{definition}[Redundanz R]
      Die \textbf{Redundanz} $R$ einer bin�ren Kodierung $c$ f�r eine Nachrichtenquelle ist die Differenz der mittleren Wortl�nge und der Entropie:\\
\begin{center}$R=L-H$\end{center}
    \end{definition}
    \bspar1
    \textbf{Optimale Kodierung}: Eine Kodierung hei�t optimal, wenn die Redundanz gleich Null ist ($R=0$)
    \bspar1
     \begin{center}
       \begin{tabular}{lcccc}
         \textbf{Zeichen a}   &   \textbf{A}     & \textbf{B}    & \textbf{C}   & \textbf{D}   \\
         \hline
         Wahrscheinlichkeit $p_{a}$ in Quelle $3$    &      $0.5$          &      $0.25$      &      $0.125$     & $0.125$ \\
         Kodierung $c_{2}$                                       &      $0$             &      $10$      &      $110$     & $111$ \\
         Wortl\"ange                                               &      $1$              &      $2$        &      $3$         & $3$ \\
         \hline
       \end{tabular}
       \bspar1
       Durchschnittliche Wortl\"ange $L_{c_{2}}=0.5*1+0.25*2+0.125*3+0.125*3=1.75$\\
       Informationsgehalt:  $H_{3}=0.5*1+0.25*2+0.125*3+0.125*3=1.75$
     \end{center}
    \bspar3
    $\Rightarrow$ Eine gute Kodierung ber�cksichtigt immer die Verteilung der Zeichen in der Nachrichtenquelle\\
    $\Rightarrow$ Mit Wissen �ber Medieneigenschaften und wie sie wahrgenommen werden, k�nnen gute Kodierungen erreicht werden (JPEG, GIF, MP3 etc.)
  \end{bsslide}
%%%%%%%%%%%%%%%%%%%%%%%%%%%%%%%%%%%%%%%%%%%%%%%%%%%%%%%%%%%%%%%%%%%%%%%%
\renewcommand{\textbufferB}{Zusammenfassung}
  \begin{bsslide}[\textbufferB]
    \colortext{Informationstheorie}
    \bspar1
    \begin{itemize}
      \item Wir wissen nun was Information ist (Zeichenvorrat, Nachricht, Kodierung)
      \item Wir k�nnen den Informationsgehalt einer Nachricht bestimmen
      \item Wir k�nnen feststelle ob Nachrichten redundante Information enthalten
    \end{itemize}
    \bspar4
    \centering Gibt es eine M�glichkeit, optimale Codes zu erzeugen und beliebige Kodierungen zu "`Komprimieren"'?
  \end{bsslide}
%%%%%%%%%%%%%%%%%%%%%%%%%%%%%%%%%%%%%%%%%%%%%%%%%%%%%%%%%%%%%%%%%%%%%%%%
\begin{bsslide}[Bibliographie]
\bspar4
\begin{itemize}
\item \textbf{Malaka, Butz, Hussmann (2009)} -
  Medieninformatik: Eine Einf�hrung (Pearson Studium - IT), Kapitel
  2.4
\item \textbf{Daugmann, John (2012)}- Lecture Notes on Information Theory and Coding, chapter 3\\ \url{http://www.cl.cam.ac.uk/teaching/0809/InfoTheory/InfoTheoryLectures.pdf}
\end{itemize}
\end{bsslide}
 %%%%%%%%%%%%%%%%%%%%%%%%%%%%%%%%%%%%%%%%%%%%%%%%%%%%%%%%%%%%%%%%%%%%%%%%
 %%%%%%%%%%%%%%%%%%%%%%%%%%%%%%%%%%%%%%%%%%%%%%%%%%%%%%%%%%%%%%%%%%%%%%%%
 %%%%%%%%%%%%%%%%%%%%%%%%%%%%%%%%%%%%%%%%%%%%%%%%%%%%%%%%%%%%%%%%%%%%%%%%
%%% Local Variables:
%%% mode: latex
%%% TeX-master: "part-de-mt-informationstheorie"
%%% End:

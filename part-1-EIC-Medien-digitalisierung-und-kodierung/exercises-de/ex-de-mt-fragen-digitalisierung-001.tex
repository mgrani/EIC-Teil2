

\begin{exercise}{Digitalisierung - Allgemeine Fragen}
\label{ex-de-mt-digitalisierung-fragen}
Beantworten sie folgende Fragen:
\begin{enumerate}
\item Erkl�ren Sie den Unterschied zwischen analogen und digitalen Signalen?
\item Wie erfolgt die Abtastung eines Signals?
\item Geben sie Beispiele f�r die Abtastung von Audio bzw. Bildsignalen hinsichtlich der Bezug- und Wertachsen.
\item Worin liegt der Vorteil/Nachteil digitaler Signale?
\item Was ist das Abtasttheorem und welcher Effekt entsteht bei der Verletzung von diesem?
\item Geben Sie Beispiele f�r Aliasing-Effekte aus dem Audio und Bildbereich an und skizzieren sie deren Entstehung.
\item Was sind Fourier-Reihen?
\item Was sind Sinusoide Funktionen und wie unterscheidet sich diese Repr�sentation von einer Repr�sentation in der komplexen Ebene? Welche Effekte hat dies auf die Interpretation der Fourier-Transformation?
\item Was ist das Fourier-Spektrum?
\item Wie unterscheidet sich die diskrete Fourier Transformation von der kontinuierlichen Fourier Transformation?
\item Was ist der Leck-Effekt?
\item Wozu ben�tigt man Fenster in der Diskreten-Fourier-Transformation?
\item Was ist ein Spectrogramm?
\answer{[Siehe Vorlesungsfolien}
\end{enumerate}
\end{exercise}


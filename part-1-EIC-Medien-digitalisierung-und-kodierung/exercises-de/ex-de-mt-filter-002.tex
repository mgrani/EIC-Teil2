
\begin{exercise}{Filter -Praxisbeispiel }
\label{ex-de-mt-filter-1}
Gegeben seien zwei diskrete Signale $x_1(t)$ und  $x_2(t)$ mit folgenden
Werten in den Sekunden $1-19$:\\
\begin{center}
  \begin{tabular}{ccc}
    Zeit[s]& $x_1(t)$    & $x_2(t)$   \\
    \hline
    1    & 0,95  & 0,95  \\
    2    & 0,59  & -0,59 \\
    3    & -0,59 & -0,59 \\
    4    & -0,95 & 0,95  \\
    5    & 0,00  & 0,00  \\
    6    & 0,95  & -0,95 \\
    7    & 0,59  & 0,59  \\
    8    & -0,59 & 0,59  \\
    9    & -0,95 & -0,95 \\
    10   & 0,00  & 0,00  \\
    11   & 0,95  & 0,95  \\
    12   & 0,59  & -0,58 \\
    13   & -0,59 & -0,59 \\
    14   & -0,95 & 0,95  \\
    15   & 0,00  & 0,01  \\
    16   & 0,95  & -0,95 \\
    17   & 0,59  & 0,58  \\
    18   & -0,58 & 0,59  \\
    19   & -0,95 & -0,95 \\
  \end{tabular}
\end{center}
\begin{itemize}
\item Laden sie die Werte in ein Spreadsheet Programm wie
  z.B. Excel. Analysieren Sie die beiden Signale und errechnen Sie das
  Summensignal $s(t)=x_1(t)+x_2(t)$. Was kann man \"uber die Eigenschaften der
  Signale sagen?
\item  Realisieren Sie einen Mittelwertefilter 1. Ordnung. Welches
  Signal k\"onnen Sie aus $s(t)$ mit diesem Filter rekonstruieren und
  wie gut ist die Rekonstruktion?
  Vergleichen Sie verschiedene Filterkoeffizienten miteinander.
\item  Realisieren Sie einen Mittelwertefilter 2. Ordnung. Welches
  Signal k\"onnen Sie aus $s(t)$ mit diesem Filter rekonstruieren?
  Vergleichen Sie verschiedene Filterkoeffizienten miteinander und
  \"uberlegen Sie, wie man eine Phasenverz\"ogerung verhindern oder
  erzwingen kann. \"Andert sich die Ordnung des Filters?
\item  Realisieren Sie einen Hochpass 1. Ordnung, sodass $x_2(t)$ aus $s(t)$
  rekonstruiert werden kann.  Vergleichen Sie verschiedene Filterkoeffizienten miteinander und
 erkl\"aren Sie die Funktionsweise des Hochpass.
\item  Realisieren Sie einen IIR Filter 1. Ordnung.  Vergleichen Sie
  verschiedene Filterkoeffizienten miteinander und zeigen Sie die
  Stabilit\"atsprobleme von IIR Filtern.
\end{itemize}
\answer{see filter-excel.xls}
\end{exercise}